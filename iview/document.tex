\documentclass{bioinfo}
\copyrightyear{2013}
\pubyear{2013}

\begin{document}
\firstpage{1}

\title[iview]{iview: Interactive WebGL Visualizer of Protein-Ligand Complex}
\author[Hongjian Li \textit{et~al}]{Hongjian Li\footnote{to whom correspondence should be addressed}, Kwong-Sak Leung and Man-Hon Wong}
\address{Department of Computer Science and Engineering, Chinese University of Hong Kong, Shatin, New Territories, Hong Kong, China}

\history{Received on XXXXX; revised on XXXXX; accepted on XXXXX}

\editor{Associate Editor: XXXXXXX}

\maketitle

\begin{abstract}

\section{Motivation:}

\section{Results:}

\section{Availability:}
http://istar.cse.cuhk.edu.hk/iview

\section{Contact:} \href{JackyLeeHongJian@Gmail.com}{JackyLeeHongJian@Gmail.com}
\end{abstract}

\section{Introduction}

The output of idock on istar include docked conformations in PDBQT format. At present, users have to download them and visualize them using their favorite molecular visualizers, such as PyMOL \citep{1221}, Chimera \citep{1219}, VMD \citep{1220}, AutoDockTools4 \citep{596}, ViewDock TDW \citep{559}, PoseView \citep{748} and LigPlot+ \citep{951}.

There are web tools for the visualization of bioinformatics data. They are based on either Java applets, Flash, or HTML5. For example, Jmol is an open-source Java viewer for chemical structures in 3D \citep{1263}. Jmol has been widely used and recognized as the \textit{de facto} molecular viewer on the web. ChemDoodle Web Components is a pure Javascript chemical graphics and cheminformatics library \citep{1264}. It allows the wielder to present publication quality 2D and 3D graphics and animations for chemical structures, reactions and spectra.  ePlant is a suite of open-source Flash-based web tools for visualizing integrative systems biology from the model organism \textit{Arabidopsis thaliana} \citep{1242}. iCanPlot is a visualizer of high-throughput omics data using interactive HTML5 canvas plotting \citep{1028}. EvolView is a web application for visualizing, annotating and managing phylogenetic trees \citep{1241}. VLPC (Virtual Laboratory Practical Class) is a HTML5 pharmacology virtual laboratory \citep{1249}. However, there is no protein-ligand complex visualizer based on HTML5. The creation of interactive 3D visualization schemes on the web platform using WebGL is presented for scientific data produced by molecular and cellular biology research \citep{1262}. PLI \citep{1288} is a web-based tool for the comparison of protein-ligand interactions observed on PDB structures. BioJS \citep{1308} is an open source JavaScript framework for biological data visualization.

We propose iview, an interactive HTML5 visualizer of protein-ligand complex (Figure \ref{istar:iview}). Users shall be able to view the protein-ligand docking results directly on istar. Typical functionalities include zooming, dragging, and plotting hydrogen bonds.

\section{Approach}


\begin{methods}
\section{Methods}

EDTSurf \citep{1297} is an fast algorithm to generating triangulated macromolecular surfaces by Euclidean distance transform.
Protein formats: PDB, PDBQT
Ligand formats: PDB, PDBQT, MOL2, SDF, XYZ

\end{methods}

\begin{figure}[!tpb]%figure1
%\centerline{\includegraphics{fig01.eps}}
\caption{Caption, caption.}\label{fig:01}
\end{figure}

\section{Conclusion}

We have developed iview, an interactive HTML5 visualizer of real-time protein-ligand docking.

\bibliographystyle{natbib}
%\bibliographystyle{achemnat}
%\bibliographystyle{plainnat}
%\bibliographystyle{abbrv}
%\bibliographystyle{bioinformatics}
%
%\bibliographystyle{plain}

\bibliography{../refworks}

\end{document}
