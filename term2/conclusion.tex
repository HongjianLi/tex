\chapter{Conclusion}

Drug discovery is an expensive and long-term business. It takes about US\$1.8 billion over 13.5 years to develop a new drug \citep{716}. Hence computer-aided drug discovery is not only timesaving, but economics tells us this is the way we should be going. We aim to develop a comprehensive computational framework for structure-based drug discovery, simulating the early phases of modern drug discovery pipeline.

We have developed idock for fast prediction of both binding conformations of small compounds against given proteins and their binding affinities. idock can shortlist a few promising compounds out of millions for further clinical investigations. idock 1.5 is capable of docking 16 ligands per minute on a high performance machine, outperforming AutoDock Vina \citep{595} by at least 8.69 times and at most 37.51 times in terms of docking speed. Nevertheless, when it comes to docking millions of ligands, it still costs several months. Even faster algorithms and implementations are of great need.

We have also developed istar to facilitate automatic virtual screening. 

We propose idock 2.0 to incorporate GPU acceleration.

We propose idock 3.0 to incorporate click chemistry.

We have applied idock to address drug discovery problems in practice. As case studies, we are collaborating with 

\chapterend
