%%%%%%%%%%%%%%%%%%%%%%% file template.tex %%%%%%%%%%%%%%%%%%%%%%%%%
%
% This is a general template file for the LaTeX package SVJour3
% for Springer journals.          Springer Heidelberg 2010/09/16
%
% Copy it to a new file with a new name and use it as the basis
% for your article. Delete % signs as needed.
%
% This template includes a few options for different layouts and
% content for various journals. Please consult a previous issue of
% your journal as needed.
%
%%%%%%%%%%%%%%%%%%%%%%%%%%%%%%%%%%%%%%%%%%%%%%%%%%%%%%%%%%%%%%%%%%%

\RequirePackage{fix-cm}

%\documentclass{svjour3}                     % onecolumn (standard format)
%\documentclass[smallcondensed]{svjour3}     % onecolumn (ditto)
%\documentclass[smallextended]{svjour3}       % onecolumn (second format)
\documentclass[twocolumn]{svjour3}          % twocolumn

\smartqed  % flush right qed marks, e.g. at end of proof

\usepackage{graphicx}

% \usepackage{mathptmx}      % use Times fonts if available on your TeX system

% insert here the call for the packages your document requires
%\usepackage{latexsym}

% please place your own definitions here and don't use \def but \newcommand{}{}

\journalname{J Comput Aided Mol Des}

\begin{document}

\title{USRT: a de novo Ligand Deduplication and Clustering Algorithm Based on Ultrafast Shape Recognition
%\thanks{Grants or other notes about the article that should go on the front page should be placed here. General acknowledgments should be placed at the end of the article.}
}
%\subtitle{Do you have a subtitle?\\ If so, write it here}

%\titlerunning{Short form of title}        % if too long for running head

\author{Hongjian Li \and Kwong-Sak Leung \and Man-Hon Wong}

%\authorrunning{Short form of author list} % if too long for running head

\institute{Hongjian Li \and Kwong-Sak Leung \and Man-Hon Wong\at
Department of Computer Science and Engineering, Chinese University of Hong Kong, Shatin, New Territories, Hong Kong\\
\email{hjli@cse.cuhk.edu.hk}           %  \\
%\emph{Present address:} of F. Author  %  if needed
%\and
%S. Author \at
%second address
}

\date{Received: date / Accepted: date}
% The correct dates will be entered by the editor

\maketitle

\begin{abstract}

We present the first,
ultrafast
can combine with other USR variants, e.g. USRCAT.

\keywords{de novo ligand design \and deduplication \and clustering \and shape recognition}
% \PACS{PACS code1 \and PACS code2 \and more}
% \subclass{MSC code1 \and MSC code2 \and more} % mathematical subject classification numbers
\end{abstract}

\section{Introduction}
\label{intro}

Given a target protein, hundreds of thousands of ligands are usually docked to find out which one has the highest binding affinity. This method is known as virtual screening. However, the entire drug-like space may contain as many as 10 to the power of 100 molecules, so it is impossible to dock all the available ligands. The de novo ligand design strategy is emerging as a complementary method.

$10^{60}$ drug-like molecules \cite{} % Review LEA3D paper

AutoGrow \cite{466}, AutoGrow 3.0 \cite{1354}, iSyn \cite{}. They use genetic algorithm. Operators include, mutation, addition, crossover, cutting, selection, etc. Goal is to design ligands that have higher binding affinities. Typical method is to grow an initial scaffold by adding fragments. Any of these types of operators could possibly lead to the generation of duplicate ligands.

In the GA operators, duplicate ligands are occasionally generated. Can be over 50\%. Elitism and fitness porportion in GA not work. Illustrate with two figures, one from iSynMCB, one from crossover (same moieties from different parent ligands, resulting from random mixing, Swap the chemical moieties of known ligands. Link the scaffold and fragment through their respective linker hydrogen atoms).

Selection operator done by igrow externally calls idock to evaluate the binding affinity of a population of de novo ligands. It is particularly challenging to tell if two ligands are duplicate when they are in significantly differnent conformations after being docked against the protein (the selection process) (Figure \ref{}).

\begin{figure*}
%\includegraphics[width=\textwidth]{MRV.png}
\caption{Two docked poses of the same ligand. CADD.pptx page 5, poses generated by idock \cite{1153}. figure rendered by iview \cite{1366}, green cubic box depicts the binding cavity on the protein surface. CCR5 in complex with MRV. PDB ID: 4MBS. Add more detail from PDB.org.}
\label{fig:2}
\end{figure*}

In iSyn \cite{}, we used USR \cite{1379}. Much faster, USR finds its applications in retrospective \cite{1332} and prospective \cite{} virtual screening. USR is independent of position and orientation, but is dependent on torsions. % \cite{1280}, \cite{1333}

USR variants include USRCAT \cite{1331}, \cite{1335}, lipophilicity into ElectroShape \cite{1338}. Cannot address the duplication issue. One can use ensemble methods like comparing mwt, nha in addition to USR. These are coarse.

We present USRT, the first algorithm.

In the following sections, we describe USR and USRT, their execution time.

\section{Methods}
\label{sec:methods}

This section introduces USR, USRT.

\subsection{USR: Ultrafast Shape Recognition}
\label{sec:usr}

USR \cite{1379}

\begin{equation}
a^2+b^2=c^2
\label{eqn:cyscore}
\end{equation}

\subsection{USRT: Ultrafast Shape Recognition with Torsions}
\label{sec:usrt}

PDBQT format, used by AutoDock series \cite{597,596}, Vina \cite{595}, idock \cite{1153}, QVina \cite{1193}. Explain the PDBQT specification.

Figure \ref{} shows two conformations with different torsions. This is a rotatable bond. This branch can rotate along the rotatable bond and flip to the right hand side. A torsion is the rotating angle, so it is in the range of –π to π. 

\begin{figure*}
%\includegraphics[width=\textwidth]{example.png}
\caption{Figure with T27 and T27 with two torsions. T27 is still too complicated. Use half of T27.}
\label{fig:2}
\end{figure*}

Reference atom is chosen to be the atom connecting to the parent frame. It is BRANCH X Y (Figure \ref{}), and is often the first of the current frame. For the ROOT frame, it is the first.

\begin{figure*}
%\includegraphics[width=\textwidth]{example.png}
\caption{Figure of PDBQT lines in Notepad++}
\label{fig:2}
\end{figure*}

Once a reference atom is determined, the next is to compute inter-atom distances and their 1st, 2nd, 3rd moments.

Frame with less than 2 heavy atoms are ignored.

\section{Results}

\begin{table}
\caption{USRT output feature vector of the nine poses of the same ligand}
\label{tab:1}
\begin{tabular}{rrr}
\hline\noalign{\smallskip}
pose & element 1 & element 2\\
\noalign{\smallskip}\hline\noalign{\smallskip}
1 & 0.2332 & 0.6982\\
2 & 0.2332 & 0.6982\\
\noalign{\smallskip}\hline
\end{tabular}
\end{table}

\subsection{Execution Time Comparison}

USR was claimed to be three orders of magnitude faster than EShape3D.

\section{Discussion}

%(see Sect.~\ref{sec:methods}).

%\paragraph{Paragraph headings} Use paragraph headings as needed.

USR has advantages in being independent of position and orientation.

USRT inherits advantages from USR, and is also independent of torsions, which are introduced by flexible ligand docking.

Output is a feature vector, which maps to a point in a high dimensional space. All existing clustering algorithms can be utilized.

Software availability.

Emphasize computational efficiency, 3 orders of magnitude faster than EShape3D, and applicable to large-scale ligand database like istar \cite{1362}, which has collected 23 million ligands from ZINC \cite{532,1178}.

The current version of USRT has some constraints: known ROOT frame, in-frame atom types, connector atom types. We will address the above limitations in future research.

\section{Availability}

USRT is free and open source under Apache License 2.0. It is written in C++ and available at https://github.com/HongjianLi/usrt. Precompiled executables for 64-bit Linux and Windows are provided. Use cases and API documentations are also provided.

\section{Conclusion}

We have developed USRT (Ultrafast Shape Recognition with Torsions), the first algorithm that can distinguish ligands with different torsions. USRT is computationally very fast, independent of torsions, compatible with USRCAT and other USR variants.
A general method that can be applied to various drug design applications, including de novo ligand deduplication and clustering.

\section{Author Contributions}

H.L. designed the study, implemented the software, ran the experiments, and wrote the manuscript. All authors discussed results and commented on the manuscript.

\begin{acknowledgements}

This work was supported by the Direct Grant from the Chinese University of Hong Kong and the GRF Grant (Project No. 2150764) from the Research Grants Council of Hong Kong SAR.

\end{acknowledgements}

\bibliographystyle{spbasic}      % basic style, author-year citations
%\bibliographystyle{spmpsci}      % mathematics and physical sciences
%\bibliographystyle{spphys}       % APS-like style for physics
\bibliography{../refworks}   % name your BibTeX data base

\end{document}

%Journal of Computer-Aided Molecular Design 3.172
%Chemical Biology & Drug Design 2.469 
%Molecular Informatics 2.338
%Journal of Molecular Graphics and Modelling 2.325
%IEEE-ACM Transactions on Computational Biology and Bioinformatics 1.616

%Suggested reviewers
%John J. Irwin, UCSF, jir322@gmail.com
%Jacob Durrant, UCSD, jdurrant@ucsd.edu
%Adrian Schreyer, Department of Biochemistry, University of Cambridge, UK, adrian@schreyer.me
%Matthew R. Reynolds, AIDS Vaccine Research Laboratory, 555 Science Dr., Madison, Wisconsin 53711, mrreynol@wisc.edu
%Masaaki Toyama, Center for chronic viral diseases, Kagoshima University, Kagoshima, Japan, toyama@m2.kufm.kagoshima-u.ac.jp
%Pablo Campomanes, German Research School for Simulation Sciences, p.campomanes@grs-sim.de
%Calvin Yu-Chian Chen, School of Medicine, College of Medicine, China Medical University, Taichung, 540402, Taiwan, ycc929@mit.edu
