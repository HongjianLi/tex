\documentclass[12pt, conference, compsocconf]{../IEEEtran}
\usepackage{xltxtra}
\usepackage{subfig}
\usepackage{booktabs}
\usepackage{flushend}
\usepackage[numbers,sort&compress]{natbib}
\setmainfont{Times New Roman}

\begin{document}

\title{Graph Reachability Queries: A Mini State-of-the-Art Survey}
\author
{
\IEEEauthorblockN
{
Hongjian Li
\IEEEauthorblockA
{
Department of Computer Science and Engineering\\
Chinese University of Hong Kong\\
hiji@cse.cuhk.edu.hk
}
}
}
\maketitle

\begin{abstract}

Background
Main focus
Discussion
Future

\end{abstract}

%\begin{IEEEkeywords}

%Reachability query, regular expression, graph segmentation, compression, label constraint, probabilistic query, tree decomposition, neural network

%\end{IEEEkeywords}

\section{Introduction}

Given two vertices $u$ and $v$ in a directed graph $G = (V, E)$, reachability queries ask whether there is a path from $u$ to $v$ in G?

Where the problem comes from?

Why it is important?
Reachability in massive graphs has always been a challenging question partly because neither online query processing nor index pre-computation offer a pragmatic solution for this well researched problem. But its applicability in many diverse areas, especially in biological domain (such as metabolic pathways, protein protein interaction networks), has elevated the need for an efficient reachability answering method that not only supports fast query response, but also requires moderate amount of indexing.

analysis of social networks, model
checking and route planning
program analysis [15, 14], for
the purpose of compiler optimisations or to find bugs


\section{Existing Methods}

Comparison
Have your own evaluation
Use others results and give reasons

The cost of reachability query computation using traditional
algorithms such as depth first search or transitive closure has
been found to be prohibitive and unacceptable in massive
graphs such as biological interaction networks, or pathways.
Contemporary solutions mainly take two distinct approaches
- precompute reachability in the form of transitive closure
(trade space for time) or use state space search (trade time
for space). A middle ground among the two approaches has
recently gained popularity. It precomputes part of the reach-
ability information as a complex index so that most queries
can be answered within a reasonable time. In this approach,
the main cost now is creation of the index, and response gen-
eration using it as well as the space needed to materialize the
structure. Most contemporary solutions favor a combination
of these costs to be efficient for a class of applications.


\citep{1066} path tree

\citep{1067} 3-HOP

\citep{1063} is state-of-the-art review in 2010.

In more and more graphs, edges may bear different types, representing different kinds of relationships. \Citeauthor{1052} propose a class of reachability queries in which an edge comes along with a regular expression of a certain form \citep{1052}. They develop an algorithm for answering reachability queries in quadratic time.

\Citeauthor{1053} make use of graph segmentation and propose a hierarchical index to reduce index size, while at the same time retain query efficiency \citep{1053}. They show that index rebuilding is not required for a large class of updates, distinguishing the index itself from all other contemporary methods.

\Citeauthor{1054} propose a new variant of bit vector compression scheme termed Partitioned Word-Aligned Hybrid (PWAH) by introducing word partitions, resulting in a more compact data structure than interval lists \citep{1054}. They observe that when computing transitive closure, reachable vertices tend to cluster together.

\citep{1055} given a label set S and two vertices u1 and u2 in a large directed graph G, we verify whether there exists a path from u1 to u2 under label constraint S. Several techniques are proposed in this paper to minimize the search space in computing path-label transitive closure.

\citep{1057} introduce a novel tree-based index framework which utilizes the directed maximal weighted spanning tree algorithm and sampling techniques to maximally compress the generalized transitive closure for the labeled graphs.

\citep{1056} each edge in a graph could be associated with a probability to appear. to determine if a source vertex could reach a destination vertex with probabilty larger than a user specified probability value t.

\citep{1058} present a novel indexing method based on the concept of tree decomposition. 

\citep{1059} GRAIL, that is based on the idea of randomized interval labeling. GRAIL is the only index that can scale to millions of nodes and edges.

\citep{1060} propose a new indexing scheme, called Path-Hop, which is even more space-efficient than those schemes based on 2-Hop cover and yet has query processing speed comparable to those chain/tree covers. 

\citep{1061} propose a hierarchical prediction framework, based on neural networks and a set of graph features and a knowledge base on past predictions, to select the optimal index for a graph database.

\section{Discussion}

Summary on current works
  Avoid overlapping with existing reviews

\section{Prospectives}

What will be done in the future? (Further emphasis on large graph)
  Consequences of current breakthroughs
  Bottleneck in the whole picture
    No universal algorithm and structure

Data structures that are more compact and more efficient for reachability queries.

\bibliographystyle{unsrtnat}
\bibliography{../refworks}

\end{document}
