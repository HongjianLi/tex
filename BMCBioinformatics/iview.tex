%% BioMed_Central_Tex_Template_v1.06
%%                                      %
%  bmc_article.tex            ver: 1.06 %
%                                       %

%%IMPORTANT: do not delete the first line of this template
%%It must be present to enable the BMC Submission system to
%%recognise this template!!

%%%%%%%%%%%%%%%%%%%%%%%%%%%%%%%%%%%%%%%%%
%%                                     %%
%%  LaTeX template for BioMed Central  %%
%%     journal article submissions     %%
%%                                     %%
%%          <8 June 2012>              %%
%%                                     %%
%%                                     %%
%%%%%%%%%%%%%%%%%%%%%%%%%%%%%%%%%%%%%%%%%


%%%%%%%%%%%%%%%%%%%%%%%%%%%%%%%%%%%%%%%%%%%%%%%%%%%%%%%%%%%%%%%%%%%%%
%%                                                                 %%
%% For instructions on how to fill out this Tex template           %%
%% document please refer to Readme.html and the instructions for   %%
%% authors page on the biomed central website                      %%
%% http://www.biomedcentral.com/info/authors/                      %%
%%                                                                 %%
%% Please do not use \input{...} to include other tex files.       %%
%% Submit your LaTeX manuscript as one .tex document.              %%
%%                                                                 %%
%% All additional figures and files should be attached             %%
%% separately and not embedded in the \TeX\ document itself.       %%
%%                                                                 %%
%% BioMed Central currently use the MikTex distribution of         %%
%% TeX for Windows) of TeX and LaTeX.  This is available from      %%
%% http://www.miktex.org                                           %%
%%                                                                 %%
%%%%%%%%%%%%%%%%%%%%%%%%%%%%%%%%%%%%%%%%%%%%%%%%%%%%%%%%%%%%%%%%%%%%%

%%% additional documentclass options:
%  [doublespacing]
%  [linenumbers]   - put the line numbers on margins

%%% loading packages, author definitions

\documentclass[twocolumn]{bmcart}% uncomment this for twocolumn layout and comment line below
%\documentclass{bmcart}

%%% Load packages
%\usepackage{amsthm,amsmath}
%\RequirePackage{natbib}
%\RequirePackage{hyperref}
\usepackage[utf8]{inputenc} %unicode support
%\usepackage[applemac]{inputenc} %applemac support if unicode package fails
%\usepackage[latin1]{inputenc} %UNIX support if unicode package fails


%%%%%%%%%%%%%%%%%%%%%%%%%%%%%%%%%%%%%%%%%%%%%%%%%
%%                                             %%
%%  If you wish to display your graphics for   %%
%%  your own use using includegraphic or       %%
%%  includegraphics, then comment out the      %%
%%  following two lines of code.               %%
%%  NB: These line *must* be included when     %%
%%  submitting to BMC.                         %%
%%  All figure files must be submitted as      %%
%%  separate graphics through the BMC          %%
%%  submission process, not included in the    %%
%%  submitted article.                         %%
%%                                             %%
%%%%%%%%%%%%%%%%%%%%%%%%%%%%%%%%%%%%%%%%%%%%%%%%%


\def\includegraphic{}
\def\includegraphics{}



%%% Put your definitions there:
\startlocaldefs
\endlocaldefs


%%% Begin ...
\begin{document}

%%% Start of article front matter
\begin{frontmatter}

\begin{fmbox}
\dochead{Software}

%%%%%%%%%%%%%%%%%%%%%%%%%%%%%%%%%%%%%%%%%%%%%%
%%                                          %%
%% Enter the title of your article here     %%
%%                                          %%
%%%%%%%%%%%%%%%%%%%%%%%%%%%%%%%%%%%%%%%%%%%%%%

\title{iview: an interactive WebGL visualizer for protein-ligand complex}

%%%%%%%%%%%%%%%%%%%%%%%%%%%%%%%%%%%%%%%%%%%%%%
%%                                          %%
%% Enter the authors here                   %%
%%                                          %%
%% Specify information, if available,       %%
%% in the form:                             %%
%%   <key>={<id1>,<id2>}                    %%
%%   <key>=                                 %%
%% Comment or delete the keys which are     %%
%% not used. Repeat \author command as much %%
%% as required.                             %%
%%                                          %%
%%%%%%%%%%%%%%%%%%%%%%%%%%%%%%%%%%%%%%%%%%%%%%

\author[
   addressref={aff1},                   % id's of addresses, e.g. {aff1,aff2}
   corref={aff1},                       % id of corresponding address, if any
%   noteref={n1},                        % id's of article notes, if any
   email={jackyleehongjian@gmail.com}   % email address
]{\inits{HL}\fnm{Hongjian} \snm{Li}}
\author[
   addressref={aff1},                   % id's of addresses, e.g. {aff1,aff2}
   email={ksleung@cse.cuhk.edu.hk}
]{\inits{KSL}\fnm{Kwong-Sak} \snm{Leung}}
\author[
   addressref={aff2},
%   corref={aff2},                       % id of corresponding address, if any
   email={nakane.t@gmail.com}
]{\inits{TN}\fnm{Takanori} \snm{Nakane}}
\author[
   addressref={aff1},                   % id's of addresses, e.g. {aff1,aff2}
   email={mhwong@cse.cuhk.edu.hk}
]{\inits{MHW}\fnm{Man-Hon} \snm{Wong}}

%%%%%%%%%%%%%%%%%%%%%%%%%%%%%%%%%%%%%%%%%%%%%%
%%                                          %%
%% Enter the authors' addresses here        %%
%%                                          %%
%% Repeat \address commands as much as      %%
%% required.                                %%
%%                                          %%
%%%%%%%%%%%%%%%%%%%%%%%%%%%%%%%%%%%%%%%%%%%%%%

\address[id=aff1]{%                           % unique id
  \orgname{Department of Computer Science and Engineering, Chinese University of Hong Kong}, % university, etc
%  \street{Waterloo Road},                     %
  %\postcode{}                                % post or zip code
  \city{Hong Kong},                              % city
  \cny{China}                                    % country
}
\address[id=aff2]{%
  \orgname{Graduate School of Medicine, Kyoto University},
%  \street{D\"{u}sternbrooker Weg 20},
%  \postcode{24105}
  \city{Kyoto},
  \cny{Japan}
}

%%%%%%%%%%%%%%%%%%%%%%%%%%%%%%%%%%%%%%%%%%%%%%
%%                                          %%
%% Enter short notes here                   %%
%%                                          %%
%% Short notes will be after addresses      %%
%% on first page.                           %%
%%                                          %%
%%%%%%%%%%%%%%%%%%%%%%%%%%%%%%%%%%%%%%%%%%%%%%

\begin{artnotes}
%\note{Sample of title note}     % note to the article
%\note[id=n1]{Equal contributor} % note, connected to author
\end{artnotes}

%\end{fmbox}% comment this for two column layout

%%%%%%%%%%%%%%%%%%%%%%%%%%%%%%%%%%%%%%%%%%%%%%
%%                                          %%
%% The Abstract begins here                 %%
%%                                          %%
%% Please refer to the Instructions for     %%
%% authors on http://www.biomedcentral.com  %%
%% and include the section headings         %%
%% accordingly for your article type.       %%
%%                                          %%
%%%%%%%%%%%%%%%%%%%%%%%%%%%%%%%%%%%%%%%%%%%%%%

\begin{abstractbox}

\begin{abstract} % abstract
\parttitle{Background} %if any
Visualization of protein-ligand complex plays an important role in elaborating protein-ligand interactions and aiding novel drug design. Most existing web visualizers either rely on slow software rendering, or lack virtual reality support. The vital feature of macromolecular surface construction is also unavailable.

\parttitle{Results} %if any
We have developed iview, an easy-to-use interactive WebGL visualizer of protein-ligand complex. It exploits hardware acceleration rather than software rendering. It features three special effects in virtual reality settings, namely anaglyph, parallax barrier and oculus rift, resulting in visually appealing identification of intermolecular interactions. It supports four surface representations including Van der Waals surface, solvent excluded surface, solvent accessible surface and molecular surface. Moreover, based on the feature-rich version of iview, we have also developed a neat and tailor-made version specifically for our istar web platform for protein-ligand docking purpose. This demonstrates the excellent portability of iview.

\parttitle{Conclusions} %if any
Using innovative 3D techniques, we provide a user friendly visualizer that is not intended to compete with professional visualizers, but to enable easy accessibility and platform independence.
\end{abstract}

%%%%%%%%%%%%%%%%%%%%%%%%%%%%%%%%%%%%%%%%%%%%%%
%%                                          %%
%% The keywords begin here                  %%
%%                                          %%
%% Put each keyword in separate \kwd{}.     %%
%%                                          %%
%%%%%%%%%%%%%%%%%%%%%%%%%%%%%%%%%%%%%%%%%%%%%%

\begin{keyword}
\kwd{structural bioinformatics}
\kwd{visualization}
\kwd{molecular docking}
\end{keyword}

% MSC classifications codes, if any
%\begin{keyword}[class=AMS]
%\kwd[Primary ]{}
%\kwd{}
%\kwd[; secondary ]{}
%\end{keyword}

\end{abstractbox}
%
\end{fmbox}% uncomment this for twcolumn layout

\end{frontmatter}

%%%%%%%%%%%%%%%%%%%%%%%%%%%%%%%%%%%%%%%%%%%%%%
%%                                          %%
%% The Main Body begins here                %%
%%                                          %%
%% Please refer to the instructions for     %%
%% authors on:                              %%
%% http://www.biomedcentral.com/info/authors%%
%% and include the section headings         %%
%% accordingly for your article type.       %%
%%                                          %%
%% See the Results and Discussion section   %%
%% for details on how to create sub-sections%%
%%                                          %%
%% use \cite{...} to cite references        %%
%%  \cite{koon} and                         %%
%%  \cite{oreg,khar,zvai,xjon,schn,pond}    %%
%%  \nocite{smith,marg,hunn,advi,koha,mouse}%%
%%                                          %%
%%%%%%%%%%%%%%%%%%%%%%%%%%%%%%%%%%%%%%%%%%%%%%

%%%%%%%%%%%%%%%%%%%%%%%%% start of article main body
% <put your article body there>

%%%%%%%%%%%%%%%%
%% Background %%
%%
\section*{Introduction}
Visualization of protein-ligand complex plays an important role in elaborating protein-ligand interactions and aiding novel drug design. To date, dozens of visualization tools already exist. VMD \cite{1220}, PyMOL \cite{1221} and Chimera \cite{1219} are very well-known and highly cited. They can interpret multiple file formats and generate multiple representations to supply precise and powerful control. AutoDockTools4 \cite{596} provides native support for the PDBQT file format, which is widely used in various protein-ligand docking software such as AutoDock \cite{596}, AutoDock Vina \cite{595}, and our idock \cite{1153}. We also developed our own method \cite{1265} to visualize structures in virtual reality settings and employ fragment-based \textit{de novo} ligand design strategy for interactive drug design. PoseView \cite{748} and LigPlot+ \cite{951}, on the other hand, plot 2D diagrams of protein-ligand interactions from 3D coordinates.

In addition, there are web visualizers based on either Java applet, Adobe Flash, or HTML5 canvas. Jmol (http://www.jmol.org), an open source Java viewer for chemical structures in 3D, has been deployed worldwide and recognized as the \textit{de facto} molecular viewer on the web. GIANT \cite{1359}, a web visualizer based on Jmol, supports analyzing protein-ligand interactions on the basis of patterns of atomic contacts obtained from the statistical analyses of 3D structures. However, Java is being disabled on more and more systems due to security concerns so that Java-free visualizers are highly required. JSmol \cite{1314}, a JavaScript-only version of Jmol, includes the full implementation of the entire set of Jmol functionalities. Although Jmol and JSmol support a large set of advanced features including scripting, they rely on software rendering which is slow on large display areas and thus prevents detailed inspection of the structure. In contrast, WebGL visualizers benefit from GPU acceleration. For instance, ChemDoodle Web Components (http://web.chemdoodle.com), a pure JavaScript chemical graphics and cheminformatics library, presents 2D and 3D graphics and animations for chemical structures, reactions and spectra, but it lacks protein surface construction. GLmol (http://webglmol.sourceforge.jp), a molecular viewer on WebGL/JavaScript using the three.js library, supports multiple file formats and representations, and features an experimental version of surface construction based on the EDTSurf algorithm \cite{1297,1350}. Another study \cite{1262} also presents a WebGL technology for rendering molecular surface using the SpiderGL library \cite{1320}. Nevertheless, none of these WebGL visualizers support virtual reality effects.

Surface representation is a convenient way to visualize protein-ligand interactions. However, macromolecular surface calculation is computationally and memory intensive. Furthermore, the calculated mesh is very complex, often exceeding 500,000 polygons. Therefore its implementation in JavaScript/WebGL was considered to be very difficult. Most existing web visualizers either rely on slow software rendering, or lack virtual reality support. Moreover, the vital feature of protein surface construction is usually unavailable, and the support for PDBQT format is not implemented.

To address the above obstacles, we have developed iview, an interactive WebGL visualizer of protein-ligand complex, featuring three special effects in virtual reality settings and four surface representations (Table \ref{tbl:features}). Furthermore, we show that iview can be easily modified to adapt to different applications. As an application example, we have recently developed a web platform called istar to automate large-scale protein-ligand docking using our idock \cite{1153}. Refactored from the feature-rich version of iview, we have also developed tailor-made version specifically for visualizing docking input data and output results of user-submitted jobs.

\section*{Implementation}
iview is refactored from GLmol 0.47, using three.js as its primary 3D engine with antialiasing support. It is based on WebGL canvas and can be easily integrated into existing HTML5 web pages to display molecular models without requiring Java or browser plugins. It loads a protein-ligand structure from the PDB (Protein Data Bank) \cite{1357} as its data source via a RESTful interface. It renders four standard representations of primary structure, namely line, stick, ball \& stick and sphere, and five standard representations of secondary structure, namely ribbon, strand, cylinder \& plate, C alpha trace and B factor tube. It colors the structure by either atom spectrum, protein chain, protein secondary structure, B factor, residue name, residue polarity, or atom type, by setting the vertex colors of the geometry object of the corresponding representation. It supports user interactions including rotation, translation, zooming and slab with mouse or hand touch manipulation. It provides both perspective and orthographic cameras, and anaglyph, parallax barrier and oculus rift effects from three.js examples for use in a virtual reality environment.

We have ported EDTSurf \cite{1297,1350}, an fast algorithm to generating triangulated macromolecular surfaces by Euclidean distance transform, to JavaScript and integrated it into iview to construct and render in real time four representations of protein surface, namely Van der Waals surface, solvent excluded surface, solvent accessible surface and molecular surface, with opacity and wireframe adjustable by users. Although the JavaScript implementation of the EDTSurf algorithm typically consumes a few seconds and 500MB to 700MB memory for computation, it is sufficiently efficient for practical applications. To limit CPU and memory usage, the calculation grid size is restricted to 180 x 180 x 180.

It is worthwhile to highlight that iview performs all parsing and rendering in the client browser, without any dependency on server side at all, ensuring the data privacy is maintained. This is unlike ChemDoodle Web Components, some of whose functions send data to a dedicated server for processing and wait for retrieval of results.

\section*{Results}

We take as example the CCR5 chemokine receptor-HIV entry inhibitor maraviroc complex \cite{1348} (PDB code: 4MBS).

Figure \ref{fig:ribbon} shows the human CCR5 secondary structure rendered as ribbon, and the ligands rendered as sphere.

Figure \ref{fig:anaglyph} shows the anaglyph effect in a virtual reality environment. When users wear a spectacle with special filters on both sides, the disparity between two superimposed molecules creates a perception of depth, leading to visually more appealing identification of intermolecular interactions.

Figure \ref{fig:surface} shows the protein surface generated by our JavaScript implementation of the EDTSurf algorithm \cite{1297,1350}. The human CCR5 is rendered as molecular surface colored by chain. The marketed HIV drug maraviroc is rendered as stick colored by chain. It can be clearly seen that the asymmetric unit is composed of two complexes, and the CCR5 forms a deep allosteric cavity where maraviroc binds.

We have successfully tested iview in Chrome 30, Firefox 25, Safari 6.1 and Opera 17. Support for IE 11 is experimental because gl\_FrontFacing is unsupported in IE 11. Refer to http://caniuse.com/webgl for compatibility of WebGL support in desktop and mobile browsers.

\section*{Application Example}
We emphasize portability and usability, and illustrate that iview can be easily modified to suit one's particular application, given that iview is free and open source under a permissive license. We take protein-ligand docking as an example. Based on the feature-rich version of iview, our tailor-made version specifically for idock jobs cleans up many dispensable functions, enabling a very neat interface. It only retains the rendering of primary structure of protein and ligand, and the construction of protein surface. Most importantly, it implements new features especially for protein-ligand docking purpose.

In the input phase of a docking job, it merely requires a PDB file, which can be obtained either from the PDB database \cite{1357} or via homology modeling, and then constructs the protein surface asynchronously in a separate web worker to keep the web page responsive. It automatically detects a binding site from the largest co-crystallized ligand first by finding the smallest cubic box that covers the entire ligand and then by extending the box by 50\% in all the three dimensions in order to reserve space for conformational sampling. In case of non-existence of co-crystallized ligand, the binding site is defaulted to the geometric center of the protein. The binding site is visually depicted in the form of a cubic box whose center and size can be manually adjusted by users in real time.

In the output phase of a docking job, it displays the user-supplied cubic box for users to confirm that the predicted ligand conformations do fall inside the desired binding site. Other than PDB format, its parsers are capable of parsing a protein and multiple top hit ligands in PDBQT format used by idock. It displays the top hit ligand IDs in a horizontally scrollable row and provides a straightforward way to switch ligands easily through a button group. It has built-in support for putative intermolecular hydrogen bond detection by finding hydrogen bond donors and acceptors from protein and ligand and setting the distance threshold to 3.5\AA. It lists the docking result files, predicted free energy and binding affinity values, molecular properties, SMILES representation, compound suppliers and annotations, and putative hydrogen bond positions and their lengths, in order to give users a quick overview of the top hit ligands and assist them in making decisions of which compounds to purchase for subsequent wet-lab experiments (Figure \ref{fig:idock}).

\section*{Conclusions}
We have designed and developed iview to be a simple and straightforward way to visualize protein-ligand complex. It enables non-experts to quickly elucidate protein-ligand interactions in a 3D manner. Furthermore, iview is free and open source, and can be easily integrated into any bioinformatics application that requires interactive protein-ligand visualization.

%%%%%%%%%%%%%%%%%%%%%%%%%%%%%%%%%%%%%%%%%%%%%%
%%                                          %%
%% Backmatter begins here                   %%
%%                                          %%
%%%%%%%%%%%%%%%%%%%%%%%%%%%%%%%%%%%%%%%%%%%%%%

\begin{backmatter}

\section*{Availability and requirements}
\textbf{Project name:} iview

\textbf{Project home page:} http://istar.cse.cuhk.edu.hk/iview

\textbf{Operating system:} Platform independent

\textbf{Programming languages}: JavaScript, HTML5, CSS3

\textbf{Other requirements:} Browser and graphics card with WebGL capability

\textbf{License:} Apache License 2.0

%\textbf{Any restrictions to use by non-academics:} licence needed

\section*{Competing interests}
The authors declare that they have no competing interests.

\section*{Author's contributions}
HL and TN developed the presented software. HL drafted the manuscript. TN, KWL and MHW edited the manuscript.% All authors read and approved the final manuscript.

\section*{Acknowledgements}
Takanori Nakane thanks Japan Society for the Promotion of Science (JSPS) for scholarship (DC1). This work was partly supported by Grant-in-Aid for JSPS Fellows (Grant Number 11J04341).

%%%%%%%%%%%%%%%%%%%%%%%%%%%%%%%%%%%%%%%%%%%%%%%%%%%%%%%%%%%%%
%%                  The Bibliography                       %%
%%                                                         %%
%%  Bmc_mathpys.bst  will be used to                       %%
%%  create a .BBL file for submission.                     %%
%%  After submission of the .TEX file,                     %%
%%  you will be prompted to submit your .BBL file.         %%
%%                                                         %%
%%                                                         %%
%%  Note that the displayed Bibliography will not          %%
%%  necessarily be rendered by Latex exactly as specified  %%
%%  in the online Instructions for Authors.                %%
%%                                                         %%
%%%%%%%%%%%%%%%%%%%%%%%%%%%%%%%%%%%%%%%%%%%%%%%%%%%%%%%%%%%%%

% if your bibliography is in bibtex format, use those commands:
\bibliographystyle{bmc-mathphys} % Style BST file
\bibliography{refworks}      % Bibliography file (usually '*.bib' )

% or include bibliography directly:
% \begin{thebibliography}
% \bibitem{b1}
% \end{thebibliography}

%%%%%%%%%%%%%%%%%%%%%%%%%%%%%%%%%%%
%%                               %%
%% Figures                       %%
%%                               %%
%% NB: this is for captions and  %%
%% Titles. All graphics must be  %%
%% submitted separately and NOT  %%
%% included in the Tex document  %%
%%                               %%
%%%%%%%%%%%%%%%%%%%%%%%%%%%%%%%%%%%

%%
%% Do not use \listoffigures as most will included as separate files

\section*{Figures}
  \begin{figure}[h!]
  \caption{\csentence{iview rendering of the CCR5 chemokine receptor-HIV entry inhibitor maraviroc complex \cite{1348} (PDB code: 4MBS).}
      The secondary structure of human CCR5 is rendered as ribbon. The marketed HIV drug maraviroc is rendered as sphere. This figure can be reproduced at http://istar.cse.cuhk.edu.hk/iview/?4MBS
}\label{fig:ribbon}
      \end{figure}

  \begin{figure}[h!]
  \caption{\csentence{iview rendering of the CCR5 chemokine receptor-HIV entry inhibitor maraviroc complex \cite{1348} (PDB code: 4MBS), with anaglyph effect enabled.}
      In a virtual reality environment, when users wear a spectacle with special filters on both sides, the disparity between two superimposed molecules creates a perception of depth, leading to visually more appealing identification of intermolecular interactions. This figure can be reproduced at http://istar.cse.cuhk.edu.hk/iview/?4MBS
}\label{fig:anaglyph}
      \end{figure}

  \begin{figure}[h!]
  \caption{\csentence{iview rendering of the CCR5 chemokine receptor-HIV entry inhibitor maraviroc complex \cite{1348} (PDB code: 4MBS), with protein surface enabled.}
      The human CCR5 is rendered as molecular surface colored by chain. The marketed HIV drug maraviroc is rendered as stick colored by chain. It can be clearly seen that the crystal structure consists of two complexes, and the CCR5 forms a deep allosteric cavity where maraviroc is buried. This figure can be reproduced at http://istar.cse.cuhk.edu.hk/iview/?4MBS}\label{fig:surface}
      \end{figure}

  \begin{figure}[h!]
  \caption{\csentence{Tailor-made version of iview specifically for visualizing docking results of user-submitted jobs.}
      It displays the user-supplied search space in the form of a cubic box so that the binding site is visually depicted. It displays the top hit ligand IDs in a horizontally scrollable row and provides a straightforward way to switch ligands easily. It lists the docking result files, predicted binding affinity values, molecular properties, compound suppliers and annotations, and putative hydrogen bonds, in order to give users a quick overview of the top hit ligands, and assist them in making decisions of which compounds to purchase for subsequent wet-lab experiments. This figure can be reproduced at http://istar.cse.cuhk.edu.hk/idock/iview/?525a0abab0717fe31a000001
}\label{fig:idock}
      \end{figure}

%%%%%%%%%%%%%%%%%%%%%%%%%%%%%%%%%%%
%%                               %%
%% Tables                        %%
%%                               %%
%%%%%%%%%%%%%%%%%%%%%%%%%%%%%%%%%%%

%% Use of \listoftables is discouraged.
%%
\section*{Tables}
\begin{table}[h!]
\caption{Full features of iview. iview is the only web visualizer that is accelerated by GPU hardware and supports three unique features: protein surface construction, virtual reality effects, and PDBQT format input.}\label{tbl:features}
      \begin{tabular}{ll}
        \hline
                 Category & Features\\
        \hline
        File format input & PDB\\
                          & PDBQT\\
        \hline
                   Camera & perspective\\
                          & orthographic\\
        \hline
               Background & black\\
                          & grey\\
                          & white\\
        \hline
       Structure coloring & atom spectrum\\
                          & protein chain\\
                          & protein secondary structure\\
                          & B factor\\
                          & residue name\\
                          & residue polarity\\
                          & atom type\\
        \hline
        Primary structure & line\\
                          & stick\\
                          & ball \& stick\\
                          & sphere\\
                          & dot\\
        \hline
      Secondary structure & ribbon\\
                          & strand\\
                          & cylinder \& plate\\
                          & C alpha trace\\
                          & B factor tube\\
        \hline
          Protein surface & Van der Waals surface\\
                          & solvent excluded surface\\
                          & solvent accessible surface\\
                          & molecular surface\\
        \hline
 Proteins surface opacity & 1.0\\
                          & 0.9\\
                          & 0.8\\
                          & 0.7\\
                          & 0.6\\
                          & 0.5\\
        \hline
Protein surface wireframe & yes\\
                          & no\\
        \hline
   Virtual reality effect & anaglyph\\
                          & parallax barrier\\
                          & oculus rift\\
        \hline
      Canvas manipulation & mouse\\
                          & hand touch\\
        \hline
            Canvas keying & rotation\\
                          & translation\\
                          & zooming\\
                          & slab\\
        \hline
            Canvas export & png\\
        \hline
      \end{tabular}
\end{table}

%%%%%%%%%%%%%%%%%%%%%%%%%%%%%%%%%%%
%%                               %%
%% Additional Files              %%
%%                               %%
%%%%%%%%%%%%%%%%%%%%%%%%%%%%%%%%%%%

%\section*{Additional Files}
%  \subsection*{Additional file 1 --- Sample additional file title}
%    Additional file descriptions text (including details of how to
%    view the file, if it is in a non-standard format or the file extension).  This might
%    refer to a multi-page table or a figure.


\end{backmatter}
\end{document}
