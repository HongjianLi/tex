\documentclass[10pt,conference,compsocconf]{../IEEEtran}
\usepackage{xltxtra}
%\usepackage{subfig}
%\usepackage{booktabs}
\usepackage{flushend}
\usepackage[numbers,sort&compress]{natbib}
\setmainfont{Times New Roman}

\begin{document}

\title{Integrating Protein-Protein and Protein-Chemical Interactions with Neo4j and Node.js} % can use linebreaks \\ within to get better formatting as desired
\author
{
\IEEEauthorblockN
{
Hongjian Li
\IEEEauthorblockA
{
Department of Computer Science and Engineering\\
Chinese University of Hong Kong\\
hiji@cse.cuhk.edu.hk
}
}
}
\maketitle

\begin{abstract}

Neo4j is a high-performance NoSQL graph database particularly suitable for storing and managing nodes and edges. Node.js is an event-driven server side javascript platform particularly suitable for building real-time web applications. In this project, I intend to utilize neo4j and node.js to integrate both protein-protein and protein-chemical interaction data, and establish a modern website for fast query and vivid visualization.

\end{abstract}

%\begin{IEEEkeywords}

%Web, HTML5, CSS3, NoSQL, Graph Database, neo4j, node.js

%\end{IEEEkeywords}

\section{Introduction}

NoSQL is referred to as a new class of database management systems (DBMS) that greatly differ from the traditional relational database management systems (RDBMS). NoSQL is special in the sense that it no longer embraces SQL as its primary query language. NoSQL neither requires fixed table schemas, nor supports join operations. NoSQL is well known for its high performance and horizontal scalability in certain data-intensive applications such as large-scale indexing, real time web, and multimedia streaming. In the recent decade, NoSQL has evolved from an experimental toy into a stable and highly productive DBMS for a wide spectrum of modern applications.

Depending on the way how the data is stored and organized, NoSQL falls into categories such as key-value stores, column stores, document stores, and graph databases. Among the many NoSQL graph databases, Neo4j \citep{1076} is the most highly recognized and widely deployed one, and is therefore used as a backend store engine in this graph-related project.

Neo4j is, as described by its official website, "a high-performance, NOSQL graph database with all the features of a mature and robust database." It is implemented in Java yet offers remarkable performance improvement on up to three orders of magnitude for lots of graph-related applications. It is widely adopted probably because of its high performance, open source design philosophy, complete and helpful documentation, support for various programming language bindings, and large user base. Its community edition is released under GPLv3 license, which allows free usage.

Meanwhile, node.js, a platform built on top of Google Chrome's V8 JavaScript runtime, has been gaining more and more popularity these years. Node.js is special that its I/O model is event-driven and non-blocking, making it particularly suitable for easily building data-intensive real-time applications in a lightweight and efficient manner. 

To the best of my knowledge, there are two neo4j drivers for node.js. One is developed by the official neo4j team, and the other is developed by Aseem Kishore et al. from the Thingdom company. Both drivers basically act as a wrapper for neo4j's REST API.

\section{Motivation}

STRING (Search Tool for the Retrieval of Interacting Genes/Proteins) \citep{1070,1071,1072,1073,1074,1075} is a database of known and predicted protein-protein interactions, including direct (physical) and indirect (functional) associations derived from four sources, namely genomic context, high-throughput experiments, conserved coexpression and previous knowledge. STRING quantitatively integrates interaction data from these sources for a large number of organisms, and transfers information between these organisms where applicable.

STTICH (Search Tool for Interactions of Chemicals) \citep{1068,1069} is a resource to explore known and predicted protein-chemical interactions. Chemicals are linked to other chemicals and proteins by evidence derived from experiments, databases and the literature.

STRING and STITCH are separately hosted and maintained. Updates to one are not automatically reflected in the other. In addition, they both use PostgreSQL, a traditional relational database management system (RDBMS), to store primary data and precomputed predictions. This could result in a large amount of join operations, the bottleneck of query performance. Furthermore, the graph visualization in their websites relies on Adobe Flash, a commercial product believed to be replaced by the open standard HTML5.

I am motivated by the desire to overcome the limitations mentioned above and thus propose to reorganize from scratch their protein-protein and protein-chemical interaction data for fast and efficient queries using state-of-the-art technologies such as neo4j NoSQL graph database, event-driven node.js, neo4j driver for node.js from Thingdom, and modern website built on top of Twitter's bootstrap template in HTML5.

\section{Datasets}

The STRING and STITCH databases are explored. The latest version of STRING is 9, covering 5,214,234 proteins from 1133 organisms with a data size of 20GB. The latest version of STITCH database is 3, containing interactions for between 300,000 small molecules and 2.6 million proteins from 1133 organisms with a data size of 21GB.

\section{Progress So Far}

I have installed neo4j v1.6.1, node.js v0.6.14, neo4j driver for node.js v0.2.4, and Twitter's bootstrap v2.0.2, and properly configured the REST web server of neo4j.

I have done a survey on graph visualization packages based on HTML5 and javascript.

\section{Plan for the Rest}

\subsection{Initializing and Populating Neo4j databases}



\subsection{Exploiting Neo4j's Traversal and Built-in Graph Algorithms}

Neo4j has a very powerful traversal yet to be further exploited. It also implements several well-known graph algorithms, e.g. shortest paths, Dijkstra, A*, to name a few.

\subsection{Exploiting Neo4j's Fulltext Search Engine Lucene}

Neo4j inherits most of its indexing capabilities from the famous Lucene project funded by the Apache Software Foundation. An Neo4j index in a graph database maps a key-value pair to a node or to a relationship between two nodes. The mapping can be either exact or approximate, with the latter being known as fulltext search.

\subsection{Building a Website Frontend}

The last task is to encapsulate the above functionalities into a modern website. Twitter's bootstrap will serve as the template for HTML5 and CSS3, and jQuery will serve as the JavaScript client for getting and postting HTTP reqeusts, and receiving and processing HTTP responses.

\section{Availability}

Upon project completion, all relevent code files and documentations will be made available under Apache License 2.0.

%\section{Conclusion}



\bibliographystyle{unsrtnat}
\bibliography{../refworks}

\end{document}
