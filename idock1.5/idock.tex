\documentclass[10pt,conference,compsocconf]{../IEEEtran}
\usepackage{xltxtra}
\usepackage{subfig}
\usepackage{booktabs}
\usepackage{flushend}
\usepackage[numbers,sort&compress]{natbib}
\setmainfont{Times New Roman}

\begin{document}

\title{idock 1.5: Further Improving Docking Speed over idock 1.0 and AutoDock Vina}
\author
{
\IEEEauthorblockN
{
Hongjian Li, Kwong-Sak Leung and Man-Hon Wong
\IEEEauthorblockA
{
Department of Computer Science and Engineering, Chinese University of Hong Kong, Hong Kong, P.R. China\\
\{hjli, ksleung, mhwong\}@cse.cuhk.edu.hk
}
}
}
\maketitle

\begin{abstract}

In a previous study, we reported idock 1.0, a multithreaded virtual screening tool for flexible ligand docking. In this study, we report idock 1.5, further improving docking speed and accuracy, inventing new functionalities, and fixing bugs. We carried out a comprehensive benchmark to better evaluate idock 1.5 and AutoDock Vina. Result showed that idock 1.5 outperformed AutoDock Vina by 17x in terms of docking speed. idock is free and open source, available at https://GitHub.com/HongjianLi/idock. For future research, we are constructing a web site to provide free real-time virtual screening online, and we will port idock to the GPU using CUDA and OpenCL for further speed up.

\end{abstract}

\section{Introduction}

Protein-ligand docking predicts the preferred conformation and binding affinity of a small ligand when it is non-covalently bound to a specific binding site of a macro protein. Up to date, there are hundreds of docking programs available. The AutoDock series is the most cited docking software in the research community. AutoDock has contributed to the discovery of several drugs, including the first clinically approved HIV integrase inhibitor \citep{1169}. Technically speaking, AutoDock is single threaded \textit{per se}. Following its initial release, several parallel implementations were developed \citep{115,560,782}, using either multithreading or computer cluster.

In 2009, AutoDock Vina \citep{595} was released. As the successor of AutoDock 4, AutoDock Vina significantly improves the average accuracy of the binding mode predictions while running two orders of magnitude faster with multithreading \citep{595}. It was compared to AutoDock 4 on selecting active compounds against HIV protease, and was recommended for docking large molecules \citep{556}. Its functionality of semi-flexible protein docking by enabling flexibility of side-chain residues was evaluated on VEGFR-2 \citep{1084}. To further facilitate the usage of AutoDock Vina, auxiliary tools were subsequently developed, including a PyMOL plugin for program settings and visualization \citep{609}, and a bootable operating system for computer clusters \citep{773}. AutoDock Vina is free and open source under Apache License 2.0. So far it has been cited by over 357 publications according to Google Scholar, making it a very competitive docking program.

In 2011, inspired by AutoDock Vina, we developed idock 1.0, a multithreaded virtual screening tool for flexible ligand docking \citep{1153}. idock inherits from AutoDock Vina the accurate scoring function and the efficient optimization algorithm, and meanwhile introduces a fruitful of innovations, such as receptor and grid map caching for large-scale virtual screening, revised numerical model for much faster approximation, capability of automatic detection and deactivation of inactive torsions, utilization of our novel thread pool to parallelize grid map creation and reuse threads, utilization of the new C++11 feature of rvalue references to avoid frequent memory reallocation, and accelerated parsers for both receptor and ligand. When benchmarked on docking 10,928 drug-like ligands against HIV reverse transcriptase, idock 1.0 achieved a speedup of 3.3 in terms of CPU time and a speedup of 7.5 in terms of elapsed time on average compared to AutoDock Vina.

Despite the amazing speedup, idock 1.0 still required about 10 hours on average to dock 10,928 drug-like ligands, not to mention massive docking of millions of ligands. Faster algorithms and implementations are highly desired. Following the release of idock 1.0 in July 2011, we then released idock 1.1 in December 2011, idock 1.2 in February 2012, idock 1.3 in March 2012, idock 1.4 in April 2012, and idock 1.5 in June 2012, further improving docking speed and accuracy, inventing new functionalities, and fixing bugs.

\section{Our Contributions to idock 1.5}

For functional improvements, idock evaluates intra-ligand free energy in addition to inter-ligand free energy, leading to lower RMSD (Root Mean Square Deviation) values between docked conformation and native conformation and thus higher prediction accuracy. idock enables automatic recovery, i.e. in case the process gets killed accidentally and restarted some time later, it not only resumes docking from the previous stopping point, skipping ligands that were already docked in a previous run, but also detects and reports possible file content errors, ensuring all the output ligands are well written. It supports as many as 29 chemical elements including rare ones like As (arsenic) and Sr (strontium), covering the majority of ligand atom types. Given a very large amount of ligands to dock, idock indirectly supports 2-phase virtual screening via two consecutive runs. In the first run, idock performs coarse but fast virtual screening without writing any conformations to file, aiming to quickly shortlist a few candidate compounds. In the second run, idock performs fine but slow virtual screening with a significantly larger number of Monte Carlo tasks per ligand, writing as many conformations to file as possible and aiming to refine the predicted free energy as well as predicted conformation of candidate compounds. Such a 2-phase docking methodology can remarkably reduce overall execution time while avoiding the risk of filtering out potentially promising compounds, controlling the false negative rate at an acceptable level.

For I/O improvements, idock supports reading and writing compressed ligand files with in gzip/bzip2 format, resulting in a file footprint as low as just one eighth of the raw size. This new functionality turns out to be extremely handy given an enormous amount of ligands to dock. idock implements our own lightweight thread-safe progress bar, reporting progress every 10\% Monte Carlo tasks per ligand. idock evaluates and outputs verbose information to docked PDBQT files, including total free energy, total inter-ligand free energy, total intra-ligand free energy, number of hydrogen bonds, and per-atom inter-ligand free energy, facilitating interaction hotspot determination and indirectly helping \textit{in silico} synthesis of potent ligands. idock writes docking summary, sorted in the ascending order of predicted free energy of docked ligands, in CSV (Comma-Separated Vector) format for subsequent analysis easily.

For computational improvements, idock better supports rvalue references and move semantics in C++11 to boost performance. idock implements our novel thread pool in order to reuse threads and maintain a high CPU utilization throughout the entire screening procedure. The thread pool parallelizes the precalculation of scoring function, the creation of grid maps, and the execution of Monte Carlo tasks. idock flattens the tree-like recursive data structure of ligand as used in AutoDock Vina into simple linear array structure to ensure a high data cache hit rate and easy coding. idock accelerates the assignment of atom types by making use of residue information for receptor and branch information for ligand. In addition to Linux and Windows, idock supports Mac OS X, FreeBSD and Solaris, making it instantly ready on the 5 mainstream operating systems.

For other miscellaneous but helpful improvements, idock allows users to specify a configuration file other than appending arguments one by one onto the command line. idock increases the number of docking examples to 12. Last the not the least, idock fixes several minor bugs. Refer to the change log for detail.

\section{Benchmarks, Results and Discussions}

idock x86\_64 v1.5 and AutoDock Vina x86 v1.1.2 were evaluated on desktop computers with Intel Core i5-2400 CPU @ 3.10GHz and 4GB DDR3 RAM under Mac OS X 10.7.4 Build 11E53. By default, both programs output 9 predicted conformations per ligand. The benchmarks include comparison of their redocking performance in terms of predicted conformations, and comparison of their virtual screening performance in terms of execution time, memory usage, predicted free energy, and predicted conformations.

\subsection{Benchmark of Redocking Performance}

Redocking refers to randomizing the native ligand conformation in a protein-ligand complex and trying to dock the randomized conformation back to its native conformation as close as possible. For the redocking benchmark, we used two databases, PDBbind v2011 \citep{529,530} and CSAR NRC HiQ Set 24Sept2010 \citep{857,960}. The refined set of PDBbind v2011 and the two sets of CSAR NRC HiQ Set 24Sept2010 comprise 2,455 and 343 protein-ligand complexes respectively, with experimentally determined binding affinity data (Kd or Ki).

Figure \ref{fig:SuccessRate} shows the success rate of idock and AutoDock Vina under various conditions regarding the RMSD values between the crystal and docked conformations. Given a redocking case, RMSD1 refers to the RMSD value between the crystal conformation and the first docked conformation, i.e. the one with the highest predicted binding affinity, while RMSDm refers to the RMSD value between the crystal conformation and the closest docked conformation, i.e. the one with the minimum RMSD value. The condition RMSDm=RMSD1 tests for how many percent the docked conformation with the highest predicted binding affinity actually turns out to be the closest one among the 9 predicted conformations. It can be seen that 74\% RMSD values are below 2.0 \AA, a publicly accepted positive control for correct bound structure prediction.

\begin{figure}
\centering
\includegraphics[width=\linewidth]{SuccessRate.pdf}
\caption{Success rate of idock and AutoDock Vina under various conditions on PDBbind v2011 and CSAR NRC HiQ Set 24Sept2010.}
\label{fig:SuccessRate}
\end{figure}

Figure \ref{fig:FECorrelation} shows the free energy correlation of idock and AutoDock Vina. On the CSAR NRC HiQ Set 24Sept2010 database, the Pearson correlations between experimental binding affinity and free energy predicted by AutoDock Vina, between experimental binding affinity and free energy predicted by idock, and between free energy predicted by AutoDock Vina and idock are -0.5998758, -0.5774972, and 0.9824049, respectively. The experimental binding affinity is positive while the predicted free energy is negative, hence the negative sign for the former two correlations. It can be seen that both programs fail to predict reliable free energy, a very common obstacle in the entire research community. As expected, the correlation between free energy predicted by both programs is very high because of their identical scoring function.

\begin{figure*}
\centering
\subfloat[PDBbind v2011]
{
  \label{subfig:PDBbindFECorrelation}
  \includegraphics[width=0.485\linewidth]{PDBbindFECorrelation.pdf}
}
\subfloat[CSAR NRC HiQ Set 24Sept2010]
{
  \label{subfig:CSARFECorrelation}
  \includegraphics[width=0.485\linewidth]{CSARFECorrelation.pdf}
}
\caption{Free energy correlation on PDBbind v2011 and CSAR NRC HiQ Set 24Sept2010.}
\label{fig:FECorrelation}
\end{figure*}

\subsection{Benchmark of Virtual Screening Performance}

We collected 12 receptors from the PDB (Protein Data Bank) \cite{540,537}. Their PDB codes are 1HCL, 1J1B, 1LI4, 1V9U, 2IQH, 2XSK, 2ZD1, 2ZNL, 3BGS, 3H0W, 3IAR, and 3KFN.

1, 10, 100, 1000 ligands
[200, 300), [300, 400), [400, 500)
144 test cases in total.

ligands were collected from slice 16\_p0.0 of the clean drug like subset of the ZINC database \cite{532}.

\begin{figure*}
\centering
\subfloat
{
  \includegraphics[width=0.485\linewidth]{MWT.pdf}
}
\subfloat
{
  \includegraphics[width=0.485\linewidth]{NRB.pdf}
}
\caption{Boxplot of molecular weight and number of rotatable bonds of sets 200-300, 300-400, and 400-500.}
\label{MWT-NRB}
\end{figure*}

10,928 drug-like ligands were docked against the 12 proteins by Vina and idock. Since Vina can dock only one ligand in each run, a script containing 10,928 lines was generated and run instead, with each line being an execution of Vina to dock one individual ligand. Arguments to both programs were left as default. The GNU Time utility was used as profiler.

Table \ref{tab:ExecutionTimeAndMemoryUsage} compares the execution time and memory usage of both programs. Vina required 428 to 504 CPU hours for one protein case, while idock required merely 88 to 184 CPU hours, resulting in a speedup of 2.5 to 4.8 and a screening performance of 1.3 drug-like ligands per CPU minute on average. In terms of elapsed time, the speedup was increased to as high as 6.3 to 10.4 because idock better utilized the CPU cores thanks to its efficient thread pool. idock also better utilized available memory to build grid maps at a high resolution and retained them along the way. Even though idock consumed more memory than Vina, its maximum resident set size did not exceed 1.5 GB, hence idock can run be on mainstream desktop computers.

\begin{table}
\centering
\begin{tabular*}
{\linewidth}
{@{\extracolsep{\fill}}rrrrr}
\toprule
Program & CPU Hours & Elapsed & CPU Util. & Max Mem Usage\\
\midrule
\multicolumn{5}{l}{\textbf{HIV RT}}\\
Vina  & 464 & 69:15:13 &  670\% & 126 MB\\
idock & 162 & 10:57:46 & 1474\% & 856 MB\\
Ratio & 2.9 &      6.3 & 0.45   & 0.15\\
\noalign{\smallskip\smallskip}
\multicolumn{5}{l}{\textbf{SAHH}}\\
Vina  & 460 & 78:53:59 &  582\% &   150 MB\\
idock & 184 & 12:24:24 & 1484\% & 1,368 MB\\
Ratio & 2.5 &      6.4 &  0.39  & 0.11\\
\noalign{\smallskip\smallskip}
\multicolumn{5}{l}{\textbf{ADA}}\\
Vina  & 504 & 74:22:37 &  677\% & 114 MB\\
idock & 127 &  8:46:12 & 1452\% & 764 MB\\
Ratio & 4.0 &      8.5 &  0.47  & 0.15\\
\noalign{\smallskip\smallskip}
\multicolumn{5}{l}{\textbf{PNP}}\\
Vina  & 428 & 62:19:55 &  687\% & 116 MB\\
idock &  88 &  5:58:19 & 1479\% & 857 MB\\
Ratio & 4.8 &     10.4 &  0.46  & 0.13\\
\noalign{\smallskip\smallskip}
\multicolumn{5}{l}{\textbf{Average}}\\
Vina  & 464 & 71:12:56 &  654\% & 124 MB\\
idock & 140 &  9:31:40 & 1472\% & 961 MB\\
Ratio & 3.3 &      7.5 & 0.44   & 0.13\\
\bottomrule
\end{tabular*}
\caption{Execution time and memory usage of docking 10,928 drug-like ligands against HIV RT, SAHH, ADA, and PNP by Vina and idock. Maximum CPU utilization is 400\% due to Intel's Hyper-Threading technology.}
\label{tab:ExecutionTimeAndMemoryUsage}
\end{table}

Table \ref{tab:RMSEAndRMSD} summarizes the root mean square errors (RMSEs) of free energies and RMSDs of conformations predicted by both programs. The RMSEs of free energies predicted by both programs vary from 0.31 to 0.46 kcal/mol, apparently less than 2.85 kcal/mol, the standard error obtained by Vina, indicating both programs predicted very similar free energies. For 27\% to 40\% of all the 10,928 ligands, the RMSD of the conformations predicted by both programs is equal to or less than 1.0 \AA, and for 49\% to 61\%, the RMSD is equal to or less than 2.0 \AA, indicating both programs predicted similar conformations for around half of the cases.

\begin{table}
\centering
\begin{tabular*}
{\linewidth}
{@{\extracolsep{\fill}}cccc}
\toprule
Protein & RMSE (kcal/mol) & Avg RMSD (\AA) & RMSD $\leq$ 2.0 \AA\\
\midrule
HIV RT & 0.35 & 2.554 & 61\%\\
SAHH   & 0.46 & 4.190 & 49\%\\
ADA    & 0.33 & 2.620 & 59\%\\
PNP    & 0.31 & 2.966 & 53\%\\
\bottomrule
\end{tabular*}
\caption{RMSEs of free energies and RMSDs of conformations predicted by Vina and idock.}
\label{tab:RMSEAndRMSD}
\end{table}

\section{Availability}

idock is free and open source under Apache License 2.0. Precompiled executables for 32-bit and 64-bit Linux, Windows, Mac OS X, FreeBSD and Solaris, 12 docking examples, and a doxygen file for generating API documentations are available at https://github.com/HongjianLi/idock.

\section{Future Directions}

We are actively developing istar, a SaaS (Software as a Service) platform for idock. The goal of istar is to automate virtual screening. Without tedious software installation, users, especially computational chemists, can submit docking jobs on the fly in either of three ways: browsing our web site, programming against our RESTful API, or sending emails conforming to our specifications. Upon job completion, users can download top hits and seek for purchasing information.

At the same time, we will port idock to the GPU using CUDA and OpenCL, aiming at further speeding up idock by at least an order of magnitude. We will also incorporate click chemistry into idock for \textit{in silico} synthesis of novel ligands.

\bibliographystyle{unsrtnat}
\bibliography{../refworks}

\end{document}
