\chapter{usr: Ultrafast Shape Recognition}

\section{Background}

RDKit
\citep{1127} Freely Available Conformer Generation Methods: How Good Are They?
\citep{1393,1394} Cyndi: a multi-objective evolution algorithm based method for bioactive molecular conformational generation
\citep{1407} Shaping the interaction landscape of bioactive molecules
\citep{1407} the 3D shape of molecules plays an important role when binding to a target. Therefore, the ligands of a protein often display similarity in their shape. This similarity can be quantified by methods based on structural alignment (Gong et al., 2013; Liu et al., 2011; Sastry et al., 2011) or shape recognition (Ballester and Richards, 2007; Ballester et al., 2009; Wirth and Sauer, 2011).
\citep{1407} Using a reference set of 224 412 molecules active on 1700 human proteins, we show that accurate target prediction can be achieved by combining different measures of chemical similarity based on both chemical structure and molecular shape.
\citep{1408} SwissTargetPrediction: a web server for target prediction of bioactive small molecules. identify new targets for uncharacterized molecules or secondary targets for known molecules. Here, we introduce SwissTargetPrediction, a web server to accurately predict the targets of bioactive molecules based on a combination of 2D and 3D similarity measures with known ligands.
\citep{1402} Ligand-based in silico target fishing can be used to identify the potential interacting target of bioactive ligands, which is useful for understanding the polypharmacology and safety profile of existing drugs. (Given approved drugs, find their new protein targets)

Compared to other existing approaches \citep{1333,1334,1335,1337,1338,1331}, istar::USRCAT has several distinctive features. First, it uses USRCAT. Second, it database has 23 million ZINC compounds. Third, it supports multiple queries in one job.

\chapterend
