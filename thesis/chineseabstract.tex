藥物研發是一項昂貴和長期的商業。開發一種新藥需要18億美元和13年半的時間。因此,我們意圖針對藥物研發開發一個計算框架,當中利用圖形處理器實現加速,從而模擬現代藥物研發的初始階段,以節省金錢和時間。

從計算的角度而言,這種模擬一般涉及基於結構的虛擬篩選以及強效藥物的計算生成。到目前為止,我們為此開發了两個工具。

我們已經使用了這两個工具,連同其他一些現有工具,為艾滋病和老人癡呆症研發潛在的新藥。有部分新藥已經被計算篩選和生成,以後將進行臨床研究。

這三個工具的開發只是一個開始。最終我們要開發一個全面的統一的計算框架,當中包括粘合位置的識別、分子對接、虛擬篩選、藥物生成、藥性預測、以及交互式的可視化。最終也是最重要的一點,我們應當利用此框架去研發新藥,以拯救數以百萬計人類的生命。

