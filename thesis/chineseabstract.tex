開發一種新藥需要多至26億美元和13年半的時間。为節省金錢和時間,我们开发了一套计算机辅助药物研发工具集,并运用该工具集寻找药物治疗癌症和流感。

我們首先实现了一个快速的蛋白-配体对接工具idock,相比一个同类流行软件获得了显著的速度提升。为辅助idock的大规模使用,我们设计了一个异构网站平台istar,收集了多达两千三百万个小分子的大型数据库。为从三维展示分子间相互作用,我们开发了一个交互式网页可视化软件iview。为生成全新的化合物,我们开发了一个基于分子片段的药物设计工具iSyn。为改进分子结合强度预测的精度,我们利用了机器学习技术随机森林去重新打分晶体及预测构象。为消除对接对可用蛋白结构的要求,我们移植了超快形状识别算法至istar。值得强调的是,这些工具全是免费和开源。

重要地,我們应用了此创新工具集至现实世界药物寻找中。我们发现抗痤疮药阿达帕林可用于治疗人类结肠癌,亦筛选出流感病毒蛋白的潜在抑制物。这些新发现可望拯救人類生命。
