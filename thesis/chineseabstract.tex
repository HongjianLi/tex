開發一種新藥需要多至26億美元和13年半的時間。為節省金錢和時間,我們開發了一套計算機輔助藥物研發工具集,並運用該工具集尋找藥物治療癌症和流感。

我們首先實現了一個快速的蛋白與配體對接工具idock,相比一個同類流行軟件獲得了顯著的速度提升。為輔助idock的大規模使用,我們設計了一個異構網站平台istar,收集了多達兩千三百萬個小分子的大型數據庫。為在網頁展示分子間相互作用,我們開發了一個交互式可視化軟件iview。為生成全新的化合物,我們開發了一個基於分子片段的藥物設計工具iSyn。為改進結合強度預測的精度,我們利用了機器學習技術隨機森林去重新打分晶體及預測構象。為尋找結構相似的化合物,我們移植了超快形狀識別算法至istar。所有這些工俱全是免費和開源。

我們應用了此創新工具集至現實世界藥物尋找中。我們發現抗痤瘡藥阿達帕林可用於治療人類結腸癌,亦篩選出流感病毒蛋白的潛在抑制物。這些新發現可望拯救人類生命。
