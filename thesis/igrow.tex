\chapter{igrow: De Novo Drug Design}

\section{Background}

Given a pharmacological protein of therapeutic interest, protein-ligand docking tries to discover promising ligands out of existing compound databases. Apparently the diversity of its outcome is limited by the diversity of the database. In other words, docking will fail if the selected database contains no promising ligands at all. Hence constructing \textit{de novo} ligands from fragments now becomes a hot research problem.

Figure \ref{igrow:LigandDesign} \citep{363} illustrates two strategies for ligand design, link/grow strategy and lattice strategy. The recent years have seen a prosperity of \textit{de novo} ligand design programs, such as MORPH \citep{365}, GARLig \citep{471}, LEA3D \citep{1223}, LigBuilder 2 \citep{749}, AutoT\&T \citep{780}, AutoGrow \citep{466}, AutoClickChem \citep{1051}, CrystalDock \citep{954}, and LigMerge \citep{1181}. Refer to review papers for a more complete list \citep{363,367,472,1006} and methodology development \citep{470,982}. Meanwhile, several databases have been established for fragment-based drug design, such as e-Drug3D \citep{1125}. Furthermore, a number of compounds that evolved from fragments have entered the clinic, and the approach is increasingly accepted as an additional route to identifying new ligands in inhibitor design \citep{363,367,472,474,1006}.

A computational \textit{de novo} ligand design program helps to explore a much larger chemical space for novel drugs, but the search space is just too huge, virtually infinite. The number of chemically feasible, drug-like molecules has been estimated to be in the order of 10\textsuperscript{60} to 10\textsuperscript{100} \citep{1104}, from which the most promising candidates have to be selected and synthesized. Hence, rather than systematic construction and evaluation of each individual compound, computational \textit{de novo} ligand design programs rely on the principle of local optimization, which does not necessarily lead to the globally optimal solution. In fact, most software implementations are non-deterministic, and rely on some kind of stochastic structure optimization.

Amongst the many \textit{de novo} ligand design programs, AutoGrow \citep{466} is a representative one which implements genetic algorithm to create a population of ligands. It uses Vina \citep{595} as external docking engine for the selection operator. AutoClickChem \citep{1051} is capable of performing \textit{in silico} click chemistry reactions, ensuring that chemical synthesis is fast, cheap, and comparatively easy for subsequent testing in biochemical assays. LigMerge \citep{1181} is an automated, ligand-based algorithm for systematically swapping the chemical moieties of known ligands to generate novel ligands with potentially improved potency. It has been shown to identify compounds predicted to inhibit peroxisome proliferator-activated receptor gamma, HIV reverse transcriptase, and dihydrofolate reductase with affinities higher than those of known ligands. FragVLib \citep{1247} is a free tool for performing similarity search across ligand-receptor complexes for identifying binding pockets similar to that of a target receptor of interest. The above programs are free and most are open source. They are used as baseline tools in our research projects.

\chapterend
