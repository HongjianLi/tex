\chapter{Introduction}

\section{Background}

Drug discovery is an expensive and long-term business. Summarized from 13 research articles published from 1980 to 2009, original estimates of the cost of drug development ranged more than 9-fold, from USD$92 million cash (USD$161 million capitalized) to USD$883.6 million cash (USD$1.8 billion capitalized) \citep{1431}. Discovering and developing a new molecular entity (NME) required 11.4 to 13.5 years using the R\&D performance productivity data from 13 large pharmaceutical companies across 2000 to 2007 \citep{716}. An recent report \citep{1427} reviewed the rates of NMEs introduction starting from 1827 through to the end of 2013, and found that two-thirds of NMEs are controlled by a handful of companies, and a growing number of NMEs are controlled by marketing organizations that have little or no internal drug discovery or development activities.

Modern drug discovery process typically includes target identification, hit identification, lead optimization and clinical trials. A biological target is any system that can potentially be modulated by a molecule to produce a beneficial effect. A target could be a fundamental pathological pathway, altering which is expected to be curative or anti-symptomatic. Hits are compounds that have activity at a predetermined level against a target, but little else is known at this early stage. Leads are optimized hits that display strong potency and selectivity, physicochemical characteristics, and absorption, distribution, metabolism, excretion and toxicity (ADMET) properties. Successful candidate leads will then be submitted to the appropriate health authorities to get permission to conduct clinical investigations on animals and humans.

\section{Motivation}

Drug discovery is economy driven \textit{per se}. Biochemical means are both cost- and time-inefficient. This highlights the need for cheaper and faster methods, and computer-aided drug discovery (CADD) thus comes into the scene. Complementing expensive laboratory experiments with cheap computer simulations is obviously the right way to go. Robust computational frameworks are indeed highly demanded by the industry in order to automate the early phases of modern drug discovery such as target and hit identifications. Although a large amount of CADD tools have been developed over recent decades, the majority of them, unfortunately, suffer from several notable problems. These tools 1) are commercial, selling at a price that most small enterprises and academic institutions cannot afford, 2) are proprietary and closed source, making third parties difficult to study the internal implementations or locate potential bugs, 3) conform to different standards and formats, resulting in weak data portability and information loss, 4) require intensive and tedious configurations and lack a friendly user interface, a great obstacle for new users to get started, 5) run rather slowly, incapable of utilizing the multi- and many-core architectures of modern computers, or even worse, 6) are declared dead immediately upon their initial release due to zero maintenance afterwards. Therefore, we will address these shortcomings in this thesis.

\section{Objective}

We aim to develop a complete but concise CADD toolchain, and ultimately apply it to the discovery of novel drugs. Keeping several key goals in mind, we design our toolchain to 1) be freely available to the general public, 2) be released under permissive open source licenses, 3) conform to official standards, 4) provide a responsive web version, 5) run reasonably fast, and 6) track bugs and incorporate user feedback. Most importantly, we shall utilize our toolchain to discover potent drugs against certain diseases of therapeutic interest and hopefully save human lives.

\section{Contributions}

TODO: summarize contributions from following chapters: idock, istar, iview, igrow, rfscore.

\section{Thesis Outline}

The thesis is organized into 10 chapters as follows:

Chapter 2 presents our tool idock for fast virtual screening. Compared with AutoDock Vina \citep{595}, idock obtains a speed up of 6.3x to 10.4x, resulting in a screening performance of 1.3 drug-like ligands per CPU minute. We have used idock in virtual screening tens of thousands of ligands against HIV reverse transcriptase with minimal side effects against four other human proteins for the treatment of AIDS.

Chapter 3 presents istar, our web platform for online drug discovery.

Chapter 4 presents iview, our WebGL visualizer.

Chapter 5 presents idock 3 with GPU acceleration.

Chapter 6 presents our new tool igrow for computational synthesis of potent ligands. Compared with AutoGrow \citep{466}, ligands generated by igrow retain 100 Da lower molecular weights, making them more likely to be refined into drugs. In terms of predicted binding affinity, igrow outperforms AutoGrow by around 10\%. In terms of execution time, igrow runs 30\% faster than AutoGrow on average. We have used igrow to computationally synthesize a few potent ligands for the treatments of AD and AIDS.

Chapter 7 presents our real life case study of influenza A H1N1.

Chapter 8 presents our real life case study of CCRK-related cancers.

Chapter 9 summarizes the thesis and outlines future directions.

The appendix lists my journal and conference publications in chronological order.

\chapterend
