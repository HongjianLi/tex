\chapter{RF-Score-v3: binding affinity prediction}

\section{Abstract}

This was a collaborative project with Pedro J. Ballester from European Bioinformatics Institute, Cambridge, United Kingdom. It was published in the \textit{Proceedings of the 11th International Meeting on Computational Intelligence Methods for Bioinformatics and Biostatistics (CIBB)} on 26 June 2014 \citep{1433}.

%in \textit{BMC Bioinformatics} on 27 August 2014 \citep{1432}

\section{Introduction}

\citep{1399} rescoring docked poses is also seen in protein-protein docking using machine learning techniques. T-PioDock, a framework for detection of a native-like docked complex 3D structure. T-PioDock aims at supporting the identification of near-native conformations from 3D models produced by docking software by scoring those models.
\citep{1389} A rotation-translation invariant molecular descriptor of partial charges and its use in ligand-based virtual screening
\citep{1389} For an extensive reference on molecular descriptors, cf. [10]. The DRAGON software [14] can compute thousands of such descriptors.
\citep{1400} QuBiLS-MIDAS: a parallel free-software for molecular descriptors computation based on multilinear algebraic maps
http://www.talete.mi.it/products/dragon\_description.htm Dragon 6 calculates 4885 molecular descriptors.

\citep{1423} comparison of confirmed inactive and randomly selected compounds as negative training examples in support vector machine-based virtual screening
\citep{1404} The influence of negative training set size on machine learning-based virtual screening

\citep{1377} Random generalized linear model: a highly accurate and interpretable ensemble predictor
\citep{1418} Predicting COPD status with a random generalized linear model

\citep{1426} Comparative Assessment of Scoring Functions on an Updated Benchmark: 1. Compilation of the Test Set
\citep{1411} Comparative Assessment of Scoring Functions on an Updated Benchmark: 2. Evaluation Methods and General Results

\citep{1432} Substituting random forest for multiple linear regression improves binding affinity prediction of scoring functions: Cyscore as a case study

\section{Future works}

Enrichment
Protein-family specific scoring functions
Generate features from subsets of atoms, like UFSRAT and USRCAT.

\chapterend
