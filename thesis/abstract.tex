Drug discovery is an expensive and long-term business. It takes US\$1.8 billion over 13.5 years to develop a new drug. Complementing laboratory experiments with computer simulations is obviously the right way to go. We therefore aim to develop a novel and concise computational framework, simulating the early phases of modern drug discovery process in order to save money and time.

We have developed idock 1.0, a fast protein-ligand docking method for finding inhibitors (i.e. ligands) of viral proteins of therapeutic interest. It predicts both binding conformations of ligands against given proteins and their binding affinities. idock features built-in support for massive docking, a novel thread pool for parallel execution, automatic detection of inactive torsions, a fast numerical approximation model, support for gzip/bzip2 compression, automatic docking recovery, detection of putative hydrogen bonds, per-atom binding affinity output, full reproducibility with code and data, as well as many other great features. idock achieves a speedup of 3.3 in terms of CPU time and a speedup of 7.5 in terms of elapsed time on average. Compared with AutoDock Vina, idock obtains a speedup of 6.3x to 10.4x, achieving a screening performance of 1.3 drug-like ligands per CPU minute.

We have developed istar, a SaaS (Software as a Service) platform to automate large-scale docking using our popular docking engine idock. Without tedious software installation, users, especially computational chemists, can submit jobs on the fly either by using our web site or by programming against our RESTful API. We have ported idock and igrep onto istar. Compared with other online docking platforms such as DOCK Blaster and iScreen, istar features ligand filtering with desired molecular properties and previewing the number of ligands to dock, monitoring job progress in real time, supplier output, innovative two-phase docking, two-level parallelism, GPU acceleration support, full reproducibility with code and graphical tutorials, as well as many other great features. Our istar website supports 1) filtering ligands by desired molecular properties and previewing the number of ligands to dock, 2) monitoring job progress in real time, and 3) visualizing ligand conformations and outputting free energy and ligand efficiency predicted by idock, binding affinity predicted by RF-Score, putative hydrogen bonds, and supplier information for easy purchase, three useful features commonly lacked on other online docking platforms like DOCK Blaster or iScreen. We have collected 17,224,424 ligands from the All Clean subset of the ZINC database, and revamped our docking engine idock to version 2.0, further improving docking speed and accuracy, and integrating RF-Score as an alternative rescoring function. Compared with state-of-the-art AutoDock Vina, idock features an amazing speedup of at least 8.69 times and at most 37.51 times, remarkably reducing the docking time from years to months given millions of ligands to dock. In the past we developed igrep, a CUDA implementation of the agrep algorithm. istar is available at http://istar.cse.cuhk.edu.hk.

We have developed iview, an easy-to-use interactive WebGL visualizer of protein-ligand complex. It exploits hardware acceleration rather than software rendering. It features three special effects in virtual reality settings, namely anaglyph, parallax barrier and oculus rift, resulting in visually appealing identification of intermolecular interactions. It supports four surface representations including Van der Waals surface, solvent excluded surface, solvent accessible surface and molecular surface. Moreover, based on the feature-rich version of iview, we have also developed a neat and tailor-made version specifically for our istar web platform for protein-ligand docking purpose. This demonstrates the excellent portability of iview.

We have developed idock 3.0, a CUDA and OpenCL implementation to utilize GPU (Graphics Processing Unit) acceleration and harness the tremendous computational power and memory bandwidth offered by modern GPU nowadays. We have surveyed the latest GPU architectures codenamed GK104 and Tahiti offered by NVIDIA and AMD respectively. We have also observed thread behaviors and identified program hotspots of idock 2.0. idock 3.0 will revolve around three strategies, which are maximizing parallel execution, maximizing memory bandwidth, and maximizing instruction throughput, aiming to further reduce the docking time from months to weeks and even to days, making large-scale docking a really pragmatic practice for potential use by computational chemists. By then we shall be able to perform proteomic-scale docking for the entire solved proteins and warehouse the results into a public database.

We have developed igrow, an de novo ligand design method. It synthesizes ligands from scratch by incorporating fragments. It utilizes idock as the backend docking engine, supporting more reaction rules, and tracking synthetic feasibility. Compared with AutoGrow, ligands generated by igrow retain 100 Da lower molecular weights and 10\% lower free energies on average. In addition, igrow runs 30\% faster. We will exploit click chemistry for realistic synthetic feasibility.

We have done a real life case study on discovering inhibitors of viral proteins of influenza A virus H1N1. We have identified the nucleoprotein and the RNA polymerase subunits PA and PB2 as druggable protein targets, and spent six months in running idock to test over 10 million ligands. We have shortlisted a few promising ligands to carry out subsequent biological assays, and we are now seeking for appropriate vendors.

We have done another real life case study on discovering inhibitors of CCRK (Cell Cycle-Related Kinase), which has been proved to involve in glioblastoma multiforme carcinogenesis, ovarian carcinomas and hepatocarcinogenesis. We have executed idock to test about 5 thousand approved drugs, and shortlisted 11 ligands for purchasing.
