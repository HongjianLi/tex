Developing a new drug costs up to US$2.6B and 13.5 years. To save money and time, we have developed a toolset for computer-aided drug discovery, and utilized our toolset to discover drugs for the treatment of cancers and influenza.

We first implemented a fast protein-ligand docking tool called idock, and obtained a substantial speedup over a popular counterpart. To facilitate the large-scale use of idock, we designed a heterogeneous web platform called istar, and collected a huge database of more than 23 million small molecules. To elucidate molecular interactions in 3D, we developed an interactive WebGL visualizer called iview. To synthesize novel compounds, we developed a fragment-based drug design tool called iSyn. To improve the predictive accuracy of molecular binding affinity, we exploited the machine learning technique random forest to re-score both crystal and docked poses. To eliminate the requirement of availability of protein structure in docking, we ported the ultrafast shape recognition algorithms to istar. It is noteworthy to highlight that all these tools are free and open source.

Importantly, we applied our novel toolset to real world drug discovery. We repurposed the anti-acne drug adapalene for the treatment of human colon cancer, and identified candidate inhibitors of influenza viral proteins. Such new findings could hopefully save human lives.
