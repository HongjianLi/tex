\chapter{istar: Software as a Service}

\section{Background}

\citep{1396} The web interface enables binding sites detection, virtual screening hits identification, and drug targets prediction in an interactive manner through a seamless interface to all adapted packages (e.g., Cavity, PocketV.2, PharmMapper, SHAFTS). Several commercially available compound databases for hit identification and a well-annotated pharmacophore database for drug targets prediction were integrated in iDrug as well. The web interface provides tools for real-time molecular building/editing, converting, displaying, and analyzing.
\citep{1396} iDrug requires Java for visualization.
\citep{1425} SwissDock: a protein-small molecule docking web service based on EADock DSS

SaaS (Software as a Service) is part of the nomenclature of cloud computing. It is a software delivery model in which software and associated data are centrally hosted on the cloud. SaaS is typically accessed by users using a thin client via a web browser. In computational biology, ten simple rules have been summarized for providing a scientific web resource \citep{677}. Software and web sites do count for getting ahead as a computational biologist \citep{260}.

Up to date, several SaaS platforms for protein-ligand docking or \textit{de novo} ligand design have been created. DOCK Blaster \citep{557} investigates the feasibility of full automation of protein-ligand docking. It utilizes DOCK \citep{1222} as the docking engine and ZINC \citep{532,1178} as the ligand database. iScreen \citep{899} is a compacted web server for TCM (Traditional Chinese Medicine) docking and followed by customized \textit{de novo} drug design. It utilizes PLANTS \citep{610,607,779} as the docking engine and TCM@Taiwan \citep{528} as the ligand database. It also utilizes LEA3D \citep{1223} for \textit{de novo} ligand design. FORECASTER \citep{1012} is a web interface consisting of a set of tools for the virtual screening of small molecules binding to biomacromolecules (proteins, receptors, and nucleic acids). It utilizes the flexible-target docking program FITTED \citep{602} as docking engine. VSDMIP \citep{848} is an automated structure- and ligand-based virtual screening platform with a PyMOL graphical user interface. It utilizes CDOCK \citep{1224} as the docking engine.

\chapterend
