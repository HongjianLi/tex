\chapter{Case study of CDK2-related cancers}
\label{cdk2}

\section{Abstract}

Human colorectal cancer has been reported to express high level of cyclin-dependent kinase 2 (CDK2), a key factor regulating the cell cycle G1 to S transition and a hallmark for cancers. In this study, we used idock prospectively for the first time to identify potential CDK2 inhibitors from 4,311 FDA-approved small molecule drugs with a repurposing strategy. Among the top compounds sorted by idock score, nine were purchased. Among them, adapalene (CD271,6-[3-(1-adamantyl)-4-methoxyphenyl]-2-naphtoic acid) exhibited the highest antiproliferative effect in human colon cancer LOVO and DLD1 cells. We demonstrated for the first time that adapalene treatment significantly increased the percentage of cells in G1 phase, and decreased the expressions of CDK2, cyclin E and Rb, as well as the phosphorylations of CDK2 on Thr160 and Rb on Ser795. We then examined the anti-cancer effect of adapalene \textit{in vivo} in BALB/C nude mice subcutaneously xenografted with human colorectal cancer DLD1 cells. Our results showed that oral adapalene treatment significantly (p<0.05) and dose-dependently inhibited tumor growth. Adapalene (20 mg/kg) exhibited strong anti-tumor activity, comparable to that of oxaliplatin (40 mg/kg). The combination with adapalene and oxaliplatin exhibited the highest therapeutic effect. These results suggested for the first time that adapalene is a potential CDK2 inhibitor and a candidate anti-cancer drug for the treatment of human colorectal cancer.

This is an ongoing collaborative project with Prof. Marie Chia-Mi Lin and Xi-Nan Shi from Biotechnology Center, Kumming Medical University, China.

\section{Background}

Cyclin-dependent kinase (CDK2) is one of the serine/threonine protein kinases. It plays a pivotal role in regulating the cell cycle transition from G1 to S phase, and thus in controlling cell proliferation. Abnormally high expression of CDK2 has been reported in many human neoplasias, such as colorectal, ovarian, breast and prostate cancers. Hence, CDK2 inhibitors are potential effective anti-cancer agents.

\section{Motivation}

Although a number of CDK2 inhibitors have been described in the literature, such as flavopiridol \citep{1596}, roscovitine \citep{1597} and olomoucine \citep{1598}, none of them is available for clinical use due to toxicity and multi-target specificity.

\section{Objective}

We utilized our free and open-source protein-ligand docking software idock \citep{1153,1362} to screen FDA-approved small molecule drugs against CDK2. We adopted the approach of structure-based virtual screening to repurpose approved toxicity-free drugs for the treatment of cancers that involve CDK2 regulation.

\section{Methods and Materials}

\subsection{Docking}

44 X-ray crystallographic structures of CDK2 in complex with a ligand (Table \ref{cdk2:PDBs}) were collected from the PDB (Protein Data Bank) \citep{540,537}. The co-crystallized ligands and waters were manually removed. The structures of FDA-approved drugs were collected from the dbap and fda catalogs of the ZINC database \citep{532,1178}. The dbap catalog of version 2014-03-19 comprising 1,738 compounds and the fda catalog of version 2012-07-25 comprising 3,176 compounds were downloaded, of which 4,311 were unique. The 44 CDK2 structures in PDB format and the 4,914 compounds in Mol2 format were then converted to PDBQT format using AutoDockTools \citep{596}. Our free and open-source docking software idock v2.1.2 \citep{1153,1362} was then executed to predict the binding conformations and the binding affinities of the 4,914 compounds when docked against the 44 CDK2 structures using an ensemble docking strategy. Finally the compounds were sorted in the ascending order of their predicted binding free energy averaged across the 44 CDK2 structures, and the top commercially available compounds were queried and purchased via http://www.chemicalbook.com/ and subsequently validated \textit{in vitro} and \textit{in vivo}.

\begin{table}
\caption{The 44 CDK2 holo structures used for ensemble docking.}
\label{cdk2:PDBs}
\begin{tabular}{ccc}
\hline
PDB ID & Resolution (\AA) & UniProt ID\\
\hline
1AQ1 & 2.00 & P24941\\
1CKP & 2.05 & P24941\\
1DI8 & 2.20 & P24941\\
1DM2 & 2.10 & P24941\\
1E1V & 1.95 & P24941\\
1E1X & 1.85 & P24941\\
1FVT & 2.20 & P24941\\
1G5S & 2.61 & P24941\\
1GIH & 2.80 & P24941\\
1GII & 2.00 & P24941\\
1GIJ & 2.20 & P24941\\
1GZ8 & 1.30 & P24941\\
1H00 & 1.60 & P24941\\
1H01 & 1.79 & P24941\\
1H07 & 1.85 & P24941\\
1H08 & 1.80 & P24941\\
1H0V & 1.90 & P24941\\
1H0W & 2.10 & P24941\\
1JSV & 1.96 & P24941\\
1JVP & 1.53 & P24941\\
1KE5 & 2.20 & P24941\\
1KE6 & 2.00 & P24941\\
1KE7 & 2.00 & P24941\\
1KE8 & 2.00 & P24941\\
1KE9 & 2.00 & P24941\\
1OIQ & 2.31 & P24941\\
1OIR & 1.91 & P24941\\
1OIT & 1.60 & P24941\\
1P2A & 2.50 & P24941\\
1PF8 & 2.51 & P24941\\
1PXI & 1.95 & P24941\\
1PXJ & 2.30 & P24941\\
1PXK & 2.80 & P24941\\
1PXL & 2.50 & P24941\\
1PXM & 2.53 & P24941\\
1PXN & 2.50 & P24941\\
1PXO & 1.96 & P24941\\
1PXP & 2.30 & P24941\\
1PYE & 2.00 & P24941\\
1R78 & 2.00 & P24941\\
1URW & 1.60 & P24941\\
1V1K & 2.31 & P24941\\
1VYZ & 2.21 & P24941\\
1W0X & 2.20 & P24941\\
\hline
\end{tabular}
\end{table}

\subsection{Chemicals and antibodies}

The selected chemicals and the leading cancer drug oxaliplatin were purchased from Sigma-Aldrich, USA. Anti-cyclin D, B1, E, CDK2, Rb, Pho-CDK2 (Thr160), Pho-Rb (Ser795) and GAPDH were obtained from Cell Signaling Technology, Inc., Danvers, Massachusetts, USA.

\subsection{Cell lines and cell culture}

Colorectal cancer cell lines LOVO and DLD1 were obtained from the American Type Culture Collection, Manassas, Virginia, USA. These cell lines were cultured in RPMI 1640 medium containing 10\% fetal bovine serum (FBS) (Invitrogen, Rockville, Maryland, USA) at 37°C in 5\% CO\textsubscript{2} and 95\% humidified air.

\subsection{Cell culture experimental conditions}

Cells were plated in 96-, 24-, or 6-well plates with 0.125\% FBS medium for 24 hours and then treated with 10\% FBS medium containing the testing compounds at various concentrations of 1, 3, 10, 30$\mu$M, and incubated for 24, 48, or 72 hours.

\subsection{MTT assay}

Cells were plated at an initial density of 9x10\textsuperscript{3} cells/well in 96 well plates and incubated with 0.5mg/ml 3-(4,5-methylthiazol-2-yl)-2,5-diphenyl-tetrazolium bromide for 4 hours. The medium was then discarded and 200$\mu$l of formazan in dimethylsulphoxide (DMSO) were added. The absorbance was measured at 570 nm according the the standard protocol.

\subsection{Cell cycle analysis}

LOVO or DLD1 cells (4x10\textsuperscript{4}) were seeded in 24-well plates in RPMI 1640 medium containing 0.125\% FBS, and cultured for 24 hours. The cells were incubated in medium containing 10\% FBS and various doses of adapalene (1, 3, 10, 30 $\mu$M) for 12, 24, 36 hours at 37°C, then fixed in ice-cold 70\% ethanol and stained with a Coulter DNA-Prep Reagents kit (Beckman Coulter, Fullerton, California, USA). Cellular DNA content of 1x10\textsuperscript{4} cells from each sample was determined with the use of an EPICS ALTRA flow cyto-meter (Beckman Coulter). Cell cycle phase distribution was analyzed with the ModFit LT 2.0 software (Verity Software House, Topsham, Maine, USA). All results were obtained from two separate experiments, each of which was done in triplicate.

\subsection{Western blotting}

Cells were lysed with RIPA buffer containing 1 mM PMSF and protease inhibitor cocktail at 4°C for 30 minutes. After centrifugation at 13,000 rpm for 15 minutes, the supernatants were recovered and the protein concentration was measured by BCA Protein Assay Kit (Thermo). Equal amounts of cell lysates were resolved in 10\% SDS-PAGE and transferred onto nitrocellulose membranes (Sigma). After blocking, the membranes were incubated sequentially with the appropriate diluted primary and secondary antibodies. Proteins were detected by the enhanced chemiluminescence detection system (Amersham, Piscataway, New Jersey, USA). To ensure equal loading of the samples, the membranes were re-probed with an anti-GAPDH antibody (Cell Signalling Technologies).

\subsection{Adapalene treatment \textit{in vivo} in nude mice xenografted with colorectal cancer DLD1 cells}

Female BALB/C nude mice, 4 to 5 weeks old from Vital River Laboratory Technology Co. Ltd, Peking, China, were kept under specific pathogen-free conditions and cared for in accordance with the guidelines of the laboratory animal ethics committee of Kunming Medical University. For the xenografted tumor growth assay, 1x10\textsuperscript{6}/0.2ml PBS DLD1 cells were injected subcutaneously into the right flank of the mice (n=3). Tumor size was measured every day. One week after inoculation when the tumors grew to a volume of 80 to 100 m\textsuperscript{3}, the mice were divided randomly into groups with 5 mice per group, and gavaged daily for 21 days with 0.5\% CMC-NaCl containing various doses (15, 20, 65 and 100mg/kg) of adapalene and oxaliplatin (40mg/kg). The mice were then sacrificed by cervical dislocation. Tumor volume was calculated by the formula $V=ab^2/2$, where $a$ is the longest axis and $b$ is shortest axis.

\subsection{Statistical analysis}

The results were obtained from at least three different experiments and expressed as mean ± SD. Statistical analysis was performed by Student’s t test and differences were considered to be statistically significant if p < 0.05.

\section{Results}

\subsection{Selection of candidate inhibitors of CDK2}

Totally 4,914 FDA-approved drugs collected from the dbap and fda catalogs of the ZINC database were docked and ranked according to their average predicted binding affinity. Based on commercial availability, nine top-scoring compounds were selected for further studies (Table \ref{cdk2:Top9}). They are Nilotinib, LS-194959, adapalene, Estradiol benzoate, NandrolonePhenylpropionate, vilazodone, Azelastine Hydrochloride, Latuda, and Paliperidone.

\begin{table}
\caption{The nine candidate CDK2 inhibitors selected from FDA-approved drugs using structure-based virtual screening by idock.}
\label{cdk2:Top9}
\begin{tabular}{ccccc}
\hline
name & ZINC ID & idock score (kcal/mol) & clinic usage & Ref.\\
\hline
06716957 & -10.46 & Nilotinib & chronic myeloid leukemia & 18\\
03830332 & -10.43 & LS-194959 & Food, drug additive & 19\\
03784182 & -10.38 & Adapalene & acne & 20\\
03830768 & -10.23 & Estradiol benzoate & estrogen & 21\\
03881613 & -10.08 & Nandrolone Phenylpropionate & osteoporosis & 22\\
01542113 & -10.06 & vilazodone & major depressive disorder & 23\\
00897240 & -10.01 & Azelastine Hydrochloride & seasonal allergic rhinitis and perennial allergic rhinitis & 24\\
33974796 &  -9.98 & Latuda & schizophrenia & 25\\
01481956 &  -9.95 & Paliperidone & schizophreni & 26\\
\hline
\end{tabular}
\end{table}

\subsection{Adapalene decreased the cell viability of colorectal cancer LOVO and DLD1}

We first evaluated the anti-cancer effect of the nine compounds by MTT assay. These nine compounds all decreased cell viability in LOVO and DLD1 cells. The IC50 was calculated by graphpad prime5. Among them, adapalene had the lowest IC50 (4.43$\mu$M for DLD1, and 7.135$\mu$M for LOVO) (Figure \ref{cdk2:MTT}A). The growth inhibition effect of adapalene was dose- and time-dependent (P<0.05) (Figure \ref{cdk2:MTT}B). Marked inhibition was observed at 10 and 30$\mu$M, while no significant effect was observed at concentrations below 3$\mu$M (0,1,3 $\mu$M)

\begin{figure}
\centering
%\includegraphics[width=\linewidth]{../cdk2/MTT.png}
\caption{Comparison of the effect of nine candidate CDK2 inhibitors on the viability of LOVO and DLD1 colorectal cancer cells.}
\label{cdk2:MTT}
\end{figure}
(A) The cell viability assay. Nine compounds had discrepant cyto-toxicity to LOVO and DLD1 cell lines at different concentrationas determined by MTT assay, with adapalene exhibited the highest cyto-toxicity compared with control(※P<0.05). (B) Adapalene exhibited dose- and time-dependent inhibition on cell viability in LOVO and DLD1 cell lines compared with control(※P<0.05)(the effect of 1, 3$\mu$M was as well as control).

\subsection{Adapalene treatment caused cell cycle arrested at the G1 phase}

To understand whether adapalene inhibited CDK2 activities in colorectal cancer cells, we analysed the effect of adapalene treatment (3, 10, 30 $\mu$M) for 6, 12, 24 hours on cell cycle profile in LOVO or DLD1 cells by flow cytometry. As shown in Figure \ref{cdk2:CellCycleDistribution}A, adapalene treatment significantly increased the percentage of cells in G1 phase as compared to the control in a dose- and time-dependent manner (P<0.05). After 24 hours adapalene treatment, the changes of cell cycle profile of G0-G1, S, and G2-M phases were shown in Figure \ref{cdk2:CellCycleDistribution}B, The increase of the G1 phase was accompanied by the simultaneous significant decrease of S and G2-M phases.

\begin{figure}
\centering
%\includegraphics[width=\linewidth]{../cdk2/CellCycleDistribution.png}
\caption{Effects of adapalene on cell cycle distribution in LOVO and DLD1 colorectal cancer cell.}
\label{cdk2:CellCycleDistribution}
\end{figure}
LOVO and DLD1 cells were treated with different concentrations (3, 10 and 30$\mu$M) of adapalene for 6, 12, 24hours, and cell cycle distributions were determined by flow cytometry. (A) adapalene treatment dose- and time-dependently increased the \% of cells in G1 phase. At 30$\mu$M, adapalene caused maximum \% of G1 phase at 6-10 hours; and at 10$\mu$M increased the \% of G1 phase continuously for 24 hours, as compared to control (※P<0.05). (B) The cell cycle distributions. The bar graph indicated the percentage of the G1, G2 and S phases at 24 hours after adapalene treatment.

\subsection{Adapalene treatment decreased the expression of CDK2, Rb, cyclinE, pho-CDk2, pho-Rb but not cyclinD and cyclin B1 in LOVO and DLD1 cells}

We investigated the effect of adapalene on the expression of important proteins involved in G1-to-S transition, including CDK2, cyclinE, Rb, pho-CDk2 and pho-Rb by Western blotting in LOVO and DLD1 cells. As shown in Figure \ref{cdk2:WesternBlot}, adapalene treatment reduced the expression of CDK2, Rb, pho-CDK2, pho-Rb and cyclin E. In contrast, the expression levels of cyclin D1 and cyclin B1 remained unchanged. These results are consistent with what is expected with a CDK2 inhibitor.

\begin{figure}
\centering
%\includegraphics[width=\linewidth]{../cdk2/WesternBlot.png}
\caption{Effects of adapalene treatment on the expressions of cyclins, CDk2 and Rb.}
\label{cdk2:WesternBlot}
\end{figure}
LOVO and DLD1cells were plated at 6-well plates with 0.125\% FBS medium for 24 hours and then with 10\% FBS medium containing adapalene at concentration 3, 10, 30$\mu$M. Cells were harvested after 6 hours incubation and proterins analyzed by Western blotting. Western blotting results showed that adapalene treatment significantly reduced the expression of CDK2, Rb, pho-CDK2, pho-Rb and cyclinE in LOVO and DLD1 cells. In contrast, the expression levels of cyclin D1 and cyclin B1 remained unchanged in both cell lines.

\subsection{Daily oral adapalene treatment reduced tumor growth \textit{in vivo} in BALB/C nude mice subcutaneously xenografted with DLD1 cells}

To evaluate the effect of adapalene on the growth of colorectal carcinoma in vivo, BALB/C nude mice were subcutaneously injected with DLD1 cells. Carcinoma volumes were measured every 3 to 4 days after the appearance of the tumors. At 7 days after tumor inoculation, the volume of tumor reached 80-100mm3, various doses (15, 65 and 100mg/kg in 0.5\% CMC-Nacl) of adapalene were administered daily for 21 days by oral gavage. In a separate experiment, we also compared the efficacy of adapalene (20mg/kg), oxaliplatin (40mg/kg) and the combination of adapalene (20mg/kg) plus oxaliplatin (40mg/kg).

Our results demonstrated that oral adapalene treatment significantly (p<0.05) inhibited tumor growth. At day 21 after treatment, 15 mg/kg adapalene produced significant reduction (P<0.05) of tumor weight and volume as compared to control (P<0.05) (Figure \ref{cdk2:Figure4}). There is no significant difference between 15 and 65 mg/kg adapalene treatment.

\begin{figure}
\centering
%\includegraphics[width=\linewidth]{../cdk2/Figure4.png}
\caption{Oral adapalene treatment significantly reduced tumor growth \textit{in vivo} in nude mice xenografted with DLD1 cells.}
\label{cdk2:Figure4}
\end{figure}
(A) Daily oral adapalene treatment (from day 1 to day 21) dose-dependently (15, 65, 100 mg/kg) reduced the tumor volume. (B) Significantreduction of tumor weight were observed at as low as 15 mg/kg adapalene treatment compare with control at day 21 after adapalene treatment(※P<0.05). (C) The volume of tumor largely decreased at as low as 15 mg/kg adapalene treatment compare with control at day 21 after adapalene treatment(※P<0.05).

In addition, the anti-tumor activity of oral adapalene (20 mg/kg) was comparable to that of oxaliplatin (40 mg/kg). Importantly, the combination therapy exhibited the highest therapeutic effect (Figure \ref{cdk2:Figure5}). These results suggested for the first time that adapalene is a potential CDK2 inhibitor and a candidate anti-cancer drug for the treatment of human colorectal cancer.

\begin{figure}
\centering
%\includegraphics[width=\linewidth]{../cdk2/Figure5.png}
\caption{Oral adapalene combinated oxaliplatin treatment significantly reduced tumor growth \textit{in vivo} in nude mice xenografted with DLD1 cells.}
\label{cdk2:Figure5}
\end{figure}
(A) Daily oral adapalene 20mg/kg, oxaliplatin 40mg/kg and adapalene 20mg/kg combinated oxaliplatin 40mg/kg treatment (from day 1 to day 21) reduced the tumor volume. (B) Significantreduction of tumor weight were observed at adapalene 20mg/kg combinated oxaliplatin 20mg/kg compare with control at day 21 after combinated treatment(※P<0.05). (C) The volume of tumor largely decreased at adapalene 20mg /kg combinated oxaliplatin 40mg/kg treatment compared with adapalene 20mg/kg only or oxaliplatin 20mg/kg only at day 21 after combinated treatment(※P<0.05).

\subsection{Structural analysis of the predicted conformation of adapalene docked against CDK2}

As shown in Figure \ref{cdk2:Figure6}A, we plotted the predicted conformation of CDK2 in complex with adapalene in a 3D manner using iview \citep{1366}. Figure \ref{cdk2:Figure6}B plots the intermolecular interaction diagram in a 2D manner using PoseView \citep{748}. adapalene was predicted to reside in the ATP-binding site of CDK2 and interact with CDK2 mainly through hydrophobic contacts with Phe82, Ile10, Leu134, Lys33 and His84.

\begin{figure}
\centering
%\includegraphics[width=\linewidth]{../cdk2/Figure6.png}
\caption{The comformation of CDK2 and adapalene interactions.}
\label{cdk2:Figure6}
\end{figure}
Figure 6A showed the predicted conformation of CDK2 in complex with adapalene in a 3D manner using iview \citep{1366}. The intermolecular interaction diagram in a 2D manner of adapalene was predicted to reside in the ATP-binding site of CDK2 and interact with CDK2 mainly through hydrophobic contacts with Phe82, Ile10, Leu134, Lys33 and His84 by PoseView \citep{748}. (Figure 6B).

% Use openbabel to generate 2D png plots of hits. obabel -o png -xp 200

\section{Discussion}

We adopted the repurposing strategy. We concentrated on repurposing approved drugs \citep{1023,944} because developing a drug \textit{de novo} is a laborious and costly endeavor.

\citep{1384} reviewed drug repositioning by structure-based virtual screening (SBVS) and highlighted the powerful synergy of \textit{in silico} methods in drug repositioning. To name a few successful repurposing cases by SBVS, \citep{1507} rediscovered 2,4-Dichlorophenoxy acetic acid, a well-known plant auxin, as a new anti-inflammatory agent through an \textit{in silico} molecular modeling and docking studies along with drug formulation and \textit{in vivo} anti-inflammatory inspection; \citep{1506} attempted to repurpose FDA-approved drugs by an integrated SBVS approach and reported the discovery of piperacillin \textbf{1} as an inhibitor of NEDD8-activating enzyme (NAE) in cell-free and cell-based systems with high selectivity.

In addition to SBVS, ligand-based virtual screening (LBVS) also finds its successful cases in repurposing. \citep{1504} used Ultrafast Shape Recognition (USR) \citep{1379} to search for compounds with similar shape to a previously reported inhibitor of protein arginine deiminase type 4 (PAD4), a new therapeutic target for the treatment of rheumatoid arthritis.

Some freely accessible drug databases include DrugBank \citep{1594}, KEGG DRUG \citep{1595} and e-Drug 3D \citep{1125}.

%\citep{565} A Machine Learning-Based Method To Improve Docking Scoring Functions in eHiTS and Its Application to Drug Repurposing.%Feb 2011
%\citep{861} Identify drug repurposing candidates by mining the Protein Data Bank.%Apr 2011

In this study we adopted the computational methodology of structure-based virtual screening (SBVS) by protein-ligand docking to shortlist candidates from FDA approved small molecule drugs. SBVS has become a routine task in pharmaceutical institutions.

CDK2 is an important target for cancer therapy. Through the interaction of CDKs and cyclins, cell cycle progress is sequentially and strictly processed29. Different cyclin/CDK complexes are activated in difference stages of the cell cycle30,32.When the cell cycle goes through G1 to S phase, the complex of cyclinD1–Cdk4/6 and cyclinE–CDK2 are ordinally activated and the retinoblastoma protein (pRB) is hyper-phosphorylated on serine and threonine residues33,34. The hyperphosphorylated pRB promotes the release of E2F transcription factors, which in turn facilitates the transcription of numerous genes required for G1 to S transition and S phase progression35. A host of CDK2 inhibitors have been described in the literature . (Table \ref{cdk2:KnownInhibitors}) However, up to now, due to drug toxicity and unselectivity, they are still not available for clinical use.

\begin{table}
\caption{CDK2 inhibitor in literature.}
\label{cdk2:KnownInhibitors}
\begin{tabular}{ccc}
\hline
name & research institution & Ref.\\
\hline
AG-24322 & Agouron & 36\\
AT-7519 & Astex & 37\\
AT-9311 & Astex & 38\\
AZD-5438 & AstraZeneca & 39\\
AZD-5597 & AstraZeneca & 40,41\\
Compound 6b & Palacký University & 42\\
SCH-727965 & Schering-Plough & 43,44\\
flavopiridol & Sanofi-Aventis & 6\\
Roscovitine & Emory University and Imperial College & 7\\
Olomoucine & Laboratoire de PhysiologieVégétaleMoléculaire CNRS & 8\\
\hline
\end{tabular}
\end{table}

This study presents the first successful prospective application of idock \citep{1153,1362} in identifying CDK2 inhibitors using a repurposing strategy. idock is an exciting development not only because it has been vigorously shown10 to outperform the state-of-the-art docking software AutoDock Vina45 in terms of docking speed by at least 8.69 times and at most 37.51 times while maintaining comparable redocking success rates, but also because it is free and open source under a permissive license. The later guarantees that users from both industry and academia can freely utilize idock in their own projects that require protein-ligand docking.

To facilitate the use of idock, its input arguments and output results were purposely designed to be similar to those of AutoDock Vina, so the existing users can easily transit to idock and benefit from considerable speedup in SBVS performance. Furthermore, to promote prospective SBVS by idock, a web server named istar10was developed by our team. istar is freely available at http://istar.cse.cuhk.edu.hk, where there are as many as 23,129,083 purchasable small molecule compounds ready for docking against any protein provided by the user. We believe both idock\citep{1153} and istar \citep{1362} can supplement the efforts of medicinal chemists in drug discovery research.

Adapalene was selected for further investigations because its IC50 was less than 10 umol/L as determined by MTT assay. Adapalene is the third generation of synthetic retinoids, mainly used for the topical therapy of acne vulgaris \citep{1599}. Its antiproliferative and proapoptotic effects \textit{in vitro} were first reported in colon carcinoma cell lines (CC-531, HT-29 and LOVO) \citep{1600} and hepatoma cells (HepG2 and Hep1B) \citep{1601} by increasing the activity of caspase 3 via up-regulating bax and down-regulating bcl-2.

In this study, we reported for the first time that adapalene is a potential CDK2 inhibitor, and demonstrated for the first time that oral administration of adapalene (20 mg/kg) exhibited significant and strong anti-cancer efficacy as compared to the leading cancer drug oxaliplatin (40 mg/kg) \textit{in vivo} in nude mice xenografted with colorectal DLD1 cells. Importantly combination of effective dose of adapalene and oxaliplatin could produce even higher therapeutic effect, suggesting that adapalene with its different mechanism, could be combined with other chemotherapy drugs to achieve higher therapy effect.

Previously, no obvious toxicity was reported by either i.p. injection of 100 mg/kg adapalene in carrageenan induced paw oedema rat, or topic use of 10\% (10 mg/ml) in UV induced erythema guinea pig46. In our study, we did not observe significant change in body weight by oral administration of adapalene (15-100 mg/kg) for 21 days, suggesting that oral or ip injection of adapalene administration is relatively safe. TASHIRO T47 and his colleagues reported that ip injection of 5 and 10 mg/kg oxaliplatin on d2 in B6D2F mice subcutaneously xenografted with colon38 significantly reduced tumor weight to average 16 and 38\% of the control level, respectively, at 21 day post-treatment. In this study, we tested oral oxaliplatin treatment by gavage at dose of 10, 20, and 40mg/kg. We found that 40mg/kg effectively reduced the tumor weight to 28\% of the control on day 21 post-treatment, without showing any body weight change, suggesting that oral administration of oxaliplatin is relative safe and effective.

In recent years, a large number of CDK inhibitors have been reported in the literature. However, due to drug toxicity and selectivity, they are not available for clinic use up to now. As a FDA-approved drug, the application of adapalene and the combination of adapalene with other chemotherapy drugs for the treatment of colorect neoplasms and other cancers warrant further investigations.

\section{Conclusions}

Among the top compounds sorted by idock score, nine were selected and purchased for further studies. Among them, adapalene (CD271,6-[3-(1-adamantyl)-4-methoxyphenyl]-2-naphtoic acid) exhibited the highest anti-cancer effect in human colorectal LOVO and DLD1 cells. Consistent with the expected properties of CDK2 inhibitors, adapalene treatment significantly increased the percentage of cells in G1 phase, decreased the expressions of CDK2, cyclin E and Rb, as well as the phosphorylationsof CDK2 on Thr160 and Rb on Ser795. Finally, the anti-cancer effect of adapalene was examined \textit{in vivo} in a BALB/C nude mice model subcutaneously xenografted with human colorectal cancer DLD1 cells. Our results demonstrated that oral adapalene treatment significantly (p<0.05) and dose-dependently inhibited tumor growth. Adapalene (20 mg/kg) exhibited a strong anti-tumor activity comparable to that of oxaliplatin (40 mg/kg), with their combinatorial therapy exhibiting the highest therapeutic effect. These results suggested for the first time that adapalene is a potential CDK2 inhibitor and a candidate anti-cancer drug for the treatment of human colorectal cancer. The potential use of adapalene, a FDA approved drug, as an anti-cancer drug especially for the combination treatment in human colorectal cancer warrants further investigations.

\chapterend
