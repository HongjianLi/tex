\chapter{Case study of CDK2-related cancers}
\label{cdk2}

\section{Abstract}

This is an ongoing collaborative project with Prof. Marie Chia-Mi Lin and Xilan Shi from Kumming Medical University, China.

\section{Background}

Some freely accessible drug databases include DrugBank \citep{1594}, KEGG DRUG \citep{1595} and e-Drug 3D \citep{1125}.

\citep{1384} reviewed drug repositioning by structure-based virtual screening (SBVS) and highlighted the powerful synergy of \textit{in silico} methods in drug repositioning. To name a few successful repurposing cases by SBVS, \citep{1507} rediscovered 2,4-Dichlorophenoxy acetic acid, a well-known plant auxin, as a new anti-inflammatory agent through an \textit{in silico} molecular modeling and docking studies along with drug formulation and \textit{in vivo} anti-inflammatory inspection; \citep{1506} attempted to repurpose FDA-approved drugs by an integrated SBVS approach and reported the discovery of piperacillin \textbf{1} as an inhibitor of NEDD8-activating enzyme (NAE) in cell-free and cell-based systems with high selectivity.

In addition to SBVS, ligand-based virtual screening (LBVS) also finds its successful cases in repurposing. \citep{1504} used Ultrafast Shape Recognition (USR) \citep{1379} to search for compounds with similar shape to a previously reported inhibitor of protein arginine deiminase type 4 (PAD4), a new therapeutic target for the treatment of rheumatoid arthritis.

%\citep{565} A Machine Learning-Based Method To Improve Docking Scoring Functions in eHiTS and Its Application to Drug Repurposing.%Feb 2011
%\citep{861} Identify drug repurposing candidates by mining the Protein Data Bank.%Apr 2011

\section{Motivation}



\section{Objective}

Repurposing
We used the approach of structure-based virtual screening to repurpose existing approved drugs for the treatment of cancers that involve CDK2 regulation. Specifically, this virtual screening campaign was done by idock \citep{1153}.

\section{Methods}

44 CDK2 structures. Ensemble docking. UniProt ID: P24941

\begin{table}
\caption{44 CDK2 structures.}
\label{cdk2:PDBs}
\begin{tabular}{cc}
\hline
PDB ID & Resolution (\AA)\_x\\
\hline
1AQ1 & 2.00\\
1CKP & 2.05\\
1DI8 & 2.20\\
1DM2 & 2.10\\
1E1V & 1.95\\
1E1X & 1.85\\
1FVT & 2.20\\
1G5S & 2.61\\
1GIH & 2.80\\
1GII & 2.00\\
1GIJ & 2.20\\
1GZ8 & 1.30\\
1H00 & 1.60\\
1H01 & 1.79\\
1H07 & 1.85\\
1H08 & 1.80\\
1H0V & 1.90\\
1H0W & 2.10\\
1JSV & 1.96\\
1JVP & 1.53\\
1KE5 & 2.20\\
1KE6 & 2.00\\
1KE7 & 2.00\\
1KE8 & 2.00\\
1KE9 & 2.00\\
1OIQ & 2.31\\
1OIR & 1.91\\
1OIT & 1.60\\
1P2A & 2.50\\
1PF8 & 2.51\\
1PXI & 1.95\\
1PXJ & 2.30\\
1PXK & 2.80\\
1PXL & 2.50\\
1PXM & 2.53\\
1PXN & 2.50\\
1PXO & 1.96\\
1PXP & 2.30\\
1PYE & 2.00\\
1R78 & 2.00\\
1URW & 1.60\\
1V1K & 2.31\\
1VYZ & 2.21\\
1W0X & 2.20\\
\hline
\end{tabular}
\end{table}

4,914 ligands

Chemicals were purchased at http://www.chemicalbook.com.

We concentrated on repurposing approved drugs \citep{944,1023} because developing a drug \textit{de novo} is a laborious and costly endeavor. Using idock 1.5 with a fine grid map granularity of 0.08\AA\ and 512 Monte Carlo tasks, we screened 1,715 FDA-approved drugs via DrugBank and 3,176 FDA-approved drugs via DSSTOX.

\section{Results}

Figures \ref{cdk2:1gz8-ZINC03830332} and \ref{cdk2:1gz8-ZINC03831625} depict the interactions between the CCRK homologous model and two high-rank ligands.

% Use openbabel to generate 2D png plots of hits. obabel -o png -xp 200

\chapterend
