% This is "sig-alternate.tex" V2.0 May 2012
% This file should be compiled with V2.5 of "sig-alternate.cls" May 2012
%
% This example file demonstrates the use of the 'sig-alternate.cls'
% V2.5 LaTeX2e document class file. It is for those submitting
% articles to ACM Conference Proceedings WHO DO NOT WISH TO
% STRICTLY ADHERE TO THE SIGS (PUBS-BOARD-ENDORSED) STYLE.
% The 'sig-alternate.cls' file will produce a similar-looking,
% albeit, 'tighter' paper resulting in, invariably, fewer pages.
%
% ----------------------------------------------------------------------------------------------------------------
% This .tex file (and associated .cls V2.5) produces:
%       1) The Permission Statement
%       2) The Conference (location) Info information
%       3) The Copyright Line with ACM data
%       4) NO page numbers
%
% as against the acm_proc_article-sp.cls file which
% DOES NOT produce 1) thru' 3) above.
%
% Using 'sig-alternate.cls' you have control, however, from within
% the source .tex file, over both the CopyrightYear
% (defaulted to 200X) and the ACM Copyright Data
% (defaulted to X-XXXXX-XX-X/XX/XX).
% e.g.
% \CopyrightYear{2007} will cause 2007 to appear in the copyright line.
% \crdata{0-12345-67-8/90/12} will cause 0-12345-67-8/90/12 to appear in the copyright line.
%
% ---------------------------------------------------------------------------------------------------------------
% This .tex source is an example which *does* use
% the .bib file (from which the .bbl file % is produced).
% REMEMBER HOWEVER: After having produced the .bbl file,
% and prior to final submission, you *NEED* to 'insert'
% your .bbl file into your source .tex file so as to provide
% ONE 'self-contained' source file.
%
% ================= IF YOU HAVE QUESTIONS =======================
% Questions regarding the SIGS styles, SIGS policies and
% procedures, Conferences etc. should be sent to
% Adrienne Griscti (griscti@acm.org)
%
% Technical questions _only_ to
% Gerald Murray (murray@hq.acm.org)
% ===============================================================
%
% For tracking purposes - this is V2.0 - May 2012

\documentclass{sig-alternate}

\begin{document}
%
% --- Author Metadata here ---
\conferenceinfo{GECCO'14,} {July 12-16, 2014, Vancouver, BC, Canada.}
\CopyrightYear{2014}
\crdata{TBA}
\clubpenalty=10000
\widowpenalty = 10000
% --- End of Author Metadata ---

\title{igrow: A Multithreaded Ligand Synthesis Tool for Structure-Based Molecular Design (max 8 pages)}
%
% You need the command \numberofauthors to handle the 'placement
% and alignment' of the authors beneath the title.
%
% For aesthetic reasons, we recommend 'three authors at a time'
% i.e. three 'name/affiliation blocks' be placed beneath the title.
%
% NOTE: You are NOT restricted in how many 'rows' of
% "name/affiliations" may appear. We just ask that you restrict
% the number of 'columns' to three.
%
% Because of the available 'opening page real-estate'
% we ask you to refrain from putting more than six authors
% (two rows with three columns) beneath the article title.
% More than six makes the first-page appear very cluttered indeed.
%
% Use the \alignauthor commands to handle the names
% and affiliations for an 'aesthetic maximum' of six authors.
% Add names, affiliations, addresses for
% the seventh etc. author(s) as the argument for the
% \additionalauthors command.
% These 'additional authors' will be output/set for you
% without further effort on your part as the last section in
% the body of your article BEFORE References or any Appendices.

\numberofauthors{5} %  in this sample file, there are a *total*
% of EIGHT authors. SIX appear on the 'first-page' (for formatting
% reasons) and the remaining two appear in the \additionalauthors section.
%
\author{
% You can go ahead and credit any number of authors here,
% e.g. one 'row of three' or two rows (consisting of one row of three
% and a second row of one, two or three).
%
% The command \alignauthor (no curly braces needed) should
% precede each author name, affiliation/snail-mail address and
% e-mail address. Additionally, tag each line of
% affiliation/address with \affaddr, and tag the
% e-mail address with \email.
%
% 1st. author
\alignauthor Hongjian Li\\
\affaddr{Department of Computer Science and Engineering}\\
\affaddr{Chinese University of Hong Kong}\\
\affaddr{Shatin, New Territories, Hong Kong}\\
\email{hjli@cse.cuhk.edu.hk}
% 2nd. author
\alignauthor Kwong-Sak Leung\\
\affaddr{Department of Computer Science and Engineering}\\
\affaddr{Chinese University of Hong Kong}\\
\affaddr{Shatin, New Territories, Hong Kong}\\
\email{ksleung@cse.cuhk.edu.hk}
% 3rd. author
\alignauthor Chun Ho Chan\\
\affaddr{Department of Computer Science and Engineering}\\
\affaddr{Chinese University of Hong Kong}\\
\affaddr{Shatin, New Territories, Hong Kong}\\
\email{an_tony0522@yahoo.com.hk}
\and  % use '\and' if you need 'another row' of author names
% 4th. author
\alignauthor Hei Lun Cheung\\
\affaddr{Department of Computer Science and Engineering}\\
\affaddr{Chinese University of Hong Kong}\\
\affaddr{Shatin, New Territories, Hong Kong}\\
\email{jerlun@hotmail.com}
% 5th. author
\alignauthor Man-Hon Wong\\
\affaddr{Department of Computer Science and Engineering}\\
\affaddr{Chinese University of Hong Kong}\\
\affaddr{Shatin, New Territories, Hong Kong}\\
\email{mhwong@cse.cuhk.edu.hk}
}
\date{15 January 2014}

\maketitle
\begin{abstract}

our abstract

\end{abstract}

% A category with the (minimum) three required fields
\category{J.3}{LIFE AND MEDICAL SCIENCES}{Biology and genetics}
%http://www.acm.org/about/class/ccs98-html

\terms{Algorithms}

\keywords{Bioinformatics, Genetic Algorithms}

\section{Introduction}
Background

\section{Method}
Overview

\subsection{AutoClickChem}

\subsection{LigMerge}

\subsection{New Reactions}

\subsection{Experimental Settings}
Datasets, programs in use, comparison, real cases

\begin{table}
\centering
\caption{Frequency of Special Characters}
\begin{tabular}{|c|c|l|} \hline
Non-English or Math&Frequency&Comments\\ \hline
\O & 1 in 1,000& For Swedish names\\ \hline
$\pi$ & 1 in 5& Common in math\\ \hline
\$ & 4 in 5 & Used in business\\ \hline
$\Psi^2_1$ & 1 in 40,000& Unexplained usage\\
\hline\end{tabular}
\end{table}

\section{Results}
Our results

\begin{figure}
\centering
%\epsfig{file=fly.eps, height=1in, width=1in}
\caption{A sample.}
%\vskip -6pt
\end{figure}

\section{Conclusions}
Our conclusions

\section{Acknowledgments}
We would like to thank Ching-Man Tse for his initial work on ligand synthesis research.

%
% The following two commands are all you need in the
% initial runs of your .tex file to
% produce the bibliography for the citations in your paper.
\bibliographystyle{abbrv}
\bibliography{../refworks.bib}  % sigproc.bib is the name of the Bibliography in this case
% You must have a proper ".bib" file
%  and remember to run:
% latex bibtex latex latex
% to resolve all references
%
% ACM needs 'a single self-contained file'!
%
\end{document}
