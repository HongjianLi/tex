\documentclass[a4paper,12pt]{article}
\usepackage[margin=0.8in]{geometry}
\usepackage[compact]{titlesec}
\usepackage[numbers,sort&compress]{natbib}
\pagestyle{empty}

\begin{document}

Our research interest is in computer-aided drug discovery (CADD), which has now been widely recognized as a cost- and time-efficient approach alternative to biochemical means.

Starting from 2010, we have been developing a \textbf{multi-disciplinary next-generation} CADD platform, which includes pragmatic \textit{in silico} tools, large-scale databases, empirical research, and wet laboratory validations. Currently our toolset includes several modules: a high-throughput molecular docking program \cite{1153}, a web platform for automatic chemoinformatics \cite{1362}, an interactive WebGL visualizer \cite{1366}, accurate scoring functions for predicting molecular binding affinity \cite{1432,1647,1434,1663}, an intelligent drug design tool \cite{1409,1387}, and a molecular shape- and pharmacophore-matching web server. Thus far, our web server at \textbf{http://istar.cse.cuhk.edu.hk} has gained \textgreater48,000 page views by \textgreater7000 users from \textgreater90 countries worldwide. We have also collected a huge database of \textgreater23 million compounds in 3D format with molecular properties, and manually curated several catalogs of approved drugs, resulting in a unified collection of \textgreater3000 clinically approved drugs, not only in US, but also in Europe, Canada and Japan.

Most importantly, collaborating with a clinical biology team, we have prospectively utilized our toolset and successfully repurposed four approved drugs that exhibited submicromolar inhibitory effects to colorectal or hepatocellular carcinoma \textit{in vitro} and \textit{in vivo} \cite{1667,1681}. These newly identified medications of marketed drugs may represent an \textbf{immediately-applicable clinical therapy} for the treatment of colon and liver cancers. For one of the four drugs, we have filed a \textbf{patent application} in China (application ID: 201510501028.8).

For future work, we propose to unify and extend our tools and datasets to ultimately automate the modern drug discovery process, and use them to study and combat a vast variety of diseases. We plan to invent new tools that can 1) repurpose post-phase I drug candidates using phenotypic-based resources of hundreds of drugs and cancer cell lines, 2) adjust a compound's chemical components intelligently with knowledge from binding pockets to guarantee strong potency and selectivity, 3) predict the on- and off-targets of a compound based on its geometric and pharmacophoric similarity to annotated compounds, 4) perform ultrahigh-throughput virtual screening of large-scale molecular databases. Practically we will apply our CADD platform to the discovery of novel inhibitors of cancers, as well as influenza A, herpes, HBV viruses.

Dr. Hongjian Li has 7 journal and 10 conference publications. He was a postdoctoral fellow at CUHK from April to September 2015, and is now a visiting researcher at CRCM, the largest cancer research center in France. He was awarded ``Excellent Teaching Assistant" in 2009.

\linespread{0.5}
\tiny
\bibliographystyle{unsrtnat}
\bibliography{../refworks}

\end{document}

