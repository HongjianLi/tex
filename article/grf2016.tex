\documentclass[a4paper,12pt]{article}
\usepackage[margin=1in]{geometry}
\usepackage{times}
\usepackage[compact]{titlesec}
\usepackage[numbers,sort&compress]{natbib}
\linespread{1.0}

\begin{document}

\section*{Background of Research}

The stages of modern drug discovery typically include target identification, hit identification, lead optimization and clinical trials. A biological target is usually a protein that can potentially be modulated by a molecule. Hits are compounds that show activity at a predetermined level against a target. Leads are optimized hits that exhibit strong potency and physicochemical characteristics. Candidate leads are to be submitted to the health authorities to get permission to conduct clinical trials on animals and humans.

Computer-aided drug discovery (CADD) has now been recognized as a cost- and time-efficient approach alternative to purely biochemical means. Robust and reliable computational tools and methods are indeed highly demanded by the industry in order to automate the early phases of modern drug discovery, and they have been thriving over the past decade.

From a practical view, we observe that most CADD tools are either commercial, proprietary, not portable, or difficult to use and update. This imposes a great obstacle on users, and hinders the progression of new drug discovery. Often, these tools are evaluated merely on retrospective benchmarks. Their performance in prospective applications remains unknown in many cases. Some tools are declared dead upon their initial release due to zero maintenance afterwards.

Therefore in the past five years, we have been constructing a next-generation CADD toolset, which encapsulates several modules: 1) idock, a high-throughput molecular docking program, 2) istar, a web platform for large-scale chemoinformatics, 3) iview, an interactive WebGL visualizer, 4) RF-Score versions 3 and 4, accurate scoring functions for predicting intermolecular binding affinity, 5) iSyn, an automatic drug design tool, and 6) USR@istar, a 3D molecular shape- and pharmacophore-matching web server. It is important to highlight that our tools are all free and open source, and well maintained and recognized. Thus far, our web server at http://istar.cse.cuhk.edu.hk has served \textgreater48,000 page views by \textgreater7000 users from \textgreater90 countries worldwide.

Meanwhile, we have also conducted empirical research. We have found the limiting factor of predictive performance of classical scoring functions, and shown that substituting multiple linear regression by machine-learning algorithms as well as incorporating low-quality structural and interaction data can substantially increase the accuracy of binding affinity prediction.

In addition to tool development, we have also collected and curated several drug databases, which are essential data sources for inputting to our tools. On one hand, we collected a huge database of more than 23 million compounds in 3D format, together with annotations of their molecular properties. On the other hand, we manually curated several catalogs of approved drugs, resulting in a unified collection of clinically approved drugs, not only in US, but also in Europe, Canada and Japan, together with purchasing information available.

Most importantly, collaborating with a clinical biology team, we have prospectively utilized our toolset and successfully repurposed four approved drugs that exhibited submicromolar inhibitory effects to hepatocellular carcinoma, both \textit{in vitro} and \textit{in vivo}. These newly identified medications of marketed drugs, which have a history of safe human use, may present an immediately-applicable clinical therapy for the treatment of liver cancer, hopefully saving millions of lives. These inspiring results have thus vigorously demonstrated the validity and utility of our toolset. For one of the four drugs, we have filed a patent application in China (application ID: 201510501028.8).

In the following subsections, we outline the works done by others and their deficiencies, followed by the works done by us and their strengths, which are categorized by sub-areas of CADD.

\subsection*{Molecular docking and scoring}

Protein-ligand docking is a computational method that predicts how a small molecule, termed ligand, binds to a target protein, as well as how strongly they bind. Hence docking is useful in elaborating intermolecular interactions and enhancing the potency and selectivity of binding.

Very often, once a target protein of interest is identified from a pathway, a large database of ligands will be docked against the protein. This is to shortlist the ligands that are predicted to show the strongest binding affinity towards proteins intended to be inhibited, or the ligands that are predicted to show the weakest binding affinity towards proteins intended not to be inhibited. This process is known as structure-based virtual screening (SBVS).

\subsubsection*{Works done by others}

There are dozens of docking tools available, e.g. \cite{595,607,617,650,596}. Among them, AutoDock Vina \cite{595} is a competitive one not only because it is free and open source, but also because it has been shown to substantially improve the average accuracy of the binding mode predictions \cite{595} and run faster than its counterpart AutoDock 4 \cite{596} by an order of magnitude. Released in the second half of 2010, Vina has been cited over 2,500 times and adopted by a large community of users.

However, Vina has to parse the input protein structure and create energy grid maps every time it attempts to dock a single ligand. This limits its performance in high-throughput virtual screening. Besides, this software is no longer maintained by its developers.

\subsubsection*{Works done by us}

In 2011, we developed a new docking tool called idock \cite{1153}, which substantially revises the numerical approximation model and enhances the fundamental implementation of various components with modern C++11 tricks, such as multithreading. Compared with Vina, our idock obtained a speedup of 3.3x in terms of CPU time and a speedup of 7.5x in terms of elapsed time on average, making it particularly suitable for carrying out large-scale virtual screening.

\subsection*{Web server and databases}

Standalone docking tools are usually for a small portion of users only, as they require intensive settings. Medicinal chemists generally seek for web servers that are easy to use. One obvious advantage of web servers over standalone tools are the transparent handling of internal logics.

\subsubsection*{Works done by others}

A few online docking platforms exist. DOCK Blaster \cite{557} investigates the feasibility of full automation of protein-ligand docking. It utilizes DOCK \cite{1222} as the docking engine and ZINC \cite{532,1178} as the ligand database. iScreen \cite{899} is a compacted web server for TCM (Traditional Chinese Medicine) docking and followed by customized \textit{de novo} drug design. It utilizes PLANTS \cite{610,607,779} as the docking engine and TCM@Taiwan \cite{528} as the ligand database.

Nevertheless, these platforms neither support property-based ligand selection, nor are able to monitor job progress in real time. They also lack post-docking analysis, a hurdle that prevents users from studying and understanding the binding mode of candidate compounds.

\subsubsection*{Works done by us}

In 2012, we designed a modern web platform called istar \cite{1362}, which is freely available at http://istar.cse.cuhk.edu.hk. We believe istar is a remarkable improvement to existing web servers because it hides implementation details of data preparation, format conversion, docking, analysis and visualization, and thus provides the whole set of functionalities as an automatic pipeline, which considerably relaxes the requirements on users' knowledge and skills.

As a side product, we collected a huge database of more than 23 million small-molecule compounds, covering a large space of chemical diversity. This is thus far the largest open source web server ever available for performing structure-based virtual screening. To ensure smooth operations of istar, we utilized the departmental HK\$4M-worth supercomputing infrastructure equipped with 736 CPU cores and 79,872 GPU cores.

\subsection*{Molecular visualization}

Visualization plays an important role in observing and elaborating protein-ligand interactions and aiding novel drug design. It helps to learn new binding patterns and preferences.

\subsubsection*{Works done by others}

VMD \citep{1220}, PyMOL (http://www.pymol.org) and Chimera \citep{1219} are well-known and highly-cited interactive visualization tools. PoseView \citep{748} and LigPlot+ \citep{951}, on the other hand, detect protein-ligand interactions from 3D coordinates and plot static 2D diagrams.

In addition to the above standalone visualizers, there are web-based visualizers to facilitate deployment. Although Jmol (http://www.jmol.org) and JSmol \citep{1314} support advanced features, they rely on software rendering, which is slow on large display areas and thus prevents detailed inspection of the structure. In contrast, WebGL visualizers benefit from GPU hardware acceleration. GLmol (http://webglmol.sourceforge.jp) is a WebGL molecular viewer and supports multiple file formats and representations.

However, none of these web visualizers are tailor-made for virtual screening. Most of them either suffer from slow software rendering, or lack the support of macromolecular surface construction. The useful feature of virtual reality is also unavailable.

\subsubsection*{Works done by us}

In 2014, we developed an interactive WebGL visualizer called iview \cite{1366}, permitting easy accessibility and platform independence. It exploits hardware rendering, and supports macromolecular surface representations as well as special effects in virtual reality settings \cite{1265}. The unique feature that distinguishes iview from other visualizers is its inherent support for docking input preparation and result analysis with detection of binding interactions. We have seamlessly integrated iview into our istar platform, freely available at http://istar.cse.cuhk.edu.hk/iview.

\subsection*{Binding affinity prediction}

Docking consists of two main operations: predicting the conformation of a ligand when docked to the protein's binding site, and predicting their binding affinity. The single critical limitation of docking has been shown to be the traditionally low accuracy of the scoring functions \cite{1362}.

\subsubsection*{Works done by others}

Classical scoring functions, e.g. Cyscore \cite{1372}, are defined by the assumption of a fixed functional form to relate the predicted binding affinity to the numerical features that characterize the protein-ligand complex. They often employ standard multivariate linear regression (MLR) on experimental data to calibrate the coefficients in a weighted sum of physically meaningful terms as an estimation of binding affinity. Recent years have seen a growing number of new developments of machine-learning scoring functions, with RF-Score \cite{564} being the first that introduced a large improvement over classical approaches.

\subsubsection*{Works done by us}

First, we investigated under what circumstances machine-learning scoring functions would outperform classical ones \cite{1432}, and found that this is the case when there are sufficient numerical features and training samples. Based on this finding, we improved RF-Score by incorporating six additional features derived from Vina \cite{595} and by expanding the training set to all available structures in the refined set of PDBbind \cite{1633}. This led to the release of our RF-Score-v3 \cite{1647}.

Next, we studied the impact of docking pose generation error on the accuracy of machine-learning scoring functions \cite{1434}, and proposed a procedure to correct a substantial part of this error which consists of calibrating the scoring functions with re-docked, rather than co-crystallised, poses. As a result, test set performance after this error-correcting procedure is much closer to that of predicting the binding affinity in the absence of pose generation error. This led to the release of our RF-Score-v4.

Last but not least, we demonstrated for the first time that training with low-quality structural and interaction data can still improve predictive performance \cite{1663}, contrary to the widely-held belief that additional performance can only be gained from high-quality data.

\subsection*{Real-world applications of our toolset in anticancer drug repurposing}

We worked on anticancer drug repurposing jointly with a clinical biology team from Kunming Medical University (the Co-I's affiliation). In this collaboration, we utilized our toolset and successfully identified four approved drugs as potential inhibitors of cyclin-dependent kinases (CDK) 2/4/6, which have been long established as key factors regulating the cell cycle and hallmarks for cancers. Subsequent biological assays \textit{in vitro} in various cancer cell lines and \textit{in vivo} in nude mice further established their roles for the treatment of hepatocellular carcinoma. Our results have been published for the first two drugs \cite{1606}. Meanwhile, we have filed a patent in China for the third drug. The patent was issued on 17 Aug 2015 with an application ID of 201510501028.8. We believe our findings may present immediately-applicable clinical therapies for the treatment of liver cancer, one of the leading causes of cancer-related deaths worldwide.

\section*{Research Plan and Methodology}

In this project, we propose to unify and comprehend our toolset, with the ultimate goal to fully automate the early phases of modern drug discovery process, and use it to study and combat a vast variety of diseases beyond just cancers. Practically we will apply our toolset to the discovery of novel inhibitors of colorectal, ovarian, prostate cancers, as well as influenza A, herpes, HBV viruses.

\section*{Data collection and curation}

TCM
Thai herbals
South African products

\subsection*{New tool development and benchmark}

Specifically, we plan to develop new tools 1) that can intelligently adjust a compound’s chemical components with knowledge from binding pockets to lead to stronger potency and selectivity against a protein, 2) that can predict the on- and off-targets of a compound based on its 3D shape similarity to other annotated compounds, 3) that can perform ultrahigh-throughput virtual screening of large-scale molecular databases.

\subsubsection*{GPU hardware acceleration}

idock 3
agrep \cite{1138}

\subsubsection*{Shape and pharmacophore matching}

Molecular shape has been widely acknowledged as a key factor for biological activity and is thus regarded as a very important pattern for drug searching. Searching a molecular database for compounds that resemble the shape of a given query molecule, be it a known inhibitor of a target protein, a natural product, or even a patented compound, finds applications in ligand-based virtual screening (LBVS) \citep{1332,1380,1281,1504,1502,1615} and target fishing \citep{1528,1407,1408,1402}.

USR (Ultrafast Shape Recognition) \citep{1379} was the first non-superposition method for molecular shape comparison, and exhibited superior computational performance at least three orders of magnitude faster than previously existing alignment-based methods. Since then, there have been a few extensions \citep{1333,1436,1437,1334,1335,1337,1338,1331,1407,1408} to augment the original method, but only two of them \citep{1436,1437} have been made available as web servers that accompany compound databases for LBVS purposes. They both employ the UFSRAT \citep{1436} method for ligand similarity search.

Nevertheless, as pointed out in \citep{1331}, UFSRAT is incapable of discriminating between long, chain-like molecules such as certain heteropeptides and long alkylchains because aromaticity is not considered as a pharmacophoric subset; besides, calculating the four centroids for each set of atoms individually is problematic because either the parameters cannot be calculated at all or the underlying distance distributions are not with respect to the overall shape of a molecule and not meaningful when some pharmacophoric features are rarer than others.

We ported USR \citep{1379} and USRCAT \citep{1331} to our istar platform, and exploit Intel's AVX-256 SIMD intrinsics to accelerate similarity score computation. Unlike idock, our USR@istar does not require the availability of protein structure, hence remarkably expanding the applicability domain and allowing jumping out of the so-called ``patented chemical space". By preloading the calculated features, our implementation is able to search the entire database of as many as 94 million conformers of 23 million compounds in just two seconds.

\subsection*{New target identification}

CDK2,4,6, PI3K, EGFR, FGFR3

\subsection*{Prospective applications}

In summary, our toolset is the product of four years’ in-depth pragmatic research in the area of computer-aided drug discovery, and it keeps evolving. Its usefulness has been vigorously demonstrated, both retrospectively and prospectively. Most importantly, it can produce high-impact and patentable findings that could possibly save human lives.

By using our novel toolset, we have successfully identified the marketed drug fluspirilene as an immediately applicable clinical therapy for the treatment of hepatocellular carcinoma, hopefully saving millions of human lives. Furthermore, our toolset can be utilized in much wider areas, e.g. virus research, showing its great potential in large-scale computer-aided drug discovery and high-impact applications in real world problems.

\newpage
\linespread{0.5}
\tiny
\bibliographystyle{unsrtnat}
\bibliography{../refworks}

\end{document}

