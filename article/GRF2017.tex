\documentclass[a4paper,12pt]{article}
\usepackage[margin=1in]{geometry}
\usepackage{times}
\usepackage[compact]{titlesec}
\usepackage[numbers,sort&compress]{natbib}
\linespread{1.0}

\begin{document}

\title{Personalized Oncology Supported by Next-Generation Computer-Aided Drug Repositioning, Ensemble Docking, Interaction Profiling, Molecular Visualization and Interdisciplinary Laboratory Validations}
\maketitle

\section*{Primary Field}

E2 2209 Bioinformatics

\section*{Secondary Field}

M2 1204 Cancer

\section*{Keywords}

Personalized medicine, anticancer therapy, drug repurposing, molecular docking, virtual screening, interaction profiling, molecular visualization, cell assay, mice test

\section*{Abstract}

Developing a new drug could cost as much as US\$2.6B over 13.5 years. Both the industry and academia are intensively demanding robust computational methods to facilitate or even automate the process of modern drug discovery. This is termed computer-aided drug discovery (CADD).

CADD tools are sometimes used as a metaphor for medicinal chemists' bread and butter, reflecting their important role in R\&D. Unfortunately, most CADD tools are commercial, proprietary, not portable, or difficult to use. This reality has inevitably imposed a huge obstacle for R\&D productivity. Therefore in the past six years, we have been contributing to creating a free, open source, next-generation CADD toolset, aiming to substantially simplify the use of such tools while taking advantages of the latest methodological advancements.

To this end, our first attempt was the development of a heterogeneous platform named istar, which encapsulated high-throughput virtual screening, machine-learning binding affinity estimation, molecular interaction profiling and visualization into an automatic and unified pipeline, freely available at http://istar.cse.cuhk.edu.hk. Since it was launched in October 2013, istar has served 24583 web sessions to 11507 users from 2201 cities in 114 countries. These promising statistics demonstrate the utility of our previous work.

Our toolset was validated not only retrospectively on community benchmarks, but also prospectively in interdisciplinary real-world problems. Collaborating with an external team of clinical physicians, we have recently rediscovered seven drugs as anticancer agents, as they were shown to exhibit submicromolar inhibitory effects to colorectal, hepatocellular, and bladder carcinomas in vitro in assays and in vivo in mice. These newly identified medications of marketed drugs vigorously present an alternative and cheap clinical therapy for the treatment of cancers, hopefully saving the precious lives of millions of patients who cannot afford expensive imported drugs. We have filed patent applications for four of these novel medications.

Building upon our initial but successful work, in this project we propose to continue to enhance our in-house CADD toolset in four key aspects as explicitly stated in the project title: drug repositioning, ensemble docking, interaction profiling and molecular visualization. Briefly speaking, drug repositioning guarantees a fast and cheap route to new treatments with a low attrition rate, as most preclinical optimizations and toxicity assessments are satisfied; ensemble docking would generate reliable binding poses as it takes into accounts structural variability of multiple protein conformations, and it also allows to focus on a patient-specific mutant subtype of the disease on a personalized basis; interaction profiling and molecular visualization permit to inspect and reveal the underlying binding mechanism through which the drug becomes effective, and also shed light on the cause of drug resistence at a fine-grained molecular level. We carefully choose these four state-of-the-art technologies to study and improve because we firmly believe that they will, when properly integrated, constitute a powerful and unprecedented approach to modern drug discovery success.

Since the proposed approach is for general purposes, its medical applications are not limited to combating cancers, but could be expanded to treat other common diseases such as hyperuricemia, which we are already working on, provided that the structure of the protein of therapeutic interest is available, which is the only requirement of using our toolset. Fortunately, such macromolecular structural data are rapidly growing year by year. For popular proteins, their structures are usually experimentally solved. For rare proteins, their structures can be computationally modeled. In either case, the only prerequisite for user input is likely to be circumvented, thus we feel confident that the proposed toolset will be of wide use.

\section*{Long-term Impact}

This proposal is designed to significantly increase drug discovery success rate in a pragmatic and practical manner. Our proposed next-generation CADD toolset will be able to reveal the fundamental mechanism that governs molecular binding, and suggest ways to strengthen or weaken such binding. With this essential knowledge, a considerably higher success rate can be anticipated when it comes to constructing novel compounds from molecular fragments, optimizing chemical scaffolds, improving target selectivity, minimizing off-target side effects, finding new indications of approved drugs, or even predicting synergistic outcome of drug combinations. A notable long-term impact will be that more alternative and cheap drug therapies will become available for a broad range of patients who will otherwise have to opt for expensive treatments.

\section*{Project Objectives}

To collect and curate compound data from multiple external sources, including approved, experimental and withdrawn drugs (not only in US, but also in Europe, UK, Canada, Japan), regulated chemicals, herbal isolates, traditional Chinese medicines, natural products, and easily synthesizable compounds.

To develop and release a new version of our molecular docking software named idock, with atomic contact parameters and scoring function terms tuned to more consistently produce low-energy conformations geometrically closer to the co-crystallized conformation.

To debug and revise the protein-ligand interaction profiling algorithm currently implemented in our molecular visualizer named iview, and verify the profiler using a database of known and putative interactions in the structural human proteome.

To rewrite, modulize, accelerate and standardize iview by exploiting the new features of the latest specification of JavaScript, codenamed ECMAScript 6.

To create a new web server, tentatively to be named edock, to streamline automatic ensemble docking of various types of compounds, by seamlessly integrating the above-mentioned data sources, docking methods, profiling algorithms and visualization features.

To understand and elaborate the binding mechanism and analyze the binding patterns at fine-grained molecular or atomic level, and to suggest directions to strengthen or weaken intermolecular interactions.

To utilize edock in identifying novel inhibitors of selected oncoproteins of patient-specific subtypes to achieve personalized medicine for oncology.

To test and evaluate the efficacy and cytotoxicity of candidate anticancer compounds in vitro and in vivo, and file patent applications for those showing significant inhibitory activities.

\section*{Background of Research}

Computer-aided drug discovery (CADD) has now been recognized as a cost- and time-efficient strategy alternative to purely biochemical approaches. Robust and reliable computational tools and methods are highly demanded by the pharmaceutical industry in order to automate or simplify the early phases of modern drug discovery prior to preclinical experiments, and they have been thriving over the past decade.

From a practical view, we observe that most CADD tools are either commercial, proprietary, not portable, or difficult to use and update. This imposes a great obstacle on users, and hinders the progression of new drug discovery. Often, these tools are evaluated merely on retrospective benchmarks. Their performance in prospective applications remains unknown in many cases. Some tools are declared dead upon their initial release due to zero maintenance afterwards.

Therefore in the past five years, we have been constructing a next-generation CADD tsmbclient //pc91106/collectedoolset, which encapsulates several modules: 1) idock \citep{1153}, a high-throughput molecular docking program, 2) istar \citep{1362}, a web platform for large-scale chemoinformatics, 3) iview \citep{1366,1265}, an interactive WebGL visualizer, 4) RF-Score versions 3 and 4 \citep{1432,1647,1434,1663}, accurate scoring functions for predicting intermolecular binding affinity, 5) iSyn \citep{1409,1387}, an automatic drug design tool, and 6) USR@istar, a 3D molecular shape- and pharmacophore-matching web server. It is important to highlight that our tools are all free and open source, and well maintained and recognized. Thus far, our web server at http://istar.cse.cuhk.edu.hk has served \textgreater48,000 page views by \textgreater7000 users from \textgreater90 countries worldwide, according to Google Analytics.

Meanwhile, we have also conducted empirical research. We have found the limiting factor of predictive performance of classical scoring functions, and shown that substituting multiple linear regression by machine-learning algorithms as well as incorporating low-quality structural and interaction data can substantially increase the accuracy of binding affinity prediction.

In addition to tool development, we have also collected and curated several drug databases, which are essential data sources for inputting to our tools. On one hand, we collected a huge database of more than 23 million compounds in 3D format, together with annotations of their molecular properties. On the other hand, we manually curated several catalogs of approved drugs, resulting in a unified collection of clinically approved drugs, not only in US, but also in Europe, Canada and Japan, together with purchasing information available.

Most importantly, collaborating with a clinical biology team, we have prospectively utilized our toolset and successfully repurposed four approved drugs that exhibited submicromolar inhibitory effects to colorectal or hepatocellular carcinoma, both \textit{in vitro} and \textit{in vivo} \citep{1667,1681}. These newly identified medications of marketed drugs, which have a history of safe human use, may present an immediately-applicable clinical therapy for the treatment of colon and liver cancers, hopefully saving millions of lives. These inspiring results have thus vigorously demonstrated the validity and utility of our toolset. For one of the four drugs, we have filed a patent application in China (application ID: 201510501028.8).

In the following subsections, we outline the works done by others and their deficiencies, followed by the works done by us and their strengths, which are categorized by sub-areas of CADD.

\subsection*{Molecular docking and scoring}

Protein-ligand docking is a computational method that predicts how a small molecule, termed ligand, binds to a target protein, as well as how strongly they bind. Hence docking is useful in elaborating intermolecular interactions and enhancing the potency and selectivity of binding.

Very often, once a target protein of interest is identified from a pathway, a large database of ligands will be docked against the protein. This is to shortlist the ligands that are predicted to show the strongest binding affinity towards proteins intended to be inhibited, or the ligands that are predicted to show the weakest binding affinity towards proteins intended not to be inhibited. This process is known as structure-based virtual screening (SBVS).

\subsubsection*{Works done by others}

There are dozens of docking tools available, e.g. \citep{595,607,617,650,596}. Among them, AutoDock Vina \citep{595} is a competitive one not only because it is free and open source, but also because it has been shown to substantially improve the average accuracy of the binding mode predictions \citep{595} and run faster than its counterpart AutoDock 4 \citep{596} by an order of magnitude. Released in the second half of 2010, Vina has been cited over 2,500 times and adopted by a large community of users.

% However, Vina has to parse the input protein structure and create energy grid maps every time it attempts to dock a single ligand. This limits its performance in high-throughput virtual screening. Besides, this software is no longer maintained by its developers.

\subsubsection*{Works done by us}

In 2011, we developed a new docking tool called idock \citep{1153}, which substantially revises the numerical approximation model from linear interpolation to table loookup and enhances the fundamental implementation of various components with modern C++14 tricks, such as multithreading. Compared with Vina, our idock obtained a speedup of 3.3x in terms of CPU time and a speedup of 7.5x in terms of elapsed time on average, making it particularly suitable for carrying out large-scale virtual screening.

\subsection*{Web server and databases}

Standalone docking tools are usually for a small portion of users only, as they require intensive settings. Medicinal chemists generally seek for web servers that are easy to use. One obvious advantage of web servers over standalone tools are the transparent handling of internal logics.

\subsubsection*{Works done by others}

A few online docking platforms exist. DOCK Blaster \citep{557} investigates the feasibility of full automation of protein-ligand docking. It utilizes DOCK \citep{1222} as the docking engine and ZINC \citep{532,1178} as the ligand database. iScreen \citep{899} is a compacted web server for TCM (Traditional Chinese Medicine) docking and followed by customized \textit{de novo} drug design. It utilizes PLANTS \citep{610,607,779} as the docking engine and TCM@Taiwan \citep{528} as the ligand database.

Nevertheless, these platforms neither support property-based ligand selection, nor are able to monitor job progress in real time. They also lack post-docking analysis, a hurdle that prevents users from studying and understanding the binding mode of candidate compounds.

\subsubsection*{Works done by us}

In 2012, we designed a modern web platform called istar \citep{1362}, which is freely available at http://istar.cse.cuhk.edu.hk. We believe istar is a remarkable improvement to existing web servers because it hides implementation details of data preparation, format conversion, docking, analysis and visualization, and thus provides the whole set of functionalities as an automatic pipeline, which considerably relaxes the requirements on users' knowledge and skills.

As a byproduct, we collected a huge database of more than 23 million small-molecule compounds, covering a large space of chemical diversity. This is thus far the largest open source web server ever available for performing structure-based virtual screening. To ensure smooth operations of istar, we utilized the departmental HK\$4M-worth supercomputing infrastructure equipped with 736 CPU cores and 79,872 GPU cores.

\subsection*{Molecular visualization}

Visualization plays an important role in observing and elaborating protein-ligand interactions and aiding novel drug design. It helps to learn new binding patterns and preferences.

\subsubsection*{Works done by others}

VMD \citep{1220}, PyMOL (http://www.pymol.org) and Chimera \citep{1219} are well-known and highly-cited interactive visualization tools. PoseView \citep{748} and LigPlot+ \citep{951}, on the other hand, detect protein-ligand interactions from 3D coordinates and plot static 2D diagrams.

In addition to the above standalone visualizers, there are web-based visualizers to enable online use. Jmol (http://www.jmol.org) and JSmol \citep{1314} support advanced features, but they rely on software rendering, which is slow on large display areas and thus prevents detailed inspection of the structure. In contrast, GLmol (http://webglmol.sourceforge.jp) is a WebGL molecular viewer that exploits hardware acceleration and supports multiple representations.

However, none of these web visualizers are tailor-made for virtual screening. Most of them either suffer from slow software rendering, or lack the support of macromolecular surface construction. The useful feature of virtual reality is also unavailable.

\subsubsection*{Works done by us}

In 2014, we developed an interactive WebGL visualizer called iview \citep{1366,1265}, permitting easy accessibility and platform independence. It exploits hardware rendering, and supports macromolecular surface representations as well as special effects in virtual reality settings \citep{1265}. The unique feature that distinguishes iview from other visualizers is its inherent support for docking input preparation and result analysis with detection of binding interactions. We have seamlessly integrated iview into our istar platform, freely available at http://istar.cse.cuhk.edu.hk/iview.

\subsection*{Binding affinity prediction}

Docking consists of two main operations: predicting the conformation of a ligand when docked to the protein's binding site, and predicting their binding affinity. The single critical limitation of docking has been shown to be the traditionally low accuracy of the scoring functions \citep{1362}.

\subsubsection*{Works done by others}

Classical scoring functions, e.g. Cyscore \citep{1372}, are defined by the assumption of a fixed functional form to relate the predicted binding affinity to the numerical features that characterize the protein-ligand complex. They often employ standard multivariate linear regression (MLR) on experimental data to calibrate the coefficients in a weighted sum of physically meaningful terms as an estimation of binding affinity. Recent years have seen a growing number of new developments of machine-learning scoring functions, with RF-Score \citep{564} being the first that introduced a large improvement over classical approaches.

\subsubsection*{Works done by us}

First, we investigated under what circumstances machine-learning scoring functions would outperform classical ones, and found that this is the case when there are sufficient numerical features and training samples \citep{1432}. Based on this finding, we improved RF-Score by incorporating six additional features derived from Vina \citep{595} and by expanding the training set to all available structures in the refined set of PDBbind \citep{1633}. This led to the release of our RF-Score-v3 \citep{1647}.

Next, we studied the impact of docking pose generation error on the accuracy of machine-learning scoring functions \citep{1434}, and proposed a procedure to correct a substantial part of this error which consists of calibrating the scoring functions with re-docked poses, rather than co-crystallised poses. As a result, test set performance after this error-correcting procedure is much closer to that of predicting the binding affinity in the absence of pose generation error. This led to the release of our RF-Score-v4.

Last but not least, we demonstrated for the first time that training with low-quality structural and interaction data can still improve predictive performance \citep{1663}, contrary to the widely-held belief that additional performance can only be gained from high-quality data.

\subsection*{Molecular shape recognition}

USR-VS \citep{1749}
TopMap \citep{1675}

\subsection*{Real-world applications of our toolset in anticancer drug repurposing}

We worked on repurposing approved drugs as anticancer agents jointly with a clinical biology team from Kunming Medical University (the Co-I's affiliation). In this collaboration, we utilized our tools and databases, and successfully identified four approved drugs as potential inhibitors of cyclin-dependent kinases (CDK) 2/4/6, which have long been established as key factors regulating the cell cycle and hallmarks for cancers. Subsequent biological assays \textit{in vitro} in various cancer cell lines and \textit{in vivo} in nude mice further established their roles for the treatment of colorectal \citep{1667} and hepatocellular \citep{1681} carcinomas. We have published our results for the first two drugs \citep{1667,1681}, and filed a patent in China for the third drug. The patent was issued on 17 Aug 2015 with an application ID of 201510501028.8. We believe our findings may present immediately-applicable clinical therapies for the treatment of colon and liver cancers.

\section*{Research Plan and Methodology}

Two main goals of this project are to unify and comprehend our CADD databases and tools, and to use them to study and combat a vast variety of real-world diseases. In the following subsections, we describe our plans for collecting and cleansing data, inventing and benchmarking new tools, and prospective applications to anticancer drug discovery.

\section*{Data collection and curation}

% ZINC15 \citep{1688}, DrugBank \citep{1594}, SWEETLEAD \citep{1511}, TTD \citep{1790}, WITHDRAWN \citep{1718}, PubChem \citep{1701}
% TCM Database@Taiwan \citep{528}, TCMSP \citep{1375}, SANCDB \citep{1680}
% Create a table to list these databases, number of compounds. Give statistics and histogram distributions in journal paper.
% Automatic data synchronization via bash scripts. SMILES to SDF to mol2 to pdbqt
% three-step curation. Copy from the adw paper.
% All are publicly available free databases.
% Such data scheme is easy to extend, to include future databases.
% This project is innovative in that different sources of compound data will be analyzed, which represents a new opportunity given the recent availability of compiled and curated sets of such molecules.
% Physiochemical properties will be retained.

We have been maintaining two practically useful datasets: a curated library of approximately 3000 clinically approved drugs with their purchasing information, and a huge database of more than 23 million diverse compounds with their molecular properties.

Here we propose to supplement our existing data with three more sources: traditional Chinese medicines \citep{528} (http://tcm.cmu.edu.tw) and South African natural products \citep{1680} (https://sancdb.rubi.ru.ac.za). They are chosen because it is well known that medicinal plants and natural products would generally cause fewer side effects than western drugs, which implies a higher success rate and a shorter time to marketing.

Additionally, we will also incorporate SCUBIDOO \citep{1682}, a database of computationally created chemical compounds optimized toward high likelihood of synthetic tractability. 

We will first filter out the compounds having missing or incomplete information (e.g. without a SMILES string, nor a 2D structure), as well as duplicate ones by comparing their canonical SMILES representation. Next, we will use molecular tools or libraries, e.g. RDKit (http://rdkit.org), to sample 3D conformers in low energy states and calculate their molecular properties. Finally we will perform format conversion, aggregate individual records into a unified dataset, and create an interface and an index for database query. % \citep{1127}

These new datasets will be publicly available for download upon project completion so as to promote academic data exchange and accelerate the progress of other drug discovery efforts.

\subsection*{Tool development and evaluation}

% Ensemble docking, ES6 visualization
% QuickVina \citep{1193}, VinaMPI \citep{1329}, QuickVina 2 \citep{1664}, hybrid \citep{1716}, Vinardo \citep{1741}, PSOVina \citep{1789}
% All are free and open-source software.
% Can apply ligand-based VS to the collected databases.
% Cite figures from latest drug discovery review papers, e.g. among 84 drug products introduced to market in 2013, new indications of existing drugs accounted for 20%.
%In our hands, roughly half of the systems we have used for docking simulations will give useful results using the default rigid model for the receptor \citep{1730}. In other cases, protein motion will cause problems for the prediction of reasonable poses. 
%For systems with larger motions of loops or domains, the relaxed complex method \citep{26} has shown success by sampling a variety of receptor conformations using molecular dynamics and then performing docking simulations on these snapshots.

We are currently providing free tools for various CADD purposes: molecular docking with idock \citep{1153}, web platform with istar \citep{1362}, molecular visualization with iview \citep{1366,1265}, binding affinity prediction with RF-Score-v3 and v4 \citep{1432,1647,1434,1663}, and drug design with iSyn \citep{1409,1387}.

Here we propose to extend our toolset by implementing new tools for next-generation CADD. They are explained in the following subsections.

\subsubsection*{Ultrahigh-throughput molecular docking}

Future work: GPU acceleration, gpu7 to gpu9 TITAN Z and K20m

Our state-of-the-art docking program idock, though capable of docking an average-sized ligand within just one second, is still far from satisfaction when millions of compounds have to be evaluated, as is the case of identifying novel compounds via large-scale virtual screening.

Therefore we propose a highly optimized implementation of idock by exploiting the 32 NVIDIA Tesla K20m GPU chips (79,872 CUDA cores in total) installed in our department. These programmable parallel processores provide extremely high computational throughput and tremendous memory bandwidth compared to conventional CPU chips. Normally, GPU performance optimization revolves around three basic strategies: maximizing parallel execution, maximizing memory bandwidth, and maximizing instruction throughput.

\subsection*{Prospective applications in anticancer drug discovery}

% CDK2/4/6, FGFR3, EGFR, PI3K, ALK, Bcr-Abl, XDH

Practically we will apply our tools and datasets to the discovery of novel inhibitors of colorectal, hepatocellular, ovarian, and prostate cancers. Cyclin-dependent kinases (CDK) 2/4/6 have been widely acknowledged and thoroughly documented as key proteins regulating the cell cycle and hallmarks for cancers. Epidermal growth factor receptor (EGFR), fibroblast growth factor receptor (FGFR) 3, phosphoinositide 3-kinase (PI3K), and murine double minute 2 (MDM2) are other important therpeutic targets of cancers.

Based on our previous successful experience in repurposing clinically approved drugs as anticancer agents, we will first collect as many structures of these oncogenic targets as possible from the PDB database \citep{537} and analyze the desired binding cavity with our iview \citep{1366}. We will then extract the protein entity, convert file formats, and invoke our idock \citep{1153} to perform ensemble docking, which has the advantage of guaranteeing a consistent binding strength on average over multiple structures of the same oncogenic target with structural variabilities. Next we will rank the compounds, visualize their predicted conformations using iview \citep{1366}, analyze their putative binding interactions, survey their reported usages from literature, and shortlist candidates for wet verifications. (Incidentally, this whole computational pipeline can be automated via advanced scripting as soon as our next-generation CADD framework is developed.) Lastly we will conduct cell viability assays, cell apotosis assays, cell cycle assays, western blotting, clinical trials on animals, and finally on humans. We expect this project to be of great impact, as our findings could possibly save human lives in millions.

In the long run, beyond just cancers, we will utilize our toolset for computer-aided drug discovery in much wider areas, including but not limited to influenza, herpes, HBV and HIV viruses.

\section*{Existing facilities and major equipment available for this research proposal}

Existing facilities and equipments are sufficient to complete the proposed project. In CUHK’s department of computer science and engineering, which is the PI’s affiliation, there are high-performance computers. In KMU’s biotechnology center, which is the Co-I’s affiliation, there are state-of-the-art laboratories and equipments, and cancer cells and animal models.

\section*{Release of Completion Report and Data Archive Possibilities and Public Access of Publications Resulting from Research Funded by the RGC}

We will release the following datasets to the public upon project completion:
1) Curated collections of ~3000 approved drugs in US, Europe, Canada and Japan, with molecular properties and purchasing information.
2) 23 million compounds in 3D format, ready for large-scale docking-based virtual screening.
3) 1.7 billion conformers of 23 million compounds in 3D format, ready for large-scale shape-based virtual screening.

\section*{Education Plan}

We will vigorously train one postdoctoral fellow, two PhD students, and a few talented undergraduate students, to carry out theoretical and systematic research in the field of computer-aided drug discovery.

We plan to conduct seminars and tutorials every semester, and provide financial support for our trainees to attend prestigious conferences overseas.

Every year, the PI leads several groups of undergraduate students for the final year projects. We will choose some groups and instruct them to use and evaluate our toolset. Then we will show them how to refine existing functionalities and possibly implement new features under our supervision.

Meanwhile, the Co-I will also foster a PhD student to perform wet experiments such as cytotoxicity assays, western blotting and oral dosage of drugs in mice.

\newpage
\linespread{0.5}
\footnotesize
\bibliographystyle{unsrtnat}
\bibliography{../refworks}

%\includefigure

\end{document}
