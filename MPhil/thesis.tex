\documentclass[12pt]{report}
\usepackage{xltxtra}
\usepackage{setspace}
\usepackage{graphics}
\usepackage{thesis}
\usepackage{latexsym}
\usepackage[dvips]{epsfig}
\usepackage{subfigure}
\usepackage{booktabs}
\usepackage{multirow}
\usepackage{amsmath}
\usepackage[numbers,sort&compress]{natbib}
\DeclareGraphicsExtensions{.jpg,.png,.eps}

\topmargin	0.0cm
\textheight 21cm

\begin{document}

\thesistitle{A Computational Framework for Structure-Based Drug Discovery with GPU Acceleration}
\authorname{LI, Hongjian}
\degree{Master of Philosophy}
\programme{Computer Science and Engineering}
\submitdate{August 2011}
\coverpage

% Committee page
\thispagestyle{empty}
\vspace*{2cm}
\begin{center}
\vskip 4cm
\Large
Thesis/Assessment Committee
\vskip 2cm
\large
Professor WONG Kin Hong (Chair)\\
\vskip 0.2cm
Professor LEUNG Kwong Sak (Thesis Supervisor)\\
\vskip 0.2cm
Professor WONG Man Hon (Thesis Supervisor)\\
\vskip 0.2cm
Professor LUI Chi Shing John (Committee Member)\\
\vskip 0.2cm
Professor LEONG Hong Va (External Examiner)\\
\end{center}
\newpage

\pagenumbering{roman}

\abstractpage

% Abstract in Chinese
\XeTeXlinebreaklocale "zh"
\newfontfamily\chinesefont{NSimSun}
\addcontentsline{toc}{chapter}{Abstract in Chinese}
\vspace*{1cm}
\large \noindent
\begin{center}
{\chinesefont 摘要}\\
\end{center}
\vskip 1cm \noindent
{\chinesefont 藥物研發是一項昂貴和長期的商業。開發一種新藥需要18億美元和13年半的時間。因此,我們意圖針對藥物研發開發一個計算框架,當中利用圖形處理器實現加速,從而模擬現代藥物研發的初始階段,以節省金錢和時間。

從計算的角度而言,這種模擬一般涉及DNA樣式的模糊匹配、基於結構的虛擬篩選、以及強效藥物的計算生成。到目前為止,我們為此開發了三個工具。

第一個工具叫CUDAagrep,是用來快速模糊匹配DNA樣式。它利用了NVIDIA圖形處理器強大的計算能力。相比於OpenMP的實現,CUDAagrep有著70倍的速度提升。相比於Bowtie和BWA,CUDAagrep的敏感度在單端模式下有25\%的提升,在雙端模式下有18\%的提升。

第二個工具叫idock,是用於基於結構的快速虛擬篩選。它預測小分子和給定蛋白質的粘合構象以及粘合程度。相比於AutoDock Vina,idock有著6.3倍至10.4倍的速度提升,每CPU分鐘可以篩選1.3個類似藥物的小分子。

第三個工具叫SmartGrow,是用來計算生成強效藥物。它通過組合小分子塊來生成藥物。相比於AutoGrow,由SmartGrow生成的藥物平均有著100 Da更低的分子量。再者,SmartGrow有著30\%的速度提升。

我們已經使用了這三個工具,連同其他一些現有工具,為艾滋病和老人癡呆症研發潛在的新藥。有部分新藥已經被計算篩選和生成,以後將進行臨床研究。

這三個工具的開發只是一個開始。最終我們要開發一個全面的統一的計算框架,當中包括粘合位置的識別、分子對接、虛擬篩選、藥物生成、藥性預測、以及交互式的可視化。最終也是最重要的一點,我們應當利用此框架去研發新藥,以拯救數以百萬計的人類生命。}
\newpage
\XeTeXlinebreaklocale "en"

\acknowledgementpage
\dedicationpage

\tableofcontents
\listoffigures
\listoftables

\newpage
\setcounter{page}{0}
\pagenumbering{arabic}
\pagestyle{headings}

\doublespacing

\input ./Introduction/Introduction.tex
\input ./Background/Background.tex
\input ./SequenceMatching/SequenceMatching.tex
\input ./VirtualScreening/VirtualScreening.tex
\input ./LigandSynthesis/LigandSynthesis.tex
\input ./Conclusion/Conclusion.tex

\appendix
\input ./publications.tex

\singlespacing

\newpage
\addcontentsline{toc}{chapter}{Bibliography}
\bibliographystyle{plainnat}
\bibliography{thesis}

\end{document}
