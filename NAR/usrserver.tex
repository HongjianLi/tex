\documentclass[a4,center,fleqn]{NAR}

% Enter dates of publication
\copyrightyear{2015}
\pubdate{31 July 2016}
\pubyear{2016}
\jvolume{37}
\jissue{12}

%\articlesubtype{This is the article type (optional)}

\begin{document}

\title{USR-VS: a webserver for large-scale prospective virtual screening using ultrafast shape recognition techniques}

\author{
Hongjian Li\,$^{1}$,
Kwong-Sak Leung\,$^{1}$,
Man-Hon Wong\,$^{1}$
and Pedro J. Ballester\,$^{2}$
\footnote{To whom correspondence should be addressed.
Tel: +33-486-977-265; Fax: +33-486-977-499; Email: pedro.ballester@inserm.fr}}

\address{
$^{1}$Department of Computer Science and Engineering, Chinese University of Hong Kong, Sha Tin, New Territories, Hong Kong.
and
$^{2}$Cancer Research Center of Marseille, INSERM U1068, F-13009, Marseille, France. Institut Paoli-Calmettes, F-13009, Marseille, France. Aix-Marseille Université, F-13284, Marseille, France. CNRS UMR7258, F-13009, Marseille, France.}
% Affiliation must include:
% Department name, institution name, full road and district address,
% state, Zip or postal code, country

\history{
Received January 1, 2015;
Revised February 1, 2016;
Accepted March 1, 2016}

\maketitle

\begin{abstract}

Finding molecules geometrically and pharmacophorically similar to a query molecule has been an important but daunting problem for a long time. The USR (Ultrafast Shape Recognition) algorithm represents a whole new alignment-free method that encodes the shape information semantically and permits superfast screening of a large molecular database. Several extensions to USR improve the original method from various perspectives and three of them, UFSRAT, EDULISS and SwissTargetPrediction, are also available as web servers. However, UFSRAT and EDULISS are unable to discriminate between long, chain-like molecules, and their calculated distributions are not meaningful when some pharmacophoric features are rarer than others. SwissTargetPrediction uses a small, well annotated, bioactive compound database and is generally for target fishing.

For prospective virtual screening purposes, in this study we have developed a highly optimized implementation of USR and its extension USRCAT (USR with Credo Atom Types). We used a large molecular library consisting of 23,129,049 diverse compounds collected from the ZINC database, and generated 93,903,333 low-energy conformers. We exploited multithreading to accelerate query execution, resulting in a lightning screening throughput of 55 millions 3D conformers per second, including calculation of scores, ranking of molecules, and writing of results. Our USR-VS supports query molecules in SDF format, and interfaces with our iview WebGL visualizer for interactive visualization of high-score hits. USR-VS is freely available at http://usr.marseille.inserm.fr (version 2) and http://istar.cse.cuhk.edu.hk/usr (version 1).

\end{abstract}

% **************************************************************
% Keep this command to avoid text of first page running into the
% first page footnotes
\enlargethispage{-65.1pt}
% **************************************************************

\section{Introduction}

Molecular shape has been widely acknowledged as a key factor for biological activity and is thus regarded as a very important pattern for drug searching. Searching a molecular database for compounds that most closely resemble the shape of a given query molecule, be it a known inhibitor of a target protein, a natural product, or even a patented compound, finds pragmatic applications in ligand-based virtual screening \cite{1332,1380,1281,1504,1502,1615} and target fishing \cite{1528,1407,1408,1402}. Therefore molecular similarity search can assist in discovering structurally novel active compounds, or in identifying potential interacting target of bioactive ligands, which is useful for understanding the polypharmacology and safety profile of existing drugs. Furthermore, this approach can be applied to other scientific disciplines such as performing similarity comparisons between proteins or designing content-based Internet search engines for 3D geometrical objects \cite{1280}.

Molecular shape similarity can be quantified by methods based on structural alignment \cite{1440,887,1439,1534} or shape recognition \cite{1379,1338,1331,1675}. Structural alignment is also known as molecular superposition, and requires precise geometric comparison, which is often computationally demanding. Shape recognition, on the other hand, encodes shape information into a numerical feature vector, which can be subsequently used to compute a similarity score between two molecules very efficiently.

USR (Ultrafast Shape Recognition) \cite{1379} was the first non-superposition method for molecular shape comparison, and demonstrated superior computational performance at least three orders of magnitude faster than previously existing alignment-based methods. USR has another major advantage of being invariant to spatial rotation and translation, and hence circumvents the problematic requirement of aligning molecules. USR defines the shape of a molecule independently and for every molecule uses a fixed set of 12 descriptors derived from the first 3 statistical moments of distributions of interatomic distances between atoms and 4 purposely-selected centroids. This encoding scheme ensures that every molecule has a unique location in the 12-dimensional chemical space spanned by the used descriptors, and consequently enables finding and visualizing clusters of molecules with similar shapes \cite{1280,1332}. Selecting the most representative molecule of each cluster can avoid repeating expensive biological tests on similar molecules \cite{1280}. The ability of USR as a standalone method was studied to identify molecules sharing common biological activities through retrospective \cite{1332} and prospective \cite{1380,1281,1504,1502,1615} virtual screening experiments. Prospectively, USR was applied to the discovery of inhibitors of arylamine NATs (N-acetyltransferases) \cite{1380}, DHQase2 (dehydroquinase type 2) \cite{1281}, PAD4 (protein arginine deiminase type 4) \cite{1504}, p53-MDM2 (murine double minute 2) \cite{1502}, and PRL-3 (phosphatase of regenerating liver) \cite{1615}. USR was also used for deduplication in a virtual screening campaign \cite{1390} and in our iSyn \cite{1409,1387} \textit{de novo} ligand design software.

Since USR was devised in 2007, there have been quite a few extensions \cite{1333,1436,1437,1334,1335,1337,1338,1331,1407,1408,1675} to augment the original method. \cite{1333} presented a hybrid approach composed of USR and the topological MACCS key descriptors, which are binary in nature and encode the presence or absence of 166 predefined structural fragments. It used the first four unbalanced moments of each distribution of atomic distances and incorporated additional chemical information through 2D structural similarity. Random Forest \cite{1309} was used for multi-class classification. Incorporating an additional central moment, the kurtosis, was found to significantly improve the performance. The addition of the fifth central moment, however, did not improve the performance sufficiently to justify the increased computational expense.

UFSRAT \cite{1436} addressed the lack of discrimination between compounds having similar shape but distinct pharmacophoric features by subdividing atoms into four subsets which are heavy, hydrophobic, hydrogen bond acceptor or donor atoms, according to their atom types. For each subset, the four centroids were calculated, and so were the 12 USR descriptors. Therefore 48 descriptors were resulted. This was to ensure that similar compounds are able to make the same type of interactions within biological systems as the query molecule. UFSRAT was prospectively applied to the discovery of inhibitors of 11$\beta$-HSD1 (hydroxysteroid dehydrogenase type 1) \cite{1505}. UFSRAT is available as a web server at http://opus.bch.ed.ac.uk/ufsrat/, which contains 28 databases with the largest one containing 4,853,000 conformers. UFSRAT is also employed for geometrical similarity searches in the EDULISS database \cite{1437} available at http://eduliss.bch.ed.ac.uk/, which comprises over 5 million commercially available compounds.

CSR \cite{1334} and USR:OptIso \cite{1335} attempted to tackle the lack of discrimination between chiral compounds. Their novel idea was to position the centroids in such a way that they clearly distinguish between enantiomers, i.e. optical isomers. They both used cross product because it is an operator that transforms equivariantly under rotations and translations, but not under reflections. The two methods differed in selecting the centroids and in replacing or supplementing the new optical isomerism descriptor \cite{1335}. CSR \cite{1334} was tested on the DUD (Directory of Decoys) dataset \cite{87}, where a significant improvement in enrichment was found over USR. USR:OptIso \cite{1335} was shown to be helpful for analyzing molecules with stereogenic centers, atropisomerism, and in the clustering of conformers generated by systematic bond rotation.

ElectroShape \cite{1337,1338} extended the CSR \cite{1334} method by encoding electrostatics and liphophilicity through additional dimensions and centroids. In \cite{1337}, the partial charge was represented as a fourth coordinate, with atoms being identified by points in four-dimensional space. ElectroShape was validated using release 2 of the DUD dataset \cite{87}, and showed a near doubling in enrichment over USR and CSR. Different implementations of partial charge were also revealed to affect the enrichment performance significantly. The addition of a fourth statistical moment, as was done in \cite{1333}, improved USR and CSR but not ElectroShape, suggesting that adding extra information might not necessarily improve enrichment but could dilute the information already included. In \cite{1338}, ElectroShape was further extended by using atomic lipophilicity as an additional molecular property, with atoms being identified by points in five-dimensional space. This version of ElectroShape showed a clear improvement in performance, indicating that adding extra independent atomic properties makes shape-based enrichments even better.

USRCAT \cite{1331} extended the UFSRAT \cite{1436} method by identifying five pharmacophoric subsets of atoms with the help of the SMARTS patterns used for atom typing in the CREDO database \cite{522,1530}. The five subsets were chosen to be heavy, hydrophobic, aromatic, hydrogen bond acceptor or donor atoms. Aromaticity was added to USRCAT as a pharmacophoric subset because USR was unable to discriminate between long, chain-like molecules such as certain heteropeptides and long alkylchains in particular. Unlike UFSRAT \cite{1436}, USRCAT \cite{1331} derived the four centroids from heavy atom coordinates and used them to calculate the distributions for all the five subset moments to improve screening performance. USRCAT was shown to outperform the traditional USR method in a retrospective virtual screening benchmark on the DUD-E (Directory of Decoys, Enhanced) dataset \cite{1185}. The highest enrichment factors were only achieved if the LEC (Lowest Energy Conformer) of an active was used as a query and if the LECs were included in the target set, but this observation could not be generalized. DUD-E was found to be not ideal to benchmark the virtual screening performance of global shape similarity algorithms such as USR and its variants due to the large variations in molecular size of the active ligands. Another extension TopMap \cite{1675} adopted a similar approach but partitioned the molecular coordinates into different charge-tiers and used five descriptors of the shape distributions to construct a feature vector.

A recent study \cite{1407} used a reference set of 224,412 molecules active on 1,700 human proteins and showed that accurate target prediction can be achieved by using a multiple logistic regression to combine different measures of chemical similarity based on both chemical structure and molecular shape, with the former using FP2 fingerprints and the latter using ElectroShape \cite{1338}. This hybrid method was subsequently encapsulated into the SwissTargetPrediction \cite{1408} web server, freely available at http://www.swisstargetprediction.ch/, to identify new targets for uncharacterized molecules or secondary targets for known molecules. With data collected from the ChEMBL database version 16 \cite{1441}, the molecular library was expanded to 280,000 compounds active on 2,686 targets of the organisms of human, mouse, rat, cow and horse. Mapping predictions by homology within and between different species, a powerful approach to translate results obtained in model organisms to human, were enabled for close paralogs and orthologs.

Table \ref{Methods} summarizes USR and its variants.

\begin{table*}
\tableparts{
\caption{Summary of USR-like methods.}
\label{Methods}
}{
\begin{tabular*}{\columnwidth}{@{}lll@{}}
\toprule
method & novelty & references\\
\colrule
USR          & encoding shape by atomic distribution & \cite{1379}\\
USR+MACCS    & incorporating chemical similarity     & \cite{1333}\\
UFSRAT       & subdiving atoms into subsets          & \cite{1436,1437}\\
CSR          & repositioning reference locations     & \cite{1334}\\
USR:OptIso   & repositioning reference locations     & \cite{1335}\\
ElectroShape & expanding coordinate dimension        & \cite{1337,1338}\\
USRCAT       & subdiving atoms into subsets          & \cite{1331}\\
TopMap       & subdiving atoms into subsets          & \cite{1675}\\
SwissTargetPrediction & mixing ElectroShape and chemical similarity & \cite{1407,1408}\\
\botrule
\end{tabular*}
}
{}%This is a table footnote
\end{table*}

Among the USR variants mentioned above, only three of them \cite{1436,1437,1408} have been made available as web servers together with different databases for different purposes. The UFSRAT \cite{1436} and EDULISS \cite{1437} web servers both employ the UFSRAT \cite{1436} method for ligand similarity search. However, as pointed out in \cite{1331}, this method is incapable of discriminating between long, chain-like molecules such as certain heteropeptides and long alkylchains because aromaticity is not considered as a pharmacophoric subset; besides, calculating the four centroids for each set of atoms individually is problematic because either the parameters cannot be calculated at all or the underlying distance distributions are not with respect to the overall shape of a molecule and not meaningful when some pharmacophoric features are rarer than others. The UFSRAT \cite{1436} web server restricts the input query molecule to be one molecule in SDF format only, and does not support online visualization. The EDULISS \cite{1437} web server requires drawing a query structure in a Java molecular editor, which is being disabled on more and more systems due to security concerns. The SwissTargetPrediction \cite{1408} web server, on the other hand, comprises well-annotated active compounds and is primarily used for predicting the target proteins of bioactive small molecules but not for prospective virtual screening purposes.

In this project we aimed to provide a web server for large-scale prospective virtual screening using USR-like methods. We chose to employ USR \cite{1379} and USRCAT \cite{1331} because they have demonstrated pragmatic usefulness in prospective \cite{1380,1281,1504,1502,1615} and retrospective \cite{1331} virtual screening experiments, respectively, and their Python source code is freely available for studying their precise implementation and porting to other programming languages such as C++ and JavaScript.

Our USR-VS has several distinct features. First, a huge database of 23,129,049 diverse small molecules with vendor information was collected from ZINC \cite{532,1178} and 93,903,333 low-energy conformers were generated for prospective virtual screening. Second, a highly optimized and multithreaded implementation of the alignment-free methods USR \cite{1379} and USRCAT \cite{1331} was developed to realize a ultrahigh screening throughput of 55 million 3D conformers per second. Third, the hardware-accelerated WebGL visualizer iview \cite{1366} was integrated to permit interactive visualization and interpretation of the results without the use of Flash nor Java plugins.

\section{MATERIALS AND METHODS}

This section first reviews the general methods of USR \cite{1379} and USRCAT \cite{1331}, and then introduces their optimized and multithreaded implementation.

\subsection{USR and USRCAT}

USR is based on the observation that the shape of a molecule is uniquely determined by the relative position of its atoms, which is in turn determined by the set of all interatomic distances. This convenient representation is independent of molecular orientation or position, and thus eliminates any need for alignment or translation. The interatomic distances are heavily constrained by the forces that hold the atoms together, and hence they contain more than sufficient information to accurately describe molecular shape. So it is possible to use a set of atomic distances from only a small number of strategic reference locations uniquely defined in every molecule, and meanwhile retain the discriminative power necessary to distinguish between molecules.

The four reference locations are selected to be the molecular centroid (ctd), the closest atom to ctd (cst), the farthest atom to ctd (fct), and the farthest atom to fct (ftf). These locations represent the center of the molecule and its extremes, and are thus well separated. In this way molecular shape is described by four distributions of atomic distances, where the number of atomic distances is proportional to the number of atoms. In order to compare molecules with different number of atoms, the first three moments of these distributions are computed and used to encode the shape information instead. These moments have semantics indeed. For instance, the 1st, 2nd and 3rd moments of distribution of atomic distances to the molecular centroid (ctd) capture the size, variance and skewness of the molecule, respectively. Selecting the first three moments provides an excellent compromise between the efficiency and the effectiveness of the method. Finally the shape similarity score of two molecules is calculated through the sum of absolute differences of their respective moments.

In this study the first three moments are computed in the same way as in ElectroShape \cite{1337}, which is slightly different than the way used in USR \cite{1379,1332,1380} and USRCAT \cite{1331}. Mathematically, for a distribution of atomic distances $\{d_k\}_{k=1}^n$ to one of the four reference locations (ctd, cst, fct, ftf), where $n$ is the number of atoms, the first three moments are semantically the mean, the standard deviation, and the cube root of the third central moment, respectively. Their exact expressions are shown in equations \eqref{moment1}, \eqref{moment2} and \eqref{moment3}. The roots are intended to provide all moments with linear space dimension in \AA, unlike the skewness, for instance, which is unitless. This computation allows the distributions to contain only one sample, in which case the 2nd and 3rd moments will be zeros, i.e. $\mu_2=\mu_3=0$ when $n=1$. Likewise, when a distribution contains only two samples, the 3rd moment will be zero, i.e. $\mu_3=0$ when $n=2$.

\begin{equation}
\mu_1=\frac{1}{n}\sum_{k=1}^{n}{d_k}
\label{moment1}
\end{equation}

\begin{equation}
\mu_2=\sqrt[2]{\frac{1}{n}\sum_{k=1}^{n}{(d_k-\mu_1)^2}}
\label{moment2}
\end{equation}

\begin{equation}
\mu_3=\sqrt[3]{\frac{1}{n}\sum_{k=1}^{n}{(d_k-\mu_1)^3}}
\label{moment3}
\end{equation}

After the 3D shape of a molecule is semantically encoded into a 12-element moment vector $\mathbf M=(\mu_1^{ctd}, \mu_2^{ctd}, \mu_3^{ctd}, \mu_1^{cst}, \mu_2^{cst}, \mu_3^{cst}, \mu_1^{fct}, \mu_2^{fct}, \mu_3^{fct}, \mu_1^{ftf}, \mu_2^{ftf}, \mu_3^{ftf})$, the dissimilarity score of two molecules $\mathbf M^i$ and $\mathbf M^j$ can be defined as the sum of absolute differences of their respective moments, much like the city block distance. Thereafter, this dissimilarity is monotonically inverted using equation \eqref{usrscore} so as to get transformed to a normalized similarity score, where the minimum shape similarity is represented by score 0 and the maximum similarity is represented by score 1. Any other inverse monotonic function, such as cosine or L2-norm inversion, can do this transformation if it preserves the ranking order. Equation \eqref{usrscore} is favored because it is simple, fast and interpretable, and has fixed upper and lower bounds.

\begin{equation}
S(\mathbf M^i, \mathbf M^j)=(1+\frac{1}{12}\sum_{k=1}^{12}|\mathbf M_k^i-\mathbf M_k^j|)^{-1}\in[0, 1]
\label{usrscore}
\end{equation}

USR \cite{1379} is highly extensible via positioning reference locations \cite{1334,1335}, incorporating higher orders of moments \cite{1333,1337}, expanding coordinate dimensions \cite{1337,1338}, mixing chemical component similarity \cite{1333,1407,1408}, or subdividing atoms into subsets \cite{1436,1331}. USRCAT \cite{1331} extends USR \cite{1379} by separately identifying five pharmacophoric subsets of atoms, which are heavy, hydrophobic, aromatic, hydrogen bond acceptor or donor atoms. Consequently the resulting moment vector is expanded from 12 descriptors to 60, with the first 12 being identical to USR moments. In the case of an empty subset, for example if no hydrogen bond donors are found, the corresponding elements in the moment vector are set to zero. The four reference locations are uniformly derived from heavy atom coordinates and are thus meaningful with respect to the overall shape of a molecule. The five sets of 12 moments are individually scaled by the factors $ow$ for all atoms, $hw$ for hydrophobic atoms, $rw$ for aromatic atoms, $aw$ for hydrogen bond acceptors and $dw$ for hydrogen bond donors, as shown in equation \eqref{usrcat0score}. Without a prior knowledge, the values of the five scaling factors are all defaulted to one fifth (Equation \eqref{usrcatscore}). USRCAT degenerates to USR when $ow=1$ and $hw=rw=aw=dw=0$.

\begin{eqnarray}
S(\mathbf M^i, \mathbf M^j)&=&(1\nonumber\\
&+&ow\times\frac{1}{12}\sum_{k= 1}^{12}|\mathbf M_k^i-\mathbf M_k^j|\nonumber\\
&+&hw\times\frac{1}{12}\sum_{k=13}^{24}|\mathbf M_k^i-\mathbf M_k^j|\nonumber\\
&+&rw\times\frac{1}{12}\sum_{k=25}^{36}|\mathbf M_k^i-\mathbf M_k^j|\nonumber\\
&+&aw\times\frac{1}{12}\sum_{k=37}^{48}|\mathbf M_k^i-\mathbf M_k^j|\nonumber\\
&+&dw\times\frac{1}{12}\sum_{k=49}^{60}|\mathbf M_k^i-\mathbf M_k^j|)^{-1}\nonumber\\
&\in&[0, 1]
\label{usrcat0score}
\end{eqnarray}

\begin{equation}
S(\mathbf M^i, \mathbf M^j)=(1+\frac{1}{60}\sum_{k=1}^{60}|\mathbf M_k^i-\mathbf M_k^j|)^{-1}\in[0, 1]
\label{usrcatscore}
\end{equation}

\subsection{USR-VS}

This subsection describes the methodological innovations of our USR-VS from three perspectives: molecular database, optimized implementation, and result visualization.

\subsubsection{Molecular database}

In 2014 we deployed our in-house docking web server idock@istar \cite{1362} for prospective structure-based virtual screening, accompanied by a large database of 23,129,083 small molecules collected from the All Clean Subset of ZINC \cite{532,1178}. We reused the same database in this project.

It is helpful to include more than one conformer per compound in the database since flexible molecules can adopt different shapes. Hence the more of these conformations included in the database, the less likely it is to miss molecules with the desired pattern. Each small organic molecule could have an average of about 200 conformations \cite{1332}, or up to 292 conformations \cite{1280}. The conformers of a particular molecule are in general geometrically distinct and have low potential energy, as conformers with high internal energy are in principle less likely to exist in nature.

There are numerous 2D-to-3D conversion tools that can generate 3D molecular conformations from a considered 2D chemical structure, such as Cyndi \cite{1393,1394} and OMEGA \cite{462}. A study \cite{1127} examined the performance of four freely available small molecule conformer generation tools, Balloon \cite{1442}, Confab \cite{1443}, Frog2 \cite{1444}, and RDKit (http://www.rdkit.org/), alongside a commercial tool, MOE (http://www.chemcomp.com/), and found that RDKit and Confab were statistically better than other methods at generating low RMSD (Root Mean Square Deviation) conformers to the known structure, and RDKit resulted as the second fastest method after Frog2. These positive results for RDKit in terms of accuracy and speed make it a valid free alternative to commercial, closed source, proprietary software.

However, even though the conformers generated by RDKit can be ensured to be at least a certain RMSD threshold apart from each other by setting an appropriate parameter, they are not guaranteed to be of low energy, so it is suggested by the manual to energy minimize them using RDKit's implementation of the Universal Force Field (UFF). After energy minimization, unfortunately, some conformers could fall into the same local energy minimum and again become structurally similar to each other with RMSD below the threshold. To resolve this conformational diversity problem, the study \cite{1127} also described a postprocessing algorithm to discriminate and keep only conformers which are both energy minimized and a certain RMSD threshold apart. In the postprocessing algorithm, the energy minimized conformers are first sorted by increasing energy value and the lowest energy conformer is retained, and then for each of the remaining conformers, it will be discarded when its RMSD from any conformers that have been retained is smaller than a fixed threshold, or it will be retained otherwise. Using RDKit in combination with the postprocessing algorithm, one can quickly build a diverse and representative set of conformers.

Due to the superior accuracy and speed of RDKit, in this study we used RDKit together with the postprocessing algorithm to generate conformers. Precisely, we used RDKit version 2015.03.1 and programed against its C++ API instead of directly using the example Python script. After conformer generation, totally 93,903,333 low-energy conformers were retained for 23,129,049 molecules, whereas the other 34 molecules failed this task. Hence on average there were four conformers for each molecule, although only one conformer was retained for more than six million molecules, and as many as 35 conformers were retained for some molecules (Figure \ref{nconfs}).

\begin{figure}
\begin{center}
\includegraphics[width=\linewidth]{../usr/nconfs.eps}
\end{center}
\caption{Distribution of number of retained conformers of the 23M molecules.}
\label{nconfs}
\end{figure}

It has been discussed that molecules with a larger number of rotatable bonds are likely to produce more conformers \cite{1127}. We observed that this is generally true when the number of rotatable bonds is no more than 15 (Figure \ref{nrb-nconfs}).

\begin{figure}
\begin{center}
\includegraphics[width=\linewidth]{../usr/nrb-nconfs.eps}
\end{center}
\caption{Boxplot of number of retained conformers of the 23M molecules plotted against their number of rotatable bonds.}
\label{nrb-nconfs}
\end{figure}

\subsubsection{Optimized implementation}

As a result of conformer generation, the size of the USRCAT descriptors of all conformers is $8B*60*93903333=42GB$ since each conformer has a USRCAT moment vector of 60 elements of double precision floating point type of 8 bytes. Given that nowadays servers are typically equipped with at least 64GB memory, it is possible to preload all descriptors to avoid reloading them every time a new query arrives and thus enable high-throughput virtual screening. In this case, the execution of a query ($t_{query}$) is sequentially composed of three parts, which are calculation of USR and USRCAT scores of all conformers ($t_{score}$), sorting of all molecules ($t_{sort}$), and reading and writing of top hits in SDF format ($t_{io}$). Equation \eqref{t_query} summaries the required time for a query. Note that or each molecule, the best USR or USRCAT score of all its conformers is assigned as the score for that molecule.

\begin{equation}
t_{query}=t_{score}+t_{sort}+t_{io}
\label{t_query}
\end{equation}

Considering that a single job may consist of multiple query molecules, and multiple users may submit multiple jobs, there are apparently several levels of parallelism to exploit. In our highly-optimized implementation, two levels of parallelism are exploited. First, in a coarse-grained manner, multiple jobs are distributed to multiple daemon instances running on multiple servers. Second, in a fine-grained manner with multithreading, the entire 23M molecules are divided into independent chunks, which are then distributed to multiple threads for concurrent scoring and sorting. Finally the top hits from each chunk are combined and sorted to provide the final top hits for a query. The number of chunks is set to be proportional to the number of CPU cores in order to achieve a high utilization of the CPU.

In the calculation of USR and USRCAT scores (Equations \eqref{usrscore} and \eqref{usrcatscore}), given that the similarity score is a monotonic inverse of the dissimilarity score, finding the highest similarity scores is equivalent to finding the lowest dissimilarity scores. Hence, to accelerate scoring, only the dissimilarity scores are calculated for the chunks, and they are monotonically inverted to calculate the similarity scores only for the final top hits. Furthermore, given that the dissimilarity score of USRCAT is a sum of 60 absolute differences, a short circuit is implemented such that if the sum of the first few absolute differences of the current conformer already exceed the sum of all 60 absolute differences of the previous conformer, the current conformer is immediately discarded.

\subsubsection{Result visualization}

The top hits are lastly visualized in a 3D WebGL canvas using iview \cite{1366}, allowing users to interactively translate, rotate, zoom in/out the 3D structure without using Flash or Java plugins as well as inspecting USR and USRCAT similarity scores, chemical properties and suppliers. The use of iview \cite{1366} benefits from GPU hardware rendering, and supports labeling atoms and exporting the canvas to a PNG image.

\section{RESULTS}

The entire process of a query is completed in less than 2 seconds, thus reaching a formidable screening speed of 55 million 3D conformers per second.

\section{DISCUSSION}



\section{CONCLUSION}

Searching for compounds that resemble the shape of a given query molecule is a widely seen yet daunting problem in ligand-based virtual screening \cite{1332,1380,1281,1504,1502,1615} and macromolecular target prediction \cite{1407,1408,1402}. The USR-like methods \cite{1379,1338,1331} represent an entirely new class of non-superposition algorithms that effectively capture the molecular shape information independent of spatial position and orientation. These methods circumvent the requirement of structural alignment and show outstanding computational efficiency with respect to superposition-based methods \cite{1440,887,1439}.

In this study we have reviewed the traditional USR method \cite{1379} and its various extensions \cite{1333,1436,1437,1334,1335,1337,1338,1331,1407,1408,1675}, as well as their applications, both retrospective \cite{1332,1331} and prospective \cite{1505,1380,1281,1504,1502,1615}. We have highlighted the pros and cons of three web servers \cite{1436,1437,1408} which use USR variants as their underlying methods. Consequently, to address the existing constraints, we have developed a highly-optimized implementation of USR \cite{1379} and USRCAT \cite{1331} and explained its methodological advancement in terms of functionality, usability, and efficiency.

First, our molecular database has been populated with more than 23 million small and diverse molecules that are collected from the ZINC database \cite{532,1178} and thus possibly commercially available to purchase for subsequent biological assays. We have generated 94 million low-energy conformers that are likely to occur in nature. The reason why a molecular database should be populated with multiple conformers of each flexible compound is to reduce the possibility of missing compounds with similar shape to the query. The USR and USRCAT features of these 94 million conformers have been precalculated, which was a one-off exercise.

Second, we have exploited two levels of parallelism to distribute multiple jobs to multiple servers and to distribute scoring and sorting to multiple threads. In this manner, all available computational resources are fully utilized so as to accelerate job execution.

Moreover, our USR-VS integrates our iview \cite{1366} WebGL visualizer to aid result interpretation in an interactive manner.

We believe our USR-VS web server for ligand-based virtual screening purpose can perfectly supplement our idock@istar \cite{1362} web server for structure-based virtual screening purpose. We suggest users try both servers to reach a consensus.

\section{ACKNOWLEDGEMENTS}

We gratefully acknowledge the Direct Grant from the Chinese University of Hong Kong, the GRF Grant (Project Reference 414413) from the Research Grants Council of Hong Kong SAR, the Inserm funding (P.J.B.), and the Medical Research Council for a Methodology Research Fellowship (Grant No. G0902106, awarded to P.J.B.).

\subsubsection{Conflict of interest statement.} None declared.
\newpage

\bibliographystyle{natbib} % Style BST file
\bibliography{../refworks} % Bibliography file (usually '*.bib' )

\end{document}
